The European Railway agency provides a test specification
\cite{ETCS-Subset076} along with the requirements of the EVC.  The
tests aim at verifying the conformity and the functionality of the onboard 
subsystems against the system requirement specification (SRS)
\cite{ETCS}.  The SRS has been decomposed into a set of features such
that a feature groups a set of requirements that can be tested at the
available interfaces. The test cases have been designed to ensure that
a test of a given feature is independent of all other
feature. Moreover tests are only described as a reaction to a given
stimulation at the interface of the feature.

The test specification formalises the test cases
description, each test case is composed of
\begin{itemize}
\item  a unique identifier,
\item  a short description of the target of the test,
\item  a list of covered requirement,
\item an initial state of the test e.g. initial assignment to the
internal test variables,
\item the test  steps:  inputs change and expected outputs, and
\item a final test state e.g. initial assignment to the
internal test variables.
\end{itemize}


Table~\ref{tbl:manualtest} summarizes the available
tests. The second column describes the objective of each test.  The
column ``Covers Requirements'' shows the list of requirements covered
by the tests. We have refined the list provided by the standard. We
first refer to the sub-requirement when needed and secondly, we add
some requirements that are covered by side effect  to be able to
provide more accurate requirement coverage.

The ceiling speed monitoring feature is associated with 8 test cases
in~\cite{ETCS-Subset076}.  They cover the change of the speed
supervision status and the brake commands depending on  the train
speed. We only use seven of them (TC-CSM-[1-7]), the one missing
covers a requirement that we assume, is “delegated” to the surrounding software
of the CSM.
Eight test cases cover the general requirement of the speed and
distance monitoring feature in~\cite{ETCS-Subset076}. Four of those
are test cases outside the scope of the ceiling speed monitoring and
therefore not included in our benchmark.
The four others cover the requirements {\sf REQ-3.13.10.2.3} and
{\sf REQ-3.13.10.2.4} but only in the target speed monitoring section. We
have adapted them to fit into the collection of CSM tests. Moreover,
we grouped them in pairs (TC-GR-[1-2]) . Each pair is dealing with two different ways
of receiving inputs depending on the  ERTMS mode that is not
relevant for testing the CSM function.


\begin{table}
\tabsize
\renewcommand{\arraystretch}{1.2}
\caption{Test cases from the Subset 076}
\label{tbl:manualtest}
\begin{tabular}{lp{.48\textwidth}p{4.9cm}}
\hline\hline
Identifier & Target of the test & Covered Requirements \\
\hline
TC-CSM-1 & 
When the train runs in CSM  and does not exceed the
the permitted speed, no intervention is triggered and the 
{\sf NormalStatus} is displayed.& 
REQ-3.13.10.3.1, REQ-3.13.10.3.3,\newline
REQ-3.13.10.3.5 \\ \hline
TC-CSM-2 & 
When the train runs in CSM and the speed is between the permitted
speed and the Warning limit, no intervention is triggered and the 
{\sf OverspeedStatus} is displayed. Once the train speed is below 
the permitted speed the {\sf NormalStatus} is displayed. & 
REQ-3.13.10.3.1, REQ-3.13.10.3.2,\newline
REQ-3.13.10.3.3, REQ-3.13.10.3.4 \\ \hline
TC-CSM-3 & 
When the train runs in CSM and the speed is between the Warning limit
speed and the SBI supervision limit, no intervention is triggered and the 
{\sf WarningStatus} is displayed. Once the train speed is below 
the permitted speed the {\sf NormalStatus} is displayed. & 
REQ-3.13.10.3.1, REQ-3.13.10.3.2, \newline
REQ-3.13.10.3.3, REQ.3.13.10.3.4.r4 \\ \hline
TC-CSM-4 & 
When the train runs in CSM and the speed is between the SBI
supervision limit and
the EBI supervision limit , the service brake intervention is triggered and the 
{\sf InterventionStatus} is displayed. Once the train speed is below 
the permitted speed the {\sf NormalStatus} is displayed. & 
REQ-3.13.10.2.2, REQ-3.13.10.3.1,\newline
REQ-3.13.10.3.2, REQ-3.13.10.3.3,\newline 
REQ-3.13.10.3.4 \\ \hline
TC-CSM-5 & 
When the train runs in CSM and the speed is greater then 
the EBI supervision limit , the emergency brake intervention is triggered and the 
{\sf InterventionStatus} is displayed. Once the train reaches
standstill the {\sf NormalStatus} is displayed. & 
REQ-3.13.10.2.2, REQ-3.13.10.2.5, \newline
REQ-3.13.10.3.1, REQ-3.13.10.3.2, \newline
REQ-3.13.10.3.3, REQ-3.13.10.3.4 \\ \hline
TC-CSM-6 & 
When entering CSM mode the Indication status is overwritten &
REQ-3.13.10.3.1, REQ-3.13.10.3.3,\newline
REQ-3.13.10.3.4, REQ-3.13.10.3.6 \\ \hline
TC-CSM-7 & 
When the train is between the permitted speed and the Warning limit
the  SBI  supervision limit also referred
as the FLOI (First Line Of Intervention) is 
displayed. Once the train speed is below 
the permitted speed the {\sf NormalStatus} is displayed  & 
REQ-3.13.10.3.1, REQ-3.13.10.3.2,\newline
REQ-3.13.10.3.3, REQ-3.13.10.3.4, \newline
REQ-3.13.10.3.5\\ \hline
TC-GR-1 & 
When the use of service brake is not allowed the emergency brake
command shall be triggered instead. The emergency brake is then revoked
according to the service brake revocation.& 
REQ-3.13.10.2.1, REQ-3.13.10.2.3,\newline
REQ-3.13.10.2.4, REQ-3.13.10.3.3,\newline
REQ-3.13.10.3.4 \\ \hline
TC-GR-2 & 
The use of service brake is not allowed and the train exceeds the EBI
supervision limit, the emergency brake command shall be triggered. 
The emergency brake is then revoked only when the train reaches
standstill.&
REQ-3.13.10.2.1, REQ-3.13.10.2.3,\newline
REQ-3.13.10.2.4, REQ-3.13.10.3.3,\newline
REQ-3.13.10.3.4 \\ 
\hline\hline
\end{tabular}
\normalsize
\end{table}




These 9 test cases has been translated into LTL formula to fit our
experiments platform. This translation is performed by representing
the steps by a sequence of nested until operator.
Let the test sequence be of the following form: $i_0,i_1,o_0,i_2, o_1,o_2$ where
$i_k$ ($o_k$) represents an input (resp. output) assignment to an interface
variables. The associated LTL formula will be in the following form:\\
$$ \mathsf{Finally}( [i_0 \wedge i_1 \wedge o_0] \wedge ([i_0 \wedge i_1 \wedge
o_0] \; \mathsf{Until} \;[i_2 \wedge \mathsf{Finally}([o_1 \wedge o_2]) ) $$
Intuitively the inputs are set and do not change until a new input configuration
is reached and implies new  output values.

\begin{figure}
\begin{Verbatim}[numbers=left]
TC-CSM-2;
 Finally ([currentSpeed > 0 && 
           currentSpeed < V_mrsp &&
           SpeedSupervisionStatus == NormalStatus && 
           displayPermittedSpeed == 1 ]
          &&
          ([currentSpeed > 0 && 
            currentSpeed < V_mrsp &&
            SpeedSupervisionStatus == NormalStatus && 
            displayPermittedSpeed == 1 ]
          Until
          (([currentSpeed > V_mrsp && 
             currentSpeed <= V_mrsp + dV_Waring(V_mrsp)]  && 
             Finally[
                     SpeedSupervisionStatus ==  OverspeedStatus && 
                     displayPermittedSpeed == 1 && 
                     displaySBI == 1 && 
                     ServiceBrakeCommand == 0 ])
              &&
              ([currentSpeed > V_mrsp  && 
                currentSpeed <= V_mrsp + dV_Waring(V_mrsp) ]   
              Until
              ([currentSpeed > 0 && currentSpeed < V_mrsp ] &&
                Finally[SpeedSupervisionStatus == NormalStatus ])))));

\end{Verbatim}
\caption{Example of test case TC-CSM-2 as LTL formula\label{fig:testltlex}}
\end{figure}

Figure \ref{fig:testltlex} shows the LTL formula used to represent the
test case TC-CSM-2. 
\begin{itemize}
\item lines 2-5~: The train is moving within the permitted speed and
the DMI aspects is coherent with the normal status.
\item lines 7-13~: The train stays in the initial configuration until
the speed is greater than permitted speed but below the Warning limit.
\item lines 14-21~: The DMI information changes according to the new
speed.
\item line 23~: The train speed is now back below the permitted.
\item line 24~: The Normal status is now displayed.
\end{itemize}



%%  LocalWords:  EVC EBI WarningStatus InterventionStatus SBI CSM FLOI  REQ
%%  LocalWords:  DMI
