

\subsubsection{Verification of the Procedure On-Sight (Systerel) }
\label{sec:}

\paragraph{Contributing project partners}
% Usually, one main partner, perhaps with contributions from others
The work has been performed by Systerel.

\paragraph{Process step}
% Classification of the activity according to D2.3a
% Name what is verified and to which report this would contribute.
% Use the numbers, e.g. System Design Verification Report (1-12).


This activity contributes to:
\begin{itemize}
\item the Software Requirement Specification  (3.16)
\item the Software Specification Verification Report  (3.18)  
\item the Software Architecture and Design Specification (4.19)
\item the SW Component Test Specification (4-22)
\item the Software Design Verification Report  (4.23)  
\item the Software Components (5.24)
\item the Software Component Verification Report  (5.26)  
\end{itemize}



\paragraph{Object of verification}
% Which openETCS artifact (github link) or other documents/programs
% etc. (provide references)

The object of verification is a formal model in Classical~B for the
procedure On-Sight. It is available at
{\url{https://github.com/openETCS/model-evaluation/tree/master/model/Classical_B_Systerel/obu_classicalB}}.


\paragraph{Available specification}
% The specification against which the object is to be checked. Usually
% coming form some openETCS artifacts (GitHub reference, process
% artifact number) or background material (reference, artifact number).

The model implements the requirements for the procedure On-Sight, as
described in Subset~026 \cite[Sec.~5]{subset-026:3.3.0}.



\paragraph{Objective}
%In an ordinary development, the main objective would be to verify or validate
%something. Here, in openETCS, it would be quite common to demonstrate
%the applicability of some method/tool, or evaluate its capabilities.


The goal is to produce a B model which implement the On-Sight
procedure.  The B model shall be formally proved (verified) and shall
allow to generate an executable C code. The main objective of the
activity is to show the applicability of the B method and its
implementing and supporting tools to the development task.



\paragraph{Method/Approach}
% Short description of how the verification/validation is performed

At first, a formal model is defined with the B method using the
Atelier~B tool.  Then the model is formally verified by proof (typing
and value constraints) and model-checking (additional checks of
invariants and freedom of deadlocks).  Functional properties can also
be defined and validated by proof or model-checking on the B model.

Finally, C code is automatically generated from the B model. To check
consistency of a modified or additionally provided implementation
model, tests for checking conformance to the formal model are generated

\paragraph{Means/Tools}
% Could be integrated with the previous paragraph. Assign an
% approrpiate tool class (T1, T2, T3) according to EN 50128. Try to assign a
% maturity level 
%https://github.com/openETCS/validation/blob/master/VerificationAndValidationPlan/V02/VnVUsrStrTmplt-140709-02.pdf


The means used are:
\begin{itemize}
\item Atelier~B to  design, check, verify and prove the model (qualified by industrial railway actors)
\item ProB to perform model-checking
\item Atelier~B translators to produce C code (Code translator shall be T3 level to obtain certified code).
\end{itemize}


\paragraph{Results}
% Results related to the objective.
% Refer to appropriate document (preferably GitHub) for more complete description.


\begin{itemize}
\item The result is a fully formal model of the procedure On-Sight.
\item The C code has been automatically generated.
\item typing and value constraints have been formally proved for the model.
\item Some properties have been checked by model checking.
\end{itemize}

See
{\url{https://github.com/openETCS/validation/blob/master/Reports/D4.3/D4.3.1-Final-VV-report-on-model/D4.3.1.pdf}}
and
{\url{https://github.com/openETCS/validation/blob/master/VnVUserStories/VnVUserStorySysterel/04-Results/b-ClassicalB-VnV/BmodelVnV.pdf}}.



%\paragraph{Observations/Comments}
% An optional section where anything can be included which has been
% observed without direct connection


\paragraph{Conclusion}
%What has been achieved, what is missing, what has been learned

The B~method, along with its verification processes and tools, meets
the goals and activities of the openETCS project in terms of quality,
rigor, safety and credibility.\\ 
To also satisfy the requirement of open proof, an open-source proof
obligation generator is to be developed, and a framework for
proving has to be built. There are  but this is compensated by the fact that work on the subject
is ongoing, and ProB is an effective tool for verification.
