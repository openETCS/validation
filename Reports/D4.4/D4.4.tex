\documentclass{template/openetcs_article}
% Use the option "nocc" if the document is not licensed under Creative Commons
%\documentclass[nocc]{template/openetcs_article}
\usepackage{lipsum,url}
\usepackage{supertabular}
\usepackage{multirow}
\usepackage{color, colortbl}
\definecolor{gray}{rgb}{0.8,0.8,0.8}
\usepackage[modulo]{lineno}
\graphicspath{{./template/}{.}{./images/}}
\begin{document}
\frontmatter
\project{openETCS}

%Please do not change anything above this line
%============================



% The document metadata is defined below

%user specified macros
\input{macros.tex}
%Background color of boxes in process graphic
\definecolor{light-gray}{gray}{0.95}

%assign a report number here
\reportnum{OETCS/WP4/D4.4V0.3}


%define your workpackage here
\wp{Work-Package 4: ``Verification \& Validation Strategy''}

%set a title here
\title{openETCS Final Report on Verification and Validation}

%set a subtitle here
\subtitle{}

%set the date of the report here
\date{December 2015}

%document approval
%define the name and affiliation of the people involved in the
%documents approbation here 
\creatorname{Marc Behrens}                                                                                                                                
\creatoraffil{ Deutsches Zentrum f\"ur Luft und Raumfahrt e.V.}                    
\techassessorname{[assessor name]}
\techassessoraffil{[affiliation]}

\qualityassessorname{Jan Welte}
\qualityassessoraffil{TU Braunschweig}

\approvalname{Klaus-R\"udiger Hase}
\approvalaffil{DB Netz}

%define a list of authors and their affiliation here


\author{Hardi Hungar}
\affiliation{DLR, main editing}
\author{Marc Behrens, Mirko Caspar, Michael M\"onsters}

\affiliation{DLR%\\
%  Lilienthalplatz 7\\
%  38108 Brunswick, Germany
%   \\eMail:hardi.hungar@dlr.de 
}

\author{Jan Peleska, Uwe Schulze}
\affiliation{University Bremen}

\author{Marielle Petit-Doche}
\affiliation{Systerel}

\author{Stefan Rieger}
\affiliation{TWT}

\author{Thorsten Schulz}
\affiliation{University Rostock}







% define the coverart
\coverart[width=350pt]{openETCS_EUPL}


%define the type of report
\reporttype{Final Report}

\begin{abstract}
%define an abstract here
  This document summarizes the approach, scope and result of the
  verification and validation activities in the project openETCS.
\end{abstract}

%=============================
%Do not change the next three lines
\maketitle

%Modification history
%if you do not need a modification history table for your document
%simply comment out the eight lines below 
%=============================
\section*{Modification History}
\tablefirsthead{
\hline 
\rowcolor{gray} 
Version & Section & Modification / Description & Author \\\hline}
\begin{supertabular}{| m{1.2cm} | m{1.2cm} | m{6.6cm} | m{4cm} |}
 0.0 & all & initial & Marc Behrens \\\hline
 0.1 & all & revision and addition & Hardi Hungar \\\hline
 0.2 & all & revision and addition & Hardi Hungar, contributions by
 partners\\\hline
 0.3 & most & added sections for Institut Telecom, included Systerel
 contributions, added TBD comments & Hardi
 Hungar
\\\hline
\end{supertabular}

\tableofcontents
\listoffiguresandtables
\newpage
%=============================
%Uncomment the next line if you need line numbers for tracebility when the document is in review
%\linenumbers

%=============================
% The actual document starts below this line
%=============================

%Start here

\section{Introduction}

According to \cite[3.1.48]{EN50128:2011}, verification is an activity
to check whether the output of a development phase meets the
requirements. This concerns formalities, traceability, and, w.r.t.\
the main content, completeness, correctness and consistency. Within
openETCS, examples of each kind of verification have been
performed. Thereby, also new methods and tools have been evaluated and
adapted. 

Validation concerns the compliance of the end result of the
development with the user requirements. This has been done employing
the demonstrator of the EVC software. 

This document summarizes the activities described in more detail in
separate reports. It explains how these separate activities fit into
the development process of openETCS as defined in the deliverable
D2.3a.    

Most verification activities are actually reviews of documents (or
even programs). For general review activities, a process has been
defined in \cite{openETCS:D1.3.1}. 

\newpage







%Examples are below
\section{Verification and Validation in the Development Lifecycle}
\label{sec:Lifecycle}

\begin{figure}[hbt]
  \centering
  \def\svgwidth{.9\textwidth}
  {\tiny
  \input{Prcss2_3a-03.pdf_tex}}
  \caption{openETCS Development Lifecycle}
  \label{fig:lifecycle2}
\end{figure}

Fig.~\ref{fig:lifecycle2} is an overview of the openETCS development
lifecycle, taken from D2.3a. It depicts the process for a complete
development of the EVC software, of which a part has been performed
within the project. Verification, resp., validation, has to be done in
each of the phases of the development.

\section{Overview of Verification and Validation Activities}
\label{sec:overview}


The verification and validation activities of openETCS fall in two
categories. 
\begin{itemize}
\item They may serve the purpose of supporting the development
of the EVC SW. These are activities as defined in the process
definition D2.3a, actual verification or validation of design
artifacts.
\item They may serve to demonstrate or evaluate methods or tools for
  V\&V. Such methods or tools are applied either to available design
  artifacts, or some such artifacts are created specifically for the
  purpose of the demonstration/evaluation.
\end{itemize}
Both kinds of activities are reported about in this document. It is
structured according to the phases of the development. 

\bgcmmnt{All subsubsectins:
  \begin{itemize}
  \item Add references to corresponding sections
  in D4.3.1 and D4.3.2
\item Address maturity level of tools (\qq{methods too})
  \end{itemize}
 }



\addtocounter{subsection}{-1}
\subsection{Verification and Validation in the Planning Phase}
\label{sec:vnv-0}

There have been reviews of the planning documents. These activities
are not reported in detail, here.

% \cmmnt{\textbf{Template Start}}
% \input{VnV-Activity-Tmplt}
% \cmmnt{\textbf{Template End}}

\subsection{Verification and Validation in the System Design Phase}
\label{sec:vnv-1}

% TWT analyzed sub-system requirements from \cite[Chapter~5]{subset-026:3.3.0}. The
% requirements have been modeled as colored Petri nets and subjected to
% formal analyses. This activity is part of the System Design
% Verification. 

%\newpage
\input{Subset-026-TWT}
\newpage


\subsubsection{Verification of Management of Radio Communication function (Systerel) }
\label{sec:}

\paragraph{Contributing project partners}
% Usually, one main partner, perhaps with contributions from others
The work has been performed by Systerel

\paragraph{Process step}
% Classification of the activity according to D2.3a
% Name what is verified and to which report this would contribute.
% Use the numbers, e.g. System Design Verification Report (1-12).


This activity contributes to:
\begin{itemize}
\item the Elaborated System Requirements (1.07)
\item the Sub-System Requirement Specification (1.10)
\item the System Design Verification Report  (1.12)  
\end{itemize}



\paragraph{Object of verification}
% Which openETCS artifact (github link) or other documents/programs
% etc. (provide references)

The object of verification is the Event-B model for the communication
establishing at {\url{https://github.com/openETCS/model-evaluation/tree/master/model/Event_B_Systerel/Subset_026_comm_session}}. It
is from the strictly formal modeling phase and represents the communication
session management of the OBU.


\paragraph{Available specification}
% The specification against which the object is to be checked. Usually
% coming form some openETCS artifacts (GitHub reference, process
% artifact number) or background material (reference, artifact number).

The model implements the requirements for the communication session management
as described in Subset-026 chapter 3.5.

This section describes the establishing, maintaining and termination of a
communication session of the OBU with on-track systems.


\paragraph{Objective}
%In an ordinary development, the main objective would be to verify or validate
%something. Here, in openETCS, it would be quite common to demonstrate
%the applicability of some method/tool, or evaluate its capabilities.


One goal is the development of a strictly formal, fully proven model of the
communication session management and to provide evidence of covering the
necessary requirements of Subset-026 as well as proving correctness of the model
wrt.\ the requirements and attaining a good coverage of the model wrt.\ the
requirements.

The second goal is to correctly implement the applicable safety requirements
identified by the safety analysis. Both functional and safety requirements
should be traced in the model and a requirement document in a standardized
format.

The formal model will represent the described functionality on the system level,
the correct functioning can be validated by step-wise simulation and
model-checking of deadlock-freeness.


\paragraph{Method/Approach}
% Short description of how the verification/validation is performed

At first, the basic functionality described in the chapter 3.5 that are
identified. These serve as basis for a first abstract model, which is refined
iteratively, adding the desired level of detail. The elements of Subset-026 are
traced using links from Event-B to the ProR file in ReqIf format. Requirements
are formalized as invariants and proven where applicable.

\paragraph{Means/Tools}
% Could be integrated with the previous paragraph. Assign an
% approrpiate tool class (T1, T2, T3) according to EN 50128. Try to assign a
% maturity level 
%https://github.com/openETCS/validation/blob/master/VerificationAndValidationPlan/V02/VnVUsrStrTmplt-140709-02.pdf


The means used are:
\begin{itemize}
\item open source Rodin tool (\url{http://www.event-b.org/}), including plug-ins
  (for details
  see~\url{https://github.com/openETCS/model-evaluation/blob/master/model/Event_B_Systerel/Subset_026_comm_session/latex/subset_3_5.pdf})
\item ProR requirements modeling tool~\url{http://www.pror.org}
\item open source ProB model checker and B model
  simulator~\url{http://www.stups.uni-duesseldorf.de/ProB/index.php5/Main_Page}
\item open source CVC3 (\url{http://www.cs.nyu.edu/acsys/cvc3/}), verIT
  (\url{www.verit-solver.org}) and Alt-Ergo (\url{http://alt-ergo.lri.fr}) SMT
  solvers
\end{itemize}

All the tools are on class T1.

\paragraph{Results}
% Results related to the objective.
% Refer to appropriate document (preferably GitHub) for more complete description.


\begin{itemize}
\item The result is a fully formal model of the communication session management
  as described in chapter 3.5 of Subset-026.
\item Each implemented element of this section is linked to the ProR
  requirements file, both specification elements that describe how something has
  to be done, as well as requirements that describe what must be achieved.
\item The model can be simulated / animated, either with the AnimB or the ProB
  plug-in, validating the functional capabilities.
\item The safety requirements are formalized as invariants in predicate logic,
  their proofs are for the most part fully automatic.
\item It was found that while the Subset-026 communication management explicitly
  allows multiple communication partners (see RBC handover), there is no
  explicit limit of established communication connections given in chapter 3.5.
\item A complete covering of the elements of Subset-026 was not realized, e.g.,\
  there is a representation of the contents of a message, but its explicit
  format is not implemented. This is considered an implementation detail without
  influence for a system level analysis. In general, Event-B models will not be
  refined up to the implementation level.
\end{itemize}

See {\url{https://github.com/openETCS/validation/blob/master/Reports/D4.3/D4.3.1-Final-VV-report-on-model/D4.3.1.pdf}} and {\url{https://github.com/openETCS/validation/blob/master/VnVUserStories/VnVUserStorySysterel/04-Results/d-EventB-VnV/EventB-Rodin-VnV.pdf}}.



%\paragraph{Observations/Comments}
% An optional section where anything can be included which has been
% observed without direct connection


\paragraph{Conclusion}
%What has been achieved, what is missing, what has been learned

Having an abstract formal model of the implemented functionality which can be
simulated, allows for interesting insights into the overall functioning of a
system. Formalized requirements are very helpful in both the identification of
ambiguous requirements and in their clarification.

The elements of Subset-026 are of very different nature. Some describe rather
low-level specification details, other describe ``real'' requirements. Without
an analysis as done with this Event-B model, it can be difficult to decide which
elements must be considered on a system level analysis and which on the lower
implementation level.


\newpage


\subsubsection{Verification of the Chapter~3.8 of Subset~026 (Institut T\'el\'ecom)  }
\label{sec:Subset-026-IT}

\paragraph{Contributing project partners}
% Usually, one main partner, perhaps with contributions from others

This work has been performed by (Institut T\'el\'ecom)

\paragraph{Process step}
% Classification of the activity according to D2.3a
% Name what is verified and to which report this would contribute.
% Use the numbers, e.g. System Design Verification Report (1-12).

This activity is part of the verification of the Elaborated System
Requirements which are based on Subset~026 \cite{subset-026:3.3.0}. It
contributes to the System Design Verification Report (1-12). In
formalizing and analyzing the procedures its findings contribute also
to the definition of the Elaborated System Requirements themselves
(1-07).

\paragraph{Object of verification}
% Which openETCS artifact (github link) or other documents/programs
% etc. (provide references)

The object of verification is the concept of the movement authority
(MA) as defined in Chapter~3.8 of Subset~026
\cite[Sec.~3.8]{subset-026:3.3.0} and a model derived from that
description.

\paragraph{Available specification}
% The specification against which the object is to be checked. Usually
% coming form some openETCS artifacts (GitHub reference, process
% artifact number) or background material (reference, artifact number).
The specification consists of safety properties relating to the
movement authority.

\paragraph{Objective}
%In an ordinary development, the main objective would be to verify or validate
%something. Here, in openETCS, it would be quite common to demonstrate
%the applicability of some method/tool, or evaluate its capabilities.

This activity shall demonstrate the applicability and usefulness of
the IF model checker for analyzing abstract models. As a by-product,
parts of the specification in Subset~026 are checked.   


\paragraph{Method/Approach}
% Short description of how the verification/validation is performed

The  control mechanism of movement authorities
and the movement of the train  in a simplified environment is modeled
by an extended timed state machine. The safety properties express that
the train stays within the bounds set by the movement authorities.

\paragraph{Means/Tools}
% Could be integrated with the previous paragraph. Assign an
% approrpiate tool class (T1, T2, T3) according to EN 50128. Try to assign a
% maturity level 
%https://github.com/openETCS/validation/blob/master/VerificationAndValidationPlan/V02/VnVUsrStrTmplt-140709-02.pdf

The approach has been realized with the IF-Tools \cite{IFTools14}.

\paragraph{Results}
% Results related to the objective.
% Refer to appropriate document (preferably GitHub) for more complete
% description.

The safety properties have been verified on the model. During
modeling, three inconsistencies, ambiguities and gaps in the
specification have been identified. They have been reported
in~\cite{specfindingsTSP}.

\paragraph{Observations/Comments}
% An optional section where anything can be included which has been
% observed without direct connection

The activity has been published in \cite{NgCa14}.

\paragraph{Conclusion}
%What has been achieved, what is missing, what has been learned

Formal modeling and formal verification can be applied to improve the
informal ETCS specification. 




\addtocounter{subsection}{2}
% \subsection{Verification and Validation in the Sub-System Architecture Design Phase}
% \label{sec:vnv-2}

% No verification to be attributed to this phase has been performed in openETCS.


% \subsection{Verification and Validation in the SW Specification Phase}
% \label{sec:vnv-3}

% No verification to be attributed to this phase has been performed in openETCS.

\subsection{Verification and Validation in the SW Design Phase}
\label{sec:vnv-4}



\subsubsection{Verification of the openETCS Architecture and Design Specification}
\label{sec:}

\paragraph{Contributing project partners}
This work has been performed by the DLR.

\paragraph{Process step}
This activity is part of the verification of the openETCS SW
Architecture and Design Specification (4-19), ADD. It contributes to the SW
Design Verification Report (4-23).

\paragraph{Object of verification}
The object of verification is D3.5.3, the openETCS Architecture and
Design Specification.

\paragraph{Available specification}
The ADD is checked against
Subset~026 \cite{subset-026:3.3.0}.

\paragraph{Objective}
The objective is to check that the procedures of ETCS OBU are completely,
correctly and consistently mapped to the components of the SW as
described in the ADD document.

\paragraph{Method/Approach}
The verification has been performed by comparing the corresponding
specifications of Subset~026 with the ADD document for each relevant
paragraph.

\paragraph{Means/Tools}
The verification has been performed manually.

\paragraph{Results}
The verification uncovered some minor inconsistencies. These have been
reported to be removed in D3.5.4 which revises D3.5.3. 

%\paragraph{Observations/Comments}
% An optional section where anything can be included which has been
% observed without direct connection

\paragraph{Conclusion}
%What has been achieved, what is missing, what has been learned
\cmmnt{TBD}

\newpage
%\input{VnV-OnSight-Sys}


\subsubsection{Verification of the Procedure On-Sight (Systerel) }
\label{sec:}

\paragraph{Contributing project partners}
% Usually, one main partner, perhaps with contributions from others
The work has been performed by Systerel.

\paragraph{Process step}
% Classification of the activity according to D2.3a
% Name what is verified and to which report this would contribute.
% Use the numbers, e.g. System Design Verification Report (1-12).


This activity contributes to:
\begin{itemize}
\item the Software Requirement Specification  (3.16)
\item the Software Specification Verification Report  (3.18)  
\item the Software Architecture and Design Specification (4.19)
\item the SW Component Test Specification (4-22)
\item the Software Design Verification Report  (4.23)  
\item the Software Components (5.24)
\item the Software Component Verification Report  (5.26)  
\end{itemize}



\paragraph{Object of verification}
% Which openETCS artifact (github link) or other documents/programs
% etc. (provide references)

The object of verification is a formal model in Classical~B for the
procedure On-Sight. It is available at
{\url{https://github.com/openETCS/model-evaluation/tree/master/model/Classical_B_Systerel/obu_classicalB}}.


\paragraph{Available specification}
% The specification against which the object is to be checked. Usually
% coming form some openETCS artifacts (GitHub reference, process
% artifact number) or background material (reference, artifact number).

The model implements the requirements for the procedure On-Sight, as
described in Subset~026 \cite[Sec.~5]{subset-026:3.3.0}.



\paragraph{Objective}
%In an ordinary development, the main objective would be to verify or validate
%something. Here, in openETCS, it would be quite common to demonstrate
%the applicability of some method/tool, or evaluate its capabilities.


The goal is to produce a B model which implement the On-Sight
procedure.  The B model shall be formally proved (verified) and shall
allow to generate an executable C code. The main objective of the
activity is to show the applicability of the B method and its
implementing and supporting tools to the development task.



\paragraph{Method/Approach}
% Short description of how the verification/validation is performed

At first, a formal model is defined with the B method using the
Atelier~B tool.  Then the model is formally verified by proof (typing
and value constraints) and model-checking (additional checks of
invariants and freedom of deadlocks).  Functional properties can also
be defined and validated by proof or model-checking on the B model.

Finally, C code is automatically generated from the B model. To check
consistency of a modified or additionally provided implementation
model, tests for checking conformance to the formal model are generated

\paragraph{Means/Tools}
% Could be integrated with the previous paragraph. Assign an
% approrpiate tool class (T1, T2, T3) according to EN 50128. Try to assign a
% maturity level 
%https://github.com/openETCS/validation/blob/master/VerificationAndValidationPlan/V02/VnVUsrStrTmplt-140709-02.pdf


The means used are:
\begin{itemize}
\item Atelier~B to  design, check, verify and prove the model (qualified by industrial railway actors)
\item ProB to perform model-checking
\item Atelier~B translators to produce C code (Code translator shall be T3 level to obtain certified code).
\end{itemize}


\paragraph{Results}
% Results related to the objective.
% Refer to appropriate document (preferably GitHub) for more complete description.


\begin{itemize}
\item The result is a fully formal model of the procedure On-Sight.
\item The C code has been automatically generated.
\item typing and value constraints have been formally proved for the model.
\item Some properties have been checked by model checking.
\end{itemize}

See
{\url{https://github.com/openETCS/validation/blob/master/Reports/D4.3/D4.3.1-Final-VV-report-on-model/D4.3.1.pdf}}
and
{\url{https://github.com/openETCS/validation/blob/master/VnVUserStories/VnVUserStorySysterel/04-Results/b-ClassicalB-VnV/BmodelVnV.pdf}}.



%\paragraph{Observations/Comments}
% An optional section where anything can be included which has been
% observed without direct connection


\paragraph{Conclusion}
%What has been achieved, what is missing, what has been learned

The B~method, along with its verification processes and tools, meets
the goals and activities of the openETCS project in terms of quality,
rigor, safety and credibility.\\ 
To also satisfy the requirement of open proof, an open-source proof
obligation generator is to be developed, and a framework for
proving has to be built. There are  but this is compensated by the fact that work on the subject
is ongoing, and ProB is an effective tool for verification.

\newpage

\subsubsection{Model-based Test Generation for the ETCS Ceiling Speed Monitor}
\label{sec:csmunibremen}

\paragraph{Contributing project partners}
% Usually, one main partner, perhaps with contributions from others
Main contribution by University of Bremen, additional contributors: DLR and Siemens 

\paragraph{Process step}
% Classification of the activity according to D2.3a
% Name what is verified and to which report this would contribute.
% Use the numbers, e.g. System Design Verification Report (1-12).
This activity is part of the SW Design (Phase 4). It
contributes to the SW Component Test
Specification (4-22). 

\paragraph{Object of verification}
% Which openETCS artifact (github link) or other documents/programs
% etc. (provide references)
The object of verification are implementations of the ETCS
Ceiling Speed Monitor (CSM).  


\paragraph{Available specification}
% The specification against which the object is to be checked. Usually
% coming form some openETCS artifacts (GitHub reference, process
% artifact number) or background material (reference, artifact number).

The specification of speed and distance monitoring in \cite[Sec.~3.13]{subset-026:3.3.0}.

\paragraph{Objective}
%In an ordinary development, the main objective would be to verify or validate
%something. Here, in openETCS, it would be quite common to demonstrate
%the applicability of some method/tool, or evaluate its capabilities.

The main objective is to evaluate and demonstrate the new input
equivalence class partition test generation method developed by the
team of the University of Bremen. The method guarantees 100 per cent
error detection inside a fault domain, and is expected to provide high
coverage outside the domain. Its results on the CSM are compared with the
relevant system test cases as defined in then ETCS standard conformity
test specification, Subset~076.


\paragraph{Method/Approach}
% Short description of how the verification/validation is performed

A test model specifying the expected behaviour of the CSM has been
developed in SysML, using state machines and block diagrams.  The
model elements have been linked to the associated ETCS system
requirements.  Since this SysML language subset can be associated with
a formal semantics, it is possible to execute algorithms that
automatically generate sets of executable test cases from the
model. These sets of test cases permit to check implementations for
compliance with the model. The tracing information enable to derive
detailed coverage and fault identification information.

The existing SUBSET-076 test cases were formalised using linear
temporal logic (LTL), so that the same test data generation concept
could be applied as for the test cases that were automatically
identified: SUBSET-076 test cases do not provide concrete test data
for every test step, but specify the general constraints from which
concrete data can be elaborated.  This approach also allows to trace
the model coverage achieved by the SUBSET-076 test cases.

All tests were executed against software mutants derived from a
reference implementation, using 3 different mutation generators in
order to avoid a mutation bias. For each testing strategy applied it
was checked
\begin{itemize}
\item which parts of the test model were covered by the test execution, and
\item which fault coverage (percentage of ``killed'' mutants) was achieved.
\end{itemize}





\paragraph{Means/Tools}
% Could be integrated with the previous paragraph. Assign an
% approrpiate tool class (T1, T2, T3) according to EN 50128. Try to assign a
% maturity level 
%https://github.com/openETCS/validation/blob/master/VerificationAndValidationPlan/V02/VnVUsrStrTmplt-140709-02.pdf

The whole approach is fully supported by RT-Tester and its model-based
testing component RTT-MBT. Test cases are described by LTL
formulas. An integrated SMT-solver generate solutions for the LTL
formulas which add  concrete data and makes the test cases
executable. From the SysML test
model, the tool automatically derives LTL
formulas which describe the test cases. For the SUBSET-076 test cases,
the LTL formulas have been provided manually and completed by the
solver. 

\paragraph{Results}
% Results related to the objective.
% Refer to appropriate document (preferably GitHub) for more complete description.
The results can be summarised as follows.
\begin{enumerate}
\item The new equivalence class testing method shows significantly
  higher test strength than all other methods used in the
  comparison. It achieved nearly 100\% fault coverage for mutants
  outside the fault domain (mutants inside the fault domain are always
  killed, due to the guaranteed fault detection properties).

\item The new method is very well suited for software testing and
  HW/SW integration testing, where the high number of test cases
  (approx.~5000 cases) can easily be executed, in particular, because
  the test suite is fully automated. The new method, however, yields
  too many test cases to be applied on system testing level with real
  trains on real tracks.


\item The SUBSET-076 test cases are missing 2 cases for the CSM in order to achieve
requirements coverage. These can be easily identified and added. As a result,
these test comprise 11 cases.

\item With the missing test cases added, the SUBSET-076 achieve only a fault coverage of
62\% -- this would certainly not suffice to obtain certification credit. It is
possible, however, to add an acceptable number of test cases to the SUBSET-076
suite for the CSM which would significantly increase its test strength.


\end{enumerate}


The results have been published in \cite{PeHu14,HHP14,BHHHPSV14,BHSHPSV14,BHHPS15}.
% \begin{itemize}
% \item Jan Peleska and Wen-ling Huang: Complete model-based equivalence
%   class testing. Int J Softw Tools Technol Transfer. Published online:
%   21 November 2014. DOI 10.1007/s10009-014-0356-8.

% \item Felix H\"ubner, Wen-ling Huang, and Jan Peleska: Experimental
%   Evaluation of a Novel Equivalence Class Partition Testing
%   Strategy. In Jasmin Christian Blanchette and Nikolai Kosmatov
%   (eds.): Tests and Proofs - 9th International Conference, TAP 2015,
%   Held as Part of STAF 2015, L'Aquila, Italy, July 22-24,
%   2015. Proceedings. Lecture Notes in Computer Science 9154, Springer,
%   2015, pp. 155-172, doi 10.1007/978-3-319-21215-9\_10.

% \item C{\'e}cile Braunstein, Anne E. Haxthausen, Wen-ling Huang, Felix
%   H\"ubner, Jan Peleska, Uwe Schulze, and Linh Vu Hong: Complete
%   Model-Based Equivalence Class Testing for the ETCS Ceiling Speed
%   Monitor. In S. Merz and J. Pang (eds.): Proceedings of the ICFEM
%   2014. Springer, LNCS 8829, pp. 380-395, 2014. DOI
%   10.1007/978-3-319-11737-9\_25.


% \item Technical Report http://www.informatik.uni-bremen.de/agbs/testingbenchmarks/
% \newline
% openETCS/ceiling-speed-monitoring/testing\_the\_etcs\_csm.pdf


% \item C{\'e}cile Braunstein, Wen-ling Huang, Felix H\"ubner, Jan
%   Peleska, and Uwe Schulze: Evaluation of Model-Based Testing
%   Strategies for the ETCS Ceiling Speed Monitor.  Submitted to
%   Software Testing, Verification and Reliability journal.

% Also available as technical report
% \end{itemize}

\paragraph{Observations/Comments}
% An optional section where anything can be included which has been
% observed without direct connection

It is interesting to note that typical model-coverage driven test
cases (e.g. transition coverage, MC/DC coverage), while achieving
higher model coverage than the SUBSET-076 tests, do not achieve much
higher fault coverage (approx.~68\%).  The reason is that these test
cases are not invariant under syntactic model transformations: with
another -- through semantically equivalent -- model, higher or lower
test strength would be achieved with the coverage-driven test cases
derived from that model.

In contrast to that, the new equivalence class testing strategy is
elaborated from the {\it semantic} representation of the model and is
therefore invariant (i.e.~always maximal) under all syntactic model
transformations that leave the behavioural semantics unchanged.

Verified Systems International GmbH who maintain the commercial
version of RT-Tester have won the runner-up trophy of the EU
Innovation Radar Innovation Prize\footnote{see {\tt
    https://www.verified.de/publications/papers-2015/\newline
    eu-innovation-radar-price-runner-up-trophy-for-verified-systems-international/}}
for implementing the equivalence class testing strategy described
above in the commercial version of RT-Tester.

\paragraph{Conclusion}
%What has been achieved, what is missing, what has been learned

The new test strategy has shown to provide superior test strength when
compared to SUBSET-076 test cases and conventional model-coverage
driven test cases that are typically provided by other model-based
testing tools. As of today, RT-Tester is the only testing tool where
the new test strategy is implemented.


\subsubsection{Model-based Testing of the ETCS Target Speed Monitor}
\label{sec:targetspeedmonitorbremen}

\paragraph{Contributing project partners}
% Usually, one main partner, perhaps with contributions from others
Main contribution by University of Bremen, additional contributors: DLR and Siemens 

\paragraph{Process step}
% Classification of the activity according to D2.3a
% Name what is verified and to which report this would contribute.
% Use the numbers, e.g. System Design Verification Report (1-12).
This activity is part of the SW Design (Phase 4). It
contributes to the SW Component Test
Specification (4-22). 

\paragraph{Object of verification}
% Which openETCS artifact (github link) or other documents/programs
% etc. (provide references)

The object of verification is an implementation of the target speed monitoring function of the EVC,
see \cite{HHP14}

\paragraph{Available specification}
% The specification against which the object is to be checked. Usually
% coming form some openETCS artifacts (GitHub reference, process
% artifact number) or background material (reference, artifact number).
ETCS system specification, SUBSET-026-3; model parts are also
available in \cite{HHP14}. The whole target speed monitoring model
will be made available on \url{http://www.mbt-benchmarks.org}.

\paragraph{Objective}
%In an ordinary development, the main objective would be to verify or validate
%something. Here, in openETCS, it would be quite common to demonstrate
%the applicability of some method/tool, or evaluate its capabilities.
For creating a SysML test model of the target speed monitoring function, both time-discrete
(e.g.~trigger of the emergency brakes) and time-continuous (e.g.~time-dependent 
train location, speed, and acceleration) variables 
need to be considered. SysML state machines
are suitable for modelling concurrent real-time behaviour of time-discrete control
functions. For time-continuous aspects, the report [1] describes how to use
parametric constraints and associated diagrams for modelling. It is also explained
how the parametric specifications are made available to the SMT solver creating 
concrete test data from models. As a result, the solver generates data that 
complies with the time-continuous physical constraints of the model. 


\paragraph{Method/Approach}
% Short description of how the verification/validation is performed

Parametric constraints represent a language aspect of the SysML which has not yet
been fully investigated in the research communities. Using so-called constraint
blocks, these constraints can be specified. Typically, parametric constraints 
represent system invariants or -- this is the relevant aspect for the target speed monitor --
physical laws, such as acceleration-dependent speed and speed-dependent location. 
For our application, these laws also comprise the ETCS braking curves modelling the
speed changes of the braking train.
Parametric constraints can be specified using general physical variables; these are 
bound to concrete model variables using parametric diagrams. 

It is shown in [1] how parametric constraints can be used to calculate physically
meaningful train behaviours, that is, meaningful changes of speed and location over
time, taking into account the braking actions. The method follows a 2-step approach: 
first, a model abstraction is created, and  the equivalence class testing 
strategy described in Section~\ref{sec:csmunibremen} is used to identify
test cases with guaranteed fault detection properties. Next, the calculated tests
are refined with respect to time-dependent behaviour, so that still the same 
equivalence classes  are used, but the representatives for location and speed
are selected in a way that complies with the physical laws.

 


\paragraph{Means/Tools}
% Could be integrated with the previous paragraph. Assign an
% approrpiate tool class (T1, T2, T3) according to EN 50128. Try to assign a
% maturity level 
%https://github.com/openETCS/validation/blob/master/VerificationAndValidationPlan/V02/VnVUsrStrTmplt-140709-02.pdf

The method has been implemented in the RT-Tester tool as part of the WP7-related
activities of the University of Bremen team.

\paragraph{Results}
% Results related to the objective.
% Refer to appropriate document (preferably GitHub) for more complete description.

The results show that the method can be automatically performed with acceptable 
computation time. 

\paragraph{Observations/Comments}
% An optional section where anything can be included which has been
% observed without direct connection

To our best knowledge, this is the first SysML-based method
for calculating test data with guaranteed fault detection properties in presence of
both time-discrete and time-continuous observables.


\paragraph{Conclusion}
%What has been achieved, what is missing, what has been learned

The method developed here is highly relevant for testing cyber-physical systems
in general. Verified Systems International GmbH who maintain the commercial version
of RT-Tester has already decided to make this method available in 2016.
\newpage
\input{TestGen-IT}


\newpage
\subsection{Verification and Validation in the SW Component Phase}
\label{sec:vnv-5}



\input{VnV-Code-Basic}
\newpage
\input{VnV-Code-URO}
\newpage
\input{VnV-SCADETrainData-DLR}
\newpage


\subsubsection{Verification of the Modes and Levels Management Function (Systerel) }
\label{sec:}

\paragraph{Contributing project partners}
% Usually, one main partner, perhaps with contributions from others
The work has been performed by Systerel

\paragraph{Process step}
% Classification of the activity according to D2.3a
% Name what is verified and to which report this would contribute.
% Use the numbers, e.g. System Design Verification Report (1-12).


This activity contributes to:
\begin{itemize}
\item the Software Requirement Specification  (3.16)
\item the Software Specification Verification Report  (3.18)  
\item the Software Architecture and Design Specification (4.19)
\item the Software Design Verification Report  (4.23)  
\item the Software Components (5.24)
\item the Software Component Verification Report  (5.26)  
\end{itemize}



\paragraph{Object of verification}
% Which openETCS artifact (github link) or other documents/programs
% etc. (provide references)

The object of verification is the Scade model for the modes and levels management function at {\url{https://github.com/openETCS/modeling/tree/master/model/Scade/System/ObuFunctions/ManageLevelsAndModes}}. 


\paragraph{Available specification}
% The specification against which the object is to be checked. Usually
% coming form some openETCS artifacts (GitHub reference, process
% artifact number) or background material (reference, artifact number).

The model implements the requirements of the modes and levels management
function, as described in \cite[Sec.~4,5]{subset-026:3.3.0}.
%{\itshape System Requirements Specification, Chapter 4 and 5}. 
Only those parts of the sections which are related to the definition
of the current mode and level are covered.


\paragraph{Objective}
%In an ordinary development, the main objective would be to verify or validate
%something. Here, in openETCS, it would be quite common to demonstrate
%the applicability of some method/tool, or evaluate its capabilities.


The goal is to produce a SCADE model, to automatically generate
executable C code from it, and to verify properties of SCADE model by
model-checking.



\paragraph{Method/Approach}
% Short description of how the verification/validation is performed

At first, a formal model is defined with the Scade suite tool.
Then functional properties are formally verified on the model by  model-checking.

Finally, C code is automatically translated from the Scade model.

\paragraph{Means/Tools}
% Could be integrated with the previous paragraph. Assign an
% approrpiate tool class (T1, T2, T3) according to EN 50128. Try to assign a
% maturity level 
%https://github.com/openETCS/validation/blob/master/VerificationAndValidationPlan/V02/VnVUsrStrTmplt-140709-02.pdf


The means used are:
\begin{itemize}
\item SCADE suite to  design, check and simulate the model 
\item Systerel Smart Solver (S3) to perform model-checking (can be certified as T2 tool)
\item KCG translator to produce C code (Code translator shall be T3 level to obtain certified code).
\end{itemize}


\paragraph{Results}
% Results related to the objective.
% Refer to appropriate document (preferably GitHub) for more complete description.


\begin{itemize}
\item The result is a Scade model of the mode and level management function integrated in the whole EVC scade model.
\item The C code has been automatically  generated.
\item Some properties have been checked by model checking.
\end{itemize}

See {\url{https://github.com/openETCS/validation/blob/master/Reports/D4.3/D4.3.1-Final-VV-report-on-model/D4.3.1.pdf}} and {\url{https://github.com/openETCS/validation/blob/master/VnVUserStories/VnVUserStorySysterel/04-Results/e-Scade_S3/Scade_S3_VnV.pdf}}.



%\paragraph{Observations/Comments}
% An optional section where anything can be included which has been
% observed without direct connection


\paragraph{Conclusion}
%What has been achieved, what is missing, what has been learned



The following benefits of formal methods in an a posteriori verification process of
critical systems have been recognized by our industrial customers.
%
\begin{itemize}
\item Contrary to a human generated test-based verification solution, a formal safety verification is
intrinsically complete. It is equivalent to an exhaustive search for every possible falsification.
\item It clearly identifies the complete list of assumptions upon which the safety relies.
\item A certified solution allows for a reduction of the testing and review efforts (only the generic safety
specification has to be reviewed).
\item The use of formal verification in the qualification of critical software sends a strong
and positive message to the market, and is sometimes even a requirement for some customers.
\end{itemize}


\newpage
\bgcmmnt{To be included: Formal code verification by Fraunhofer}
% \input{VnV-Code-Fraunhofer}

\newpage
\subsection{Verification and Validation in the SW Integration Phase}
\label{sec:vnv-6}

\bgcmmnt{There have been automated integration tests on the SW components.} 

\subsection{Verification and Validation in the SW Validation Phase}
\label{sec:vnv-7}

There have been validations on
\begin{itemize}
\item the integrated software within the \qq{SCADE simulation
    environment}, subjecting the SW with a simulated environment to
  operational use cases.
\item an integration of the SW on a reference hardware, applying
  operational use cases.  
\end{itemize}

bgcmmnt{to be included: validation by SCADE simulation }

\input{VnV-Validation-DLR}

\section{Conclusion}
\label{sec:conclusion}

\bgcmmnt{The conclusion will be written after the completion of the
  V\&V activities.}

%\nocite{*}

\bibliographystyle{unsrt}

\bibliography{bibliography,brmn}



%===================================================
%Do NOT change anything below this line

\end{document}
