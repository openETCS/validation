%\subsection{System Integration  Testing (Uni Bremen/DLR)}

Section \ref{subsec:mbt} describes the main objective of system
integration testing. Uni Bremen will focus on the following items~:
\begin{enumerate}
\item Create a test model in SysML. From the model evaluation
  activities, the management of the radio communication is already
  available. To cover different aspect of the specification, the
  ceiling speed monitoring model will be also provided.
\item Generate test cases according to the defined interface given by
  DLR. (RT-Tester) 
\item Provide simulation environment for the
  track-to-train simulation (including braking curves/speed
  profiles)\footnote{ OpenETCS system testing for the EVC on-board
    computer requires test execution in real physical time and also
    track layout with realistic speed profiles.  We also want to
    contribute to the track and Speed Simulation by automatically
    generates ``relevant'' layout for OpenETCS.}  along routes used
  for testing
\item Study the automatic generation of these track layouts and speed
  simulations; if feasible, implement a generator and integrated it in
  the DLR laboratory environment
\item Set up a test environments for
  \begin{itemize}  
  \item Hardware-in-the-loop Testing within DLR laboratory.
  \item Software-in-loop testing with code provided by SCADE (Siemens)
  \end{itemize}
\end{enumerate}


\paragraph{Track simulation}
OpenETCS system testing for the EVC on-board computer requires test
execution in real physical time and also track layout with realistic
speed profiles.  We also want to contribute to the track and Speed
Simulation by automatically generates ``relevant'' layout for OpenETCS.


\paragraph{Test cases generation}
We also plan different activities to ensure the pertinence of our test cases.
\begin{enumerate}
\item Check the test model (SysML) : RTT-BMC or other model checkers
\item Add relevant LTL properties if needed
\item Test case analysis by 
  \begin{itemize}
    \item Structural coverage
    \item Requirement coverage
    \item Mutation coverage 
    \item Data coverage
  \end{itemize}
\end{enumerate}

\paragraph{Test cases analysis -- comparison to Subset 76}
\begin{enumerate}
\item Provide techniques and Howto describing how test cases from
  Subset 76 can be executed in the RT-Tester environment, either as SW
  integration test or as HiL test in the DLR simulation environment
\item Create new set of test cases for the ceiling speed monitoring
  (As far as we know,they do not yet exist in Subset 76)
\item Compare new test cases created by RT-Tester to new test cases
  for ceiling speed monitoring provided by ERTMS standardization
  group, as soon as available; suggest improvements for the Subset 76
  test cases. 
 \end{enumerate}

%\paragraph{Object Code Verification}
%---> please suggest something!!!

\paragraph{Exchange Formats}
Test models represented in XMI/Ecore are used as SysML test modeling
standard. RT-Tester model parsers are extended to cope with this
format.

Test procedures will be represented in a general abstract syntax
format, so that procedures generated by RT-tester can be run on any
test execution platform.

Test results (test execution logs) will be represented in a general
format, so that exchange of test results between tools (for example,
for simulation purposes) becomes possible.



A first part of these activities have already been done for model of
radio management communication (see the model-evaluation for
\href{https://github.com/openETCS/model-evaluation/blob/master/model/EA-SysML/new_version/doc/ea_sysml_report.pdf}{EA/RT-tester}).