\section{Conclusion}

We have presented a novel approach to automated model-based testing of mixed 
time-discrete and time-continuous systems. It has been shown that SysML is a suitable 
formalism for creating test models of this kind: time-discrete control aspects 
are reflected by state machines, and time-continuous constraints are 
represented by constraint blocks and parametric diagrams. Concrete test cases were generated
in a two-step approach. At first, an abstraction  of the concrete system
was constructed, and an input equivalence class partition strategy with proven fault detection capabilities was applied to generate a test suite of abstract test cases.
The inputs associated with each of these test cases are sequences of input equivalence classes. For the second step, concrete solutions for each of these abstract test cases were generated. For this purpose, each abstract test case is transformed 
into constraints over concrete model variables, and the additional constraints coming
from time-continuous conditions specified in constraint blocks are taken into account as well. 

Using an SMT solver that is capable of processing floating point arithmetics, it was shown that the approach is suitable for practical application. 
To this end, tests for 
a safety-relevant 
control function of the ETCS onboard controller have been generated in an automated way.  
The results show that MBT is feasible for hybrid systems
that are comparable to the complexity of this real-world example. 
The model-based test generation and execution process has been implemented in an experimental version of  an industrial-strength testing tool. As mentioned, we are currently exploring numerical solvers or computer algebra systems to get a better performance and to support more complex mathematical expressions. 

%The model-based test generation and execution process has been implemented in an experimental version of  the  RT-Tester tool~\cite{EPTCS111.1}.\todo[inline]{Replace with prototype implementation in generic tool}