\section{Introduction}
\label{sec:intro}
% ===========================================================================
\subsection{Background}

Model-based testing (MBT) has received much attention over the last years~\cite{Petrenko:2012:MTS:2347096.2347101,DBLP:journals/jss/AnandBCCCGHHMOE13}. In particular, 
we advocate the application of 
the MBT approach for verification and validation (V\&V) of safety-critical control systems. State-of-the-art MBT technology allows for a high degree of automation 
(typically aided by the use of automated constraint solvers), so
that testing experts are relieved of 
the burden of manual test data generation and test procedure development. 
As a consequence, more tests can be executed with acceptable effort 
than would be possible in a scenario where test case and test data generation, as well
as test procedure programming still has to be elaborated manually. Moreover,
automated MBT allows for the application of more complex test case generation strategies,
so that not only the quantity, but also the quality of test cases can be improved.
These benefits make up for the additional effort of having to develop a test model 
as the key enabler for a highly automated testing process. Since the availability
of test models also facilitates the compilation of traceability data -- this aspect will
be described in more detail in the subsequent sections of this article -- 
V\&V  activities for safety-critical systems become considerably more efficient,
while simultaneously increasing the trustworthiness of V\&V results through 
more comprehensive test suites with higher test strength.

Among the more complex test strategies, those possessing \emph{completeness} 
properties are of special interest. Completeness is defined as \emph{soundness}
-- a correct system under test (SUT) never fails a test of a   suite that is sound -- in combination with \emph{exhaustiveness} -- every erroneous SUT fails at least one test case of an exhaustive suite~\cite{DBLP:journals/cn/Tretmans96}. In other words,
complete testing strategies are  able to {\it prove} the correctness of implementations.
In the context of black-box testing, exhaustiveness cannot be unconditionally 
guaranteed by any test suite, because the SUT may contain internal state variables
(so-called ``time bombs'') whose changing values trigger erroneous behaviour  after a certain number of processing steps, and this internal state information is not observable
 during black-box testing. Therefore completeness of black-box test suites can
 only be guaranteed in relation to a \emph{fault model} 
 ${\cal F} = ({\cal S},\preceq,{\cal D})$~\cite{petrenko1996}. Fault models consist of
 a reference model ${\cal S}$ specifying the desired behaviour of implementations and
 a conformance relation $\preceq$ representing the decision criterion  for
 acceptable SUT behaviour: if an implementation behaves like another model ${\cal S}'$,
 its behaviour is accepted if and only if ${\cal S}' \preceq {\cal S}$.
 Finally, the fault domain ${\cal D}$ specifies a collection of models 
 ${\cal S}'\in {\cal D}$ that may or may not conform to the reference model ${\cal S}$.
 Test suites that are exhaustive with respect to ${\cal F}$ guarantee that every
 conformance violation   ${\cal S}' \not\preceq {\cal S}$ will be detected by at 
 least one test case, as long as ${\cal S}'$ is a member of the fault domain ${\cal D}$.
 
 % ===========================================================================
\subsection{Objectives}
Due to their ability to provide conformance proofs, complete test suites are of 
 considerable theoretical value. Their practical application, however, raises two
 critical questions. (Q1) How shall we cope with situations where
  the test suites generated by complete 
 strategies are too large to be  executed with  acceptable effort? 
 (Q2) In practical V\&V campaigns it is hard to determine whether the true behaviour
 of an SUT is reflected by a model ${\cal S}'$ which is still inside the fault domain.
 For ${\cal S}' \not\in{\cal D}$, exhaustiveness of the suite can no longer be guaranteed.
Therefore it remains to be investigated whether simpler model-coverage strategies (these 
are also described in the next sections) perform equally well or even better 
for ${\cal S}' \not\in{\cal D}$, while resulting in smaller test suites.
In the light of these considerations, the main objective of this article is twofold.
\begin{enumerate}
\item Perform an in-depth evaluation of MBT strategies, with special emphasis
on the test strength of model-coverage strategies considered already as ``standard''
in the MBT community,
in comparison to a complete equivalence class partition testing  strategy developed by 
three of the authors~\cite{peleska_sttt_2014,huebner15}.

\item Apply the insight gained in 1 to assess the existing test suite specified for 
the European Train Control System (ETCS) in the standard~\cite{ETCS-Subset076}, which
has been manually designed by domain experts. As a working 
example, the ETCS \emph{Ceiling Speed Monitor (CSM)} is used. Its functionality is 
specified in the ETCS system specification~\cite{ETCSSRS-Principles}.
\end{enumerate}


Question (Q1) stated
 above is addressed in this article by applying a complete strategy that
allows for an abstraction of the state space by means of state equivalence class 
partitions and input equivalence classes. The CSM has inputs like current speed and
applicable speed restrictions whose domains are too large (even if discretisation techniques are applied) to be enumerated explicitly in test suites.
We show that for control systems of similar
complexity like the CSM, test suites of manageable size are generated. 
Question (Q2) 
is addressed by showing that a randomisation of the complete strategy which
preserves the exhaustiveness property for SUT behaviours inside the fault domain
also results in superior  test strength for the cases where ${\cal S}' \not\in{\cal D}$.




%The test suites used and compared with respect to their effectiveness stem from
%three sources: (A) ETCS-SUBSET076 of the ETCS standard, where conformance test cases
%are compiled that have been elaborated manually by domain experts. (B) ``Standard'' model coverage test cases that are considered as well-understood in the MBT communities. 
%(C) Test cases generated according to the equivalence class partitioning strategy recently developed by the authors. 

% ===========================================================================
\subsection{Main Contributions}
The work presented in this article is based on the previous 
publication~\cite{peleska_csm_2014}.
There, an initial version of the CSM test model has been published, 
and the application of model-based testing has been exemplified  in a qualitative 
way, that is, for a small set of hand-crafted examples that were suitable to highlight
the effects of each individual test strategy on specific classes of errors. The
tests from the standard~\cite{ETCS-Subset076} had not yet been considered.
 The novel contributions of the present article are characterised  as follows.
\begin{itemize}
\item The initial CSM test model version has been slightly updated to improve
its readability. In contrast to the qualitative assessment
 in~\cite{peleska_csm_2014}, however, we perform a detailed 
{\it quantitative}  test strength evaluation for the test strategies involved. This is achieved 
by using automated mutant generators on a reference implementation of the CSM programmed in Java.

\item Moreover, we use an extended version of the equivalence testing strategy   applied
in~\cite{peleska_csm_2014}: as shown in~\cite{peleska_sttt_2014}, 
that strategy is complete with respect to a given fault model. 
It has been demonstrated in~\cite{huebner15} that this complete strategy can be
modified   to obtain improved test strength when applied against SUTs whose true behaviour is reflected by  models {\it outside} the fault domain. This is achieved by performing
random selections from each input equivalence class, instead of testing with one representative per class only. This modified strategy is applied for the results presented in
this article.

\item Finally, this article presents a detailed comparison of the model-based test strategies with the manually designed CSM test suite published in the ETCS standard. It turns out that the model-based test strategies  have considerably higher test strength than the tests 
from the standard. This understandable, because the test suites  from the standard 
do not claim completeness, but aim at checking the basic conformity of EVC implementations 
to the standard. We can, however, still suggest to extend the current baseline of the ETCS test suites
in a way that will improve the trustworthiness of conformity verdicts, without unduly increasing the
number of test cases. 
\end{itemize}

In 2011 the {\it model-based testing benchmarks website} www.mbt-benchmarks.org has been 
created. Its objective is to publish test models that may serve as challenges or benchmarks 
for validating testing theories   and for comparing the capabilities of MBT tools~\cite{pel2011a}. 
The results presented in this article are publicly available on this website. The material
published there comprises 
\begin{itemize}
\item a SysML model of the CSM, 
\item the model-based test suites generated
according to the various MBT strategies, 
\item a formalisation of the relevant ETCS test cases from the standard~\cite{ETCS-Subset076}
as formulas specified in the temporal logic LTL,
\item a CSM reference implementation in Java, and
\item configuration parameters that were used for the mutation generator in order to 
create faulty test candidates. 

\end{itemize}

All automated test generation strategies described here have been implemented 
in the model-based test automation tool RT-Tester~\cite{EPTCS111.1} which applies 
the SONOLAR SMT solver\footnote{http://smtcomp.sourceforge.net/2012/reports/sonolar.pdf} for constraint solving and test data calculation. RT-Tester licences are
 freely available for academic research.



% ===========================================================================
\subsection{Overview}
In Section~\ref{chap:model} the SysML model for the ceiling speed monitor
is described, and we briefly summarise 
its behavioural semantics. With a test model at hand, two complementary approaches to test 
case generation are available: (1) test cases can be derived from requirements which 
can be traced back to the model, and (2) test cases can be derived from model coverage
goals. The underlying methods for these alternative approaches are described in 
Section~\ref{sec:req}, and test case examples for the CSM are presented. 
In 
Section~\ref{sec:ecpt} we describe the complete equivalence class partition testing method
in more detail, together with its randomisation that aims at higher test strength for
SUT outside the fault domain. For formal proofs of the strategy's properties we refer 
to previous publications.
We present the experimental results in Section~\ref{chap:results}, where the 
  test suites created by the different approaches described before
    are evaluated and compared.
In Section~\ref{sec:related} related work is described, and Section~\ref{sec:conc}
presents the conclusions and ongoing work related to the results presented here.
 