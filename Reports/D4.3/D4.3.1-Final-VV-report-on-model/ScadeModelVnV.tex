\subsection{Classical verification processes applicable to a SCADE~model}

The verifier shall be independent and shall neither be Requirements Manager, Designer nor Implementer as defined in the safety standards EN 50128 v2011.

The input documents needed are all the necessary System and Software Documentation used for the SCADE design activity and all the documentation produced during this phase, such as the SCADE Design Description, the SCADE Design Test Specification and the SCADE Design Test Report.

\subsubsection{Respect of modelling rules}

Syntactic rules of SCADE language are verified with the Quick Check tool available in the publisher. If an error is detected it must be corrected or justified in the SCADE Design Description document by the designer. The verifier shall ensure that no error remains or the justification associated is correct.

For specific modelling rules the verification has to be made manually by the verifier and described in the Verification Report. A grid of verification may be created in order to prove the compliance of the model with the rules. On some cases, dedicated tools can be developed.

Some modelling rules and constraints on Scade language can be defined and justified according CENELEC standard. Then these rules can be verified.

\subsubsection{Specification traceability check}

The verification of the compliance of the SCADE model with each requirement has to be made manually, by the verifier. 

The Scade model shall be correct according to the informal requirements and the informal specification shall be completely covered : each specification requirement must be traced in the SCADE model. The specification requirements which are not covered by the SCADE model must be listed and justified in the SCADE Design Description document by the designer.

\subsubsection{Testing and Validation of the model}

The verifier shall control the activity of software testing performed by the tester.

The software testing uses the Model Test Coverage (MTC) and the Generic Qualified Testing Environment (QTE) tools from SCADE. Five steps are performed.
\begin{itemize}
\item Establish the Test Specification document.
\item Writing scenarios in order to test the different functions independently.
\item Running scenarios on the SCADE model.
\item Extraction and analysis of results and the associated coverage (nodes, branches, branch conditions,...).
\item Establish the Test Report.
\end{itemize}

\subsubsection{Results}

All these different verifications activities shall be described in the Verification and Validation Plan, and their results shall be record in a Verification Report. Each disparity must be corrected or justified.

\paragraph{Verification report content}

The verifier shall produce a Verification Report containing the proof of the compliance of the SCADE model. It shall include the following points:
\begin{itemize}
\item the identity, version and configuration of SCADE model;
\item the verifier name;
\item the goal of the Verification Report;
\item the result of each verification process with:
\subitem - items which do not conform to the specifications;
\subitem - components, data, structures and algorithms poorly adapted to the problem;
\subitem - detected errors or deficiencies.
\item the fulfilment of, or deviation from, the Software Verification Plan;
\item assumptions if any;
\item a summary of the verification results.
\end{itemize}

\subsubsection{Conclusion}

The use of SCADE with its verification processes is compliant with the CENELEC norm but as it is not developed as open-source it is not compliant with the goal of openETCS project. 