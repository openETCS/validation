\newcommand{\q}{\textbf{q}}

\section{Improving Tests through Equivalence Class Testing Strategy}\label{sec:ecpt}
\label{sec:ecpt}

 In~\cite{peleska_sttt_2014}, two of the authors have presented a novel complete 
input equivalence class partition (IECP) testing strategy. In \cite{peleska_csm_2014}
this approach has first been applied to the Ceiling Speed Monitor and the analysis of
manually created mutants indicated that this strategy is superior to model derived test cases relying on
structural criteria of the state machine. In \cite{huebner15} the approach was evaluated 
with automatically generated mutants from a Java implementation. Here different refinements of
the strategy were compared and evaluated. For the Ceiling Speed Monitor a mutation score of 100 \%
could be achieved using a combination of equivalence class testing and boundary value tests.  



%@todo: copied from TAP publication, update the following sections for journal
%version

\subsection{Semantic Domain.} 

The novel equivalence class partition testing strategy presented 
in~\cite{peleska_sttt_2014} is applicable to deterministic, livelock-free systems 
with conceptually infinite input domains and finite internal state and output domains.
``Conceptually infinite'' means that the domains are too large to be explicitly 
enumerated for test purposes. This includes physical models with real-valued inputs,
but can also apply to finite but very large data types such as 64 bit integers or
doubles as used in typical programming languages or modelling formalisms. As pointed out 
in~\cite{peleska_sttt_2014,peleska_csm_2014}, this class of systems is quite
significant in the embedded systems domain: typical candidates are controllers 
processing analogue inputs and deriving discrete control decisions from these inputs, such
as thrust reversal controllers in aircrafts, or the speed monitors and airbag controllers described
in this paper.

The strategy has been proven to  be complete on the semantic domain of \emph{Reactive Input Output State Transition Systems (RIOSTS)} ${\cal S} = (S,s_0,R,V,D)$. These systems have state spaces $S$,
initial state $s_0\in S$, and transition relations $R\subseteq S\times S$. Their state spaces
consist of valuation functions $s: V\fun D$, where $V$ is a set of variable symbols and $D$ is 
the union of all variable domains. The variable symbols can be partitioned into $V = I\cup M \cup O$,
where $I$ comprises input variables, $M$ (internal) model variables, and $O$ output variables.
RIOSTS distinguish between \emph{quiescent} states $s\in S_Q$ and \emph{transient} states $s'\in S_T$,
such that $S_Q\cup S_T$ partitions the state space $S$. Transitions from quiescent states only change
input valuations, while internal model variables and output variables remain unchanged. The
resulting post-states may be quiescent or transient. Transitions from transient states always have
uniquely determined quiescent post-states (so we only allow deterministic RIOSTS here), and the associated transitions  leave the inputs
unchanged. This concept represents a natural abstraction of timed formalisms, where 
delay transitions allow for time to pass and inputs to be changed, while discrete
transitions produce output and change internal state, but are executed in zero 
time~\cite[p.~687]{DBLP:books/daglib/0020348}.

By associating atomic propositions $AP$ with free variables in $V$, any RIOSTS can be extended to a 
Kripke Structure~\cite{clarke_em-etal:1999a} 
$K({\cal S}) = (S,s_0,R,V,D,L,AP)$. The labelling function $L:S\fun 2^{AP}$
maps $s\in S$ to the set of all atomic propositions $p\in AP$ that evaluate to $\ist$,
when replacing every free variable $v$ of $p$ by its valuation $s(v)$ in state $s$.

\paragraph{Notation.}
In the exposition below, variable symbols are enumerated with the naming conventions
   $I = \{ x_1,\dots,x_k\}$, $M = \{m_1,\dots, m_p\}$, $O = \{y_1,\dots,y_q\}$. We use   notation
$\vec x = (x_1,\dots,x_k)$  for input variable vectors, and their   valuation in state $s$ is written as $s(\vec x) = (s(x_1),\dots,s(x_k))$. $D_I = D_{x_1}\times\dots\times D_{x_k}$ denotes the Cartesian product of the input variable domains. Tuples $\vec m, \vec y$ and $D_M$ and $D_O$ are defined over model variables and outputs in an analogous way. By $s\oplus\{\vec x \mapsto \vec c\}, \ \vec c \in D_I$ we denote the state $s'$ which coincides with $s$ on all variables from $M\cup O$, but maps the input vector to valuation $s'(\vec x) = \vec c$. 
For $(s_1,s_2)\in R$ we also use the shorter expression $R(s_1,s_2)$.  Restricting a state $s$ to variable symbols from a set $U \subseteq V$ is denoted by $s|_U$. This function has domain $U$ and coincides with $s$ on this domain.

% --------------------------------------------------------------------------
\subsection{Application to Concrete Modelling Formalisms.} 
The test strategy described below is elaborated on the semantic domain of RIOSTS. Every concrete
modelling formalism whose behavioural semantics can be represented by RIOSTS is automatically 
equipped with such a test strategy: the concrete model $M$ is translated into its corresponding RIOSTS ${\cal S}$. Then the test strategy is applied to ${\cal S}$, and this results in a set of test cases, each case represented by a finite sequence of inputs to the SUT. When executing the test cases, 
the transition relation of ${\cal S}$ is used to determine whether the SUT's reactions to these
input sequences are adequate. In this article, concrete models are expressed by SysML state machines,
and these can be associated with RIOSTS semantics which is consistent with the semi-formal specification of state machine behaviour in the UML/SysML standards~\cite{uml_2_4,SysML12}.




% --------------------------------------------------------------------------
\subsection{Equivalence Classes.}
We use the term {\it trace} to denote finite sequences of states, input vectors, or output vectors.
Applying a trace $\iota = \vec c_1\dots\vec c_n$ 
of input vectors $\vec c_i\in D_I$ to an RIOSTS ${\cal S} = (S,s_0,R,V,D)$  residing in some quiescent state
$s\in S$ stimulates a sequence of state transitions, each pair of consecutive states
connected by the transition relation $R$, and with associated  output changes  
as triggered by these inputs. 
Restricting this   sequence to quiescent states, this results in
a trace of states
$\tau = s_1.s_2\dots s_n$ such that $s_i(\vec x) = \vec c_i, i = 1,\dots,n$, and
$s_i(\vec y)$ is the last STS output resulting from application of $\vec c_1\dots\vec c_i$ 
to state $s$.\footnote{Observe that the restriction to quiescent states does not result in a loss
of information. Every transient state has the internal and output variable valuations coinciding with its quiescent pre-state, and its input valuation is identical to that of its quiescent
post-state.} This trace $\tau$ is  denoted by $s/\iota$. The restriction of $s/\iota$ to output variables is denoted by the trace $(s/\iota)|_O$.
Since transient states have unique quiescent post-states, $(s/\iota)|_O$ is a uniquely determined
output trace.
Two quiescent states $s, s'$ are {\it I/O-equivalent}, written $s\sim s'$, 
if every non-empty input trace $\iota$, when applied to $s$ and $s'$, results in the same outputs, that is,  
$(s/\iota)|_O = (s'/\iota)|_O$. Two STS ${\cal S}, {\cal S}'$ with the same input domain are I/O-equivalent, if their initial states 
are I/O-equivalent.
Note that  $s\sim s'$ asserts equivalent I/O-behaviour {\it in the future}, while it still admits that states $s$ and $s'$ show different output valuations, i.e.~$s|_O \neq s'|_O$. 
 

Since I/O-equivalence  $\sim$ is an equivalence relation on quiescent states, 
we can factorise $S_Q$ with respect to  $\sim$.
The \emph{initial input equivalence class partitioning (IECP)} ${\cal I}\subseteq\mathbb{P}(D_I)$ 
associated with $S_Q/_\sim$ is 
the coarsest partitioning of $D_I$ such that for all $\q\in S_Q/_\sim$, $X\in {\cal I}$,
there exists a uniquely determined I/O-equivalence class $ \delta(\q,X)\in S_Q/_\sim$, 
such that
\begin{equation}\label{eq:iecpa}
 \forall s\in \q, \vec c\in X: s/\vec c \in \delta(\q,X)
\end{equation}
and there exists a well-defined output $\omega(\q,X)\in D_O$, such that
\begin{equation}\label{eq:iecpb}
\forall s\in \q, \vec c\in X: (s/\vec c)|_O =  \omega(\q,X)
\end{equation}
It is shown in~\cite{peleska_sttt_2014} that $S_Q/_\sim$ is finite
 if the RIOSTS ${\cal S}$
has finite internal state domains and finite output domains, while the input domains may be infinite.
Moreover,   the coarsest partitioning ${\cal I}$  exists, and it is finite and uniquely determined under these prerequisites.
For these RIOSTS, 
properties (\ref{eq:iecpa}) and (\ref{eq:iecpb}) induce an abstraction    
  to DFSMs   
with state space $S_Q/_\sim$, input alphabet ${\cal I}$, and output alphabet
$D_O$:  (\ref{eq:iecpa})
specifies a well-defined total transition function $\delta : S_Q/_\sim\times {\cal I} \fun S_Q/_\sim$, and (\ref{eq:iecpb}) a well-defined
output function $\omega : S_Q/_\sim\times {\cal I} \fun D_O$.
When partitioning ${\cal I}$ further to a refined IECP $\overline{\cal I}$,
the characteristic properties (\ref{eq:iecpa}),(\ref{eq:iecpb}) are preserved.  

A finite sequence $X_1\dots X_k, X_i\in{\cal I}$   is called an \emph{abstract test case}:
concrete test input vectors $\vec c_i$  can be selected from each $X_i$, and, when applied
to the initial state $s_0$, this selection induces a trace $s_1\dots s_k$ of 
quiescent states, such that
$$
\exists \q_1,\dots,\q_k\in  S_Q/_\sim:\forall i \in \{ 1,\dots,k \}: 
s_i\in \q_i\wedge \q_i = \delta(\q_{i-1},X_i)
$$
The IECP properties imply that the
\emph{expected results} associated with this test case are then specified by the
output trace $\omega(\q_{i-1},X_i), i=1,\dots,k$.


 

In~\cite{peleska_sttt_2014} an algorithm for calculating  $S_Q/_\sim$ and ${\cal I}$ is given.
This algorithm produces propositions over variables from $V$, 
specifying the members of $S_Q/_\sim$ and ${\cal I}$, respectively. Making use of an SMT solver, the algorithm allows for identifying the  reachable I/O-equivalence classes
 $\q \in S_Q/_\sim$. As a consequence, every proposition characterising an abstract test
 case $X_1\dots X_k$  is actually feasible: this means that we can find concrete
traces in ${\cal S}$ such that, after deleting the transient states, the resulting
quiescent state sequence $s_0.s_1\dots s_k$ fulfils $s_i\in\q_i$ for $i = 0,\dots,k$ 
and $s(\vec x)\in X_i$ for $i = 1,\dots,k$.

For the model described in section~\ref{chap:model}, input equivalence
classes are unions of convex subsets of $\mathbb{R}^n$. It should be noted, however, that the
notion of I/O-equivalence and IECPs introduced here is far more general, since arbitrary propositional specifications of I/O-equivalence classes can be handled by the 
underlying theory. The input equivalence classes identified in~\cite[Example~1]{peleska_sttt_2014}, for example, contain members 
$z$ specified by conditions $z\!\! \mod m = n$.




% --------------------------------------------------------------------------
\subsection{Fault Models.}
%Complete black box test strategies are typically defined with respect to a given fault model 
%${\cal F} = ({\cal S},\sqsubseteq,{\cal D})$: ${\cal S}$ is the reference model 
%representing the ideal SUT behaviour. Relation $\sqsubseteq$ is a conformance relation describing
%the acceptable SUT behaviours,  so that an SUT whose behaviour is represented by some  
%${\cal S}'$ shall be accepted if and only if ${\cal S}' \sqsubseteq {\cal S}$.
%The fault domain ${\cal D}$ specifies a set of models that may or may not conform to ${\cal S}$,
%and it is assumed that the true behaviour of the SUT is identical to that of an   ${\cal S}'$
%contained in ${\cal D}$.

For the semantic domain of RIOSTS,
the fault models ${\cal F} = ({\cal S},\sim,{\cal D}({\cal S},m,\overline{{\cal I}}))$ are specified as follows.
The reference models ${\cal S}$ are semantic RIOSTS representations  of models elaborated in concrete formalisms, such that the expected behaviour of the SUT is specified by ${\cal S}$ up to 
I/O-equivalence. We use  I/O-equivalence as conformance relation.


Positive integer $m$ fulfils   
$m\ge n$, where $n$ is the number of I/O-equivalence classes of ${\cal S}$.
IECP $\overline{{\cal I}}$ is a refinement of the initial coarsest IECP ${\cal I}$  associated with ${\cal S}$.
Then the
  members ${\cal S}'$ of the fault domain ${\cal D}({\cal S},m,\overline{{\cal I}})$ 
are RIOSTS specified as follows.
\begin{enumerate}
\item The states of ${\cal S}'$ are defined over the same I/O variable space $ I \cup O$ as defined for the model ${\cal S}$.

\item Initial state $s_0'$ of ${\cal S}'$ coincides with initial state 
$s_0$ of ${\cal S}$ on $I\cup O$. 

\item ${\cal S}'$ generates only finitely many different output values. 

\item ${\cal S}'$ has a well-defined reset operation allowing to re-start the system from its initial state.

\item The number of I/O-equivalence classes of ${\cal S}'$ is less or equal $m$.
 


\item
\label{item:adequacy}  If ${\cal I}$, ${\cal I}'$ are the initial coarsest IECP of  ${\cal S}$, ${\cal S}'$, respectively, 
fulfilling the characteristic properties~(\ref{eq:iecpa}), (\ref{eq:iecpb}), then 
$\overline{{\cal I}}$ fulfils
  the following \emph{adequacy condition}: 
\begin{equation}\label{eq:x2}
\begin{array}{l}
\forall X\in {\cal I}, X'\in {\cal I}':  
\big(X\cap X'\neq\varnothing \Rightarrow
\exists \overline{X}\in \overline{{\cal I}}: \overline{X} \subseteq X\cap X'\big)
\end{array}
\end{equation}


\end{enumerate}



The intuition behind the adequacy condition~\ref{item:adequacy} is as follows. Every possible behaviour of a fault domain member ${\cal S}'$ can be exercised by visiting a state in 
some I/O-equivalence class $\q'$ and applying an input of some IECP member $X'\in {\cal I}'$
to this state. Using the refined IECP $\overline{{\cal I}}$ in the test suite as described below,
ensures that an input from $\overline{X}\subseteq X'\in {\cal I}'$ will be selected when 
${\cal S}'$ resides in $\q'$, so the behaviour associated with $(\q',X')$ will be stimulated in
at least one of the test cases. If,  when in a 
state of $\q'$,  ${\cal S}'$
conforms to the behaviour of ${\cal S}$ for all inputs from $X \setminus X'$, but fails
for inputs from $X\cap X'$, inputs selected from $\overline{X}\subseteq X\cap X'$ will uncover this error.

Conversely, suppose now that
 the reference model  ${\cal S}$ behaves differently, when IECs $X_1, X_2\in {\cal I}$
are applied in some state $\q$. Suppose further that ${\cal S}'$ fails to make this case 
distinction in a corresponding state $\q'$. Then there exists $X'\in {\cal I}'$ such that
 ${\cal S}'$ shows the same behaviour for all $\vec c\in X'$, but $X_1\cap X'\neq \varnothing$
 and $X_2\cap X'\neq \varnothing$, so two different behaviours should be visible according
 to the reference model. Now the adequacy condition guarantees that there exist two IEC
$\overline{X}_1, \overline{X}_2\in \overline{{\cal I}}$, such that 
$\overline{X}_1\subseteq X_1\cap X'$ and 
$\overline{X}_2\subseteq X_2\cap X'$.
 As a consequence, if inputs from every input class 
of $\overline{{\cal I}}$ are exercised, the behavioural differences for inputs from
$X_1\cap X'$ and $X_2\cap X'$ will be revealed. Summarising, the adequacy condition ensures that 
the IECP $\overline{{\cal I}}$ from where input data to the SUT is selected is fine-grained 
enough to stimulate every possibly deviating behaviour of ${\cal S}$ and ${\cal S}'$. These facts
are exploited in the complete test strategy described next.
 
 
 

% ==========================================================================
\subsection{Complete Finite Test Suite.}

The complete DFSM abstraction ${\cal M}$
 of ${\cal S}$ with states $S_Q/_\sim$, input alphabet
 $\overline{{\cal I}}$, transition function and output function
  as characterised in (\ref{eq:iecpa}), (\ref{eq:iecpb}), allows for application of 
finite complete DFSM testing strategies, such as the \emph{W-Method} introduced 
in~\cite{chow:wmethod,vasilevskii1973}. 
The general form of a W-Method test suite is 
\begin{equation}\label{eq:wtestsuite}
{\cal W} = P.\big(\bigcup_{i=0}^{m-n}\overline{\cal I}^i.W\big)
\end{equation}
where $P$ is the state transition cover, $\overline{\cal I}^i$ denotes the input trace segments of length $i$, and $W$ is the characterisation set.  Every test of  ${\cal W}$ consists of a (possibly empty) input trace from $P$, concatenated with an arbitrary input trace of length zero up to $m-n$, and  terminated by an input trace from the characterisation set. 
$P$  is the union of a {\it state cover} $C$ and  a
{\it transition cover} 
$C.\overline{\cal I}$: $C$ contains   the empty trace $\varepsilon$,
 and for any state $\q$ of ${\cal M}$, there exists an input trace in $C$ which, 
 when applied to the initial state, ends at $\q$. The transition cover  
 is defined by 
 $C.\overline{\cal I}=\{\iota.X~|~\iota \in C, X\in \overline{\cal I}\}.$ Summarising, the input sequences of a state transition cover ensure that (1) every state of the reference 
DFSM ${\cal M}$ associated with the reference model ${\cal S}$ is visited, and (2) every transition from every state is exercised.    A characterisation set is a set of input traces distinguishing each pair of states in a minimal DFSM. Using minimisation algorithms such as the one specified in~\cite{gill62}, characterisation sets can be constructed as a by-product of the minimisation process.

The test suite generated according to (\ref{eq:wtestsuite}) is called an 
\emph{abstract test suite}, because its elements are abstract test cases as defined above:
the inputs to be used in each test case are not yet
represented by concrete input vectors $\vec c$, but by input equivalence classes 
$\overline{X}\in \overline{{\cal I}}$. For creating an executable test suite,  inputs
$\vec c\in \overline{X}$ have to be selected for every   
$\overline{X}\in\overline{{\cal I}}$. 

The W-Method is complete for the fault model of all DFSM over the
same input and output alphabet and with at most $m$ states. It is shown in~\cite{peleska_sttt_2014}
that the associated   test suites with concrete inputs $\vec c\in \overline{X}$ are also complete for 
${\cal F} = ({\cal S},\sim,{\cal D}({\cal S},m,\overline{{\cal I}}))$. This completeness 
result is independent on the choice of concrete input data selected from each input 
equivalence class $X\in\overline{\cal I}$.

The fault domain ${\cal D}({\cal S},m,\overline{{\cal I}})$ introduced above
can be extended by increasing $m$ or by refining $\overline{{\cal I}}$. Increasing
$m$ increases the maximal length of input sequences in test cases in a linear way.
This affects the size of the test suite exponentially, 
but allows for fault domain members with 
higher \emph{recurrence diameters} $r$~\cite{biere_linear_2006}: 
this is the length of the longest loop-free path in
a Kripke structure. Erroneous SUT behaviour that only occurs at the end of such a longest loop 
free path may only be detected if the test cases use input sequences that are long enough
to traverse the SUT state up to the length of the recurrence diameter. 

Refining $\overline{{\cal I}}$ increases the size of the IECP, and this size 
increases the number of test cases in a polynomial way. It has to be noted, however,
that uniformly refining all members of $\overline{{\cal I}}$ -- for example, by using 
a sub-paving strategy as it is well known from interval analysis~\cite{jaulin2001} -- increases the size of the IECP exponentially with each new refinement step. 
The resulting fault domain contains members ${\cal S}'$ 
possessing narrower \emph{trapdoors}: these
are refined input guard conditions $g\wedge \delta$ applicable in certain ${\cal S}'$-states,
where  ${\cal S}'$ should behave uniformly for all inputs satisfying $g$. 
The true behaviour of ${\cal S}'$, however, 
conforms to the expected behaviour modelled by ${\cal S}$ only for inputs
fulfilling $g\wedge \neg\delta$, while   erroneous behaviour is revealed for inputs 
satisfying $g\wedge \delta$.

\subsection{Refinement and Randomisation of Equivalence Partition Tests}\label{sec:ecpt:refinement}
%\subsection{Randomisation of Equivalence Partition Tests}\label{sec:randomisation}

As we have seen above, the fault domain ${\cal D}({\cal S},m,\overline{{\cal I}})$ can be enlarged
via $m$ or $\overline{{\cal I}}$. As a drawback this seriously affects the size of the
resulting complete test suite ${\cal W}$. Therefore a refinement of the input equivalence 
class partitioning $\overline{{\cal I}}$ should be used which effectively increases the resulting 
test suites ability to uncover faults. In \cite{huebner15} two refinement heuristics were investigated and it could be
shown that these heuristics are able to improve the test strength of the resulting test suite.
Basically these two improvements work as follows:

\begin{enumerate}
  \item An input partitioning $\overline{\cal I}$ is used. $\overline{\cal I}$ is a refinement of the initial coarsest IECP and
reflects all case distinctions visible in guard conditions.
\item Additionally all boundary value conditions can be used in order to generate boundary
value tests of the original equivalence classes
\footnote{This can be done by adding extra constraints defining the
  boundaries of the original equivalence classes. As an example
  consider the guard condition $x\leq n$ restricting the input
  variable $x$ in input equivalence class $\overline{X}\in
  \overline{\cal I}$ to values less or equal $n$. In this case the
  additional constraint $\gamma$ defined as $x=n$ yields two new
  equivalence classes $\overline{X}_\gamma\subset\overline{X} $ and
  $\overline{X}_{\neg\gamma}\subset\overline{X}$ with
  $\overline{X}_{\gamma}\cap
  \overline{X}_{\neg\gamma}=\emptyset$. $\overline{X}_{\gamma}$ is the
  set of input vectors $\vec c_{\overline{X}_\gamma}$ that fulfill
  $x=n$ and thus are the boundaries of the original equivalence class
  $\overline{X}$ containing all input vectors, where $x<n$ or $x=n$.}.
\end{enumerate}


% As we have seen above, enlarging the fault domain 
% ${\cal D}({\cal S},m,\overline{{\cal I}})$ via $m$ or $\overline{{\cal I}}$ 
% seriously affects the size of the
% resulting complete test suite ${\cal W}$. 

In \cite{huebner15} another improvement of the equivalence class approach has been
presented. This strategy 
% We therefore investigate an alternative approach
% in this paper that 
aims at increasing the test strength of ${\cal W}$
   for SUTs ${\cal S}''$ whose
true behaviour is reflected by RIOSTS   {\it outside} 
${\cal D}({\cal S},m,\overline{{\cal I}})$. For obvious reasons it is assumed that these 
SUTs still fulfil  the RIOSTS 
compatibility requirements 1 -- 4 of the fault domain definition. This means that
${\cal S}''$ may have more than $m$ I/O-equivalence classes and may need an IECP
that is more fine-grained than $\overline{{\cal I}}$, but it is still assumed that 
${\cal S}''$ is an RIOSTS using the same I/O variables and possessing the same visible
initial state and fulfilling a reset condition.

To this end, we observe that the completeness property of the test suites introduced 
above does not depend on the concrete values selected from each input equivalence 
class $\overline{X}\in \overline{{\cal I}}$. For members ${\cal S}'\in {\cal D}({\cal S},m,\overline{{\cal I}})$ it would suffice to fix one input vector 
$\vec c_{\overline{X}}$ for every
$\overline{X}\in \overline{{\cal I}}$. Alternatively, we could also choose different members at random,
each time an input from some class $\overline{X}$ 
is required according to the abstract test suite
definition.
% and decide at random whether to choose internal or boundary 
% values from   $\overline{X}$.
While this alternative would not affect the suite's completeness property
when applied against members of ${\cal D}({\cal S},m,\overline{{\cal I}})$,
it favourably affects the test strength against RIOSTS outside 
${\cal D}({\cal S},m,\overline{{\cal I}})$: the chances for uncovering trapdoors
are obviously increased. This approach results in an 
\emph{adaptive random testing strategy}, where the selection of input data is no longer 
performed uniformly over the complete input domain, but selectively for each input equivalence
class $\overline{X}\in \overline{{\cal I}}$. Moreover, the random values from such 
an $\overline{X}$ are only applied when an $\overline{X}$-input is required according to the 
abstract test suite constructed from Equation~(\ref{eq:wtestsuite}). The experimental results in
\cite{huebner15} indicate, that the randomisation of the input selection is able to improve the 
mutation score for erroneous implementations that are {\it outside} the fault domain without increasing
the size of the resulting test suite and without nullifying the completeness result for 
implementations inside the fault domain.
  


% Technically the randomisation is implemented by running an SMT solver
% repeatedly to find concrete values of every input equivalence class
% $\overline{X}\in \overline{{\cal I}}$. The abstract test suite constructed from
% Equation~(\ref{eq:wtestsuite}) is a sequence on input equivalence classes. According to
% our equivalence class construction \cite{peleska_sttt_2014},
% an input equivalence class $\overline{X}\in \overline{{\cal I}}$ is defined by
% a proposition\footnote{The proposition is
% guaranteed to have a solution, since it describes an input equivalence class,
% which has at least one member and thus at least one assignment that fulfills the
% proposition.}  $g_{\overline{X}}$, containing solely variables in $I$.
% Using an SMT solver to solve  $g_{\overline{X}}$ results in a concrete input
% vector $\vec c \in{\overline{X}}$. Rerunning the solver for the same
% $\overline{X}$ and prohibiting existing solutions $c_1, \ldots, c_{n-1}$
% with a refined constraint
% $g_{\overline{X}}\land\overset{n-1}{\underset{i=1}{\bigwedge}}\neg c_i$ will
% result in a new solution $c_n$, i.e.
% a new concrete input $c_n\in \overline{X}$ of the input equivalence class. 
% The negation of existing solution yields an exponential growth of the runtime of
% the SMT solver in the worst case. Therefore two other heuristics were
% implemented:\\
% (a) the internal heuristics of the SAT solver have been randomized to get a
% ``random'' solution of $g_{\overline{X}}$. 
% (b)
% Interval analysis can be used to find a subpaving, that is an inner approximation of $g_{\overline{X}}$. From
% this subpaving random elements can be selected using a random number generator.
% As another runtime optimization the input selection can be parallelized. Once
% the input equivalence class partitioning $\overline{\cal I}$ is  available, candidates
% from every input equivalence class
% $\overline{X}$ can be calculated separately and in parallel, to find as many
% different concrete values as needed. 
% % The construction of the test oracles, i.e.
% % the expected outputs resulting from the execution of an input sequence from the
% % test suite ${\cal W}$ can be calculated from the abstract test suite and is
% % independent of the concrete inputs. 

% It has to be noted, however, that the complete test suite generated according to
% (\ref{eq:wtestsuite}) will not guarantee that every pair $(\q,\overline{X})$ with $\q\in S/_\sim$ and
% $\overline{X}\in\overline{\cal I}$ will be exercised the same number of times. Therefore we add
% test cases to ensure a minimal number $a$ each $(\q,\overline{X})$ is exercised, each time with 
% a new random selection $\vec c\in \overline{X}$. For these additional test cases we just repeat
% suitable cases from ${\cal W}$. If $p$ estimates the probability to detect a trapdoor when
% selecting a random value in $(\q,\overline{X})$, then the probability to uncover this during the
% randomised test suite is $1 - (1 - p)^a$.
