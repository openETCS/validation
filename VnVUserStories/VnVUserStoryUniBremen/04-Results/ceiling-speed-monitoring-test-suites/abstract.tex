\begin{abstract}
Apart from functional and non-functional 
requirements specifications, the published standard of the European Train Control System (ECTS) also contains a collection of system test suites. One of these suites is aimed at testing the Ceiling Speed Monitor (CSM) which is a function of the European Vital Computer (EVC), the main onboard controller of ETCS trains.
In this article we present a detailed comparison of   the CSM test suite specified in the ETCS standard with
tests generated from a CSM test model, using several automated generation strategies.  
The test strength of the suites is evaluated using mutations of a CSM software 
implementation.
It turn out that the test suite specified in the ETCS standard is significantly weaker than any of the suites generated with the model-based approach. The greatest test strength is provided by an equivalence partition testing strategy which has been previously elaborated by the authors. The CSM test model has been elaborated by the authors from the ETCS system requirements. Both the model and the model-based test suites have been made   publicly available.
\end{abstract}

\keywords
{Model-based testing, Equivalence class partition testing,
SysML, European Train Control System ETCS, Ceiling Speed Monitoring
}
