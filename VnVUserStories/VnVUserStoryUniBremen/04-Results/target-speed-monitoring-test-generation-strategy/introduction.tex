\section{Introduction}


\paragraph{Model-based testing}
Model-based testing (MBT) has gained much attention during the last 
decade~\cite{utting_taxonomy_2012-2,Petrenko:2012:MTS:2347096.2347101,anand_orchestrated_2013}. This is mainly due to the fact that 
MBT enables a high degree of automation, increasing the efficiency of test-related
verification and validation activities in a considerable way. 
The main automation benefits are mechanized test case creation from the model, 
test data calculation by means of mathematical constraint solvers, test procedure generation using model-based code generation techniques, and compilation of traceability data relating testing artifacts to requirements by exploiting traceability mechanisms available in the modeling languages~\cite{EPTCS111.1}.
At the same time, MBT
allows for the application of more complex test strategies. These provide higher test strength, but the test case generation algorithms involved can no longer be managed in a manual way; examples of these more complex strategies are given 
in~\cite{hierons_testing_2004,petrenko_testing_2014,huang_complete_2014}. 


For automated MBT, 
the modeling formalism applied needs to be associated with a formalized behavioral semantics describing how model states, inputs, and outputs evolve over time.
For test models described in the SysML formalism \cite{SysML15},  formalization options are described, for example, in \cite{EPTCS111.1,hilken_unified_2015}.
With these results at hand, model-based testing against concurrent real-time SysML models 
can be considered as a solved problem for continuous time/discrete control systems depending on notions of discrete or dense time, 
but producing discrete control outputs only. 
The system behavior over time is modeled, for example,
by means of concurrent SysML state machines, whose trigger conditions depend 
on variable values and timer conditions. This is then 
formally specified
by a transition relation describing how discrete control steps or time-delays are performed. Using, for example, an SMT solver that is also capable of floating point arithmetics, the possible transition steps 
can be calculated. Specific test objectives can be encoded as additional constraints
used in conjunction with the transition relation, so that the solutions provided by the
constraint solver describe at the same time valid state transitions of the model and suitable candidates for the test objective under consideration.

% ====================================================================
\paragraph{The challenge}
For real-time systems depending on mixed time-discrete and time-continuous
evolutions of observables and/or control variables,  
no comprehensive MBT methodology exists yet. While the formal semantics of 
these so-called \emph{hybrid systems}   has been thoroughly 
investigated~\cite{Hen96,alur01hierarchical},
the automated calculation of suitable test data for practical MBT
still remains a challenge. This is mainly due to the fact that best practices for
specifying time-continuous evolutions in test models and creating associated concrete
data by means of constraint solving are still subject to discussions. 


% ====================================================================
\paragraph{Objectives and main contributions}
In this paper, a novel approach to MBT in a hybrid systems context is presented, based
on the SysML modeling language. The utilization of blocks and associated diagrams for 
decomposing the functionality of the system under test (SUT) and the use of state machines
 is ``imported'' from proven MBT technology for time-discrete systems. 
These description means are, however, combined with an abstraction technique and 
extended by \emph{constraint blocks} and
\emph{parametric diagrams}~\cite[Section~10]{SysML15} for modeling time-continuous dependencies between inputs, outputs, and model variables.
From this descriptions means our proposed approach is able to generate concrete test cases in a fully automatic way.


%####Moved to 3
%Using a 2-step
%approach,  
%guard conditions of all state machines are  first 
%abstracted to Boolean   variables. The resulting abstract system is a so-called \emph{simulation} of the concrete system~\cite{clarke_em-etal:1999a}; the former represents an over-approximation of the latter. Abstract test cases are constructed from this simulation model using an
%input equivalence class partitioning 
%test strategy with proven error detection capabilities~\cite{huang_complete_2014}.  
%The  inputs associated with an abstract test case are sequences of input equivalence classes.
%In the second step, the abstract input sequences are resolved to sequences of concrete
%model variable valuations, using a mathematical constraint solver. For this step,   both the bindings of abstract Boolean
%condition variables to concrete model variables and the  additional physical constraints 
%encoded in constraint blocks and bound to concrete model variables by means of 
%parametric diagrams are taken into account. 


Our approach is illustrated and a proof of concept is given by application to a complex real-world system.
We create a test model of a  control problem from the
\emph{European Train Control System (ETCS)}, using the system requirement specification~\cite{ETCSSRS-Principles}. We describe how the expected EVC behavior can be modeled using the SysML subset indicated above, and the computational effort needed for automated test  generation by means of 
an SMT solver is evaluated. 





