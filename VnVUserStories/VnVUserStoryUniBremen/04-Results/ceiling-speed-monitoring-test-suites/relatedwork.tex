\section{Related Work}
\label{sec:related}
% ===========================================================================

% ===========================================================================
%@todo felix, jp: add references about automated MBT and safety-critical 
%systems testing
% ===========================================================================



 The European Train Control System (ETCS) official specification
includes a set of test cases to ensure the functional conformity,
compatibility and interoperability of train onboard units. Railway
domain experts designed these ETCS standard tests in the Subset 076
\cite{ETCS-Subset076}. They are created manually from the System Requirement
Specification (SRS), the requirement coverage could then be
evaluated. The chapter \cite{lars_ebrecht_verification_2012} shows how
the tests are actually used in their test environment execution to
prove the technical interoperabilty on the onboard units.

The authors \cite{Arriola_fault_2012} point out the deficiency of the
test specification for safety assessment. Furthermore, in
\cite{bonifacio_improvement_2011} a method  is proposed to generate new
tests respecting the existing tests of the standard, using a 
 formalism  to improve the latter
by automatically generating the tests from the specification.



In \cite{torens_inverse_2011,torens_starting_2011} the authors propose to start
from the ETCS standard tests  to generate a
functional behavior prototype of the onboard unit.  The prototype can be a
starting point for the architects or can be refined to a stronger test model, it
can also be used as a model-in-the-loop software within a simulation
environment.  Their idea is to take advantage of the huge amount of domain
specific knowledge that experts have put in designing the tests. Moreover, the
tests cases are already linked to the requirements, hence requirements can be
directly traced to the model.

