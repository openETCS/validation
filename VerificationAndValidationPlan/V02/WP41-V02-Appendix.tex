% \appendix

% \chapter{Requirements on \VV}
% \label{sec:appendix}


\section{Requirements on \VV from D2.9}
\label{sec:requirements-vv-D29}
%\todo{Adapt the intro text} 

The already provided requirements require a
safety plan compliant to the CENELEC EN~50126, 50128 and 50129.  This
pulls a number of requirements on V\&V, including Verification and
Validation plans. On the topic of compliance to EN~50128, one shall
also refer to the D2.2 document.


\reqfixed{02}{061}{A Verification plan shall be issued and complied
  with.}  
\subreqfixed{02}{061}{01}{The verification plan shall
  provide a method to demonstrate the requirements covering all the
  development artifacts.}  
\subreqfixed{02}{061}{02}{The verification
  plan shall state all verification activities required for each of
  these development artifacts.} 
 \reqfixed{02}{062}{A Validation Plan
  shall be issued and complied with.}  
\subreqfixed{02}{062}{01}{The
  validation plan shall provide a method to validate all functional
  and safety requirements over all development artifacts.}
\subreqfixed{02}{062}{02}{The validation plan shall state all
  validation activities required for each of these development
  artifacts.}

\reqfixed{01}{021}{The test plan shall comply the mandatory documents
  of the SUBSET-076, restricted to the scope of the OpenETCS project.}
\begin{justif}
  It will possibly be difficult to model all the tests in the course
  of the project, but the test plan should at least be complete.
\end{justif}


\reqfixed{02}{063}{Each design artifact needs a reference artifact
  which it implements (\emph{e.g.} code to detailed model, SFM to SSRS
  model\dots)} 

\subreqfixed{02}{063}{01}{The implementation between them relation
  shall be specified in detail.}  e.g.\ for state machine and a higher
level state machine mapping of interfaces, states and transition is
required.  This includes additional invariants, input assumptions and
further restrictions. This informaiton is the basis for verification
activities.

\subreqfixed{02}{063}{02}{The design of the artifacts shall be made
  such to allow verifiability as far as possible.}

\reqfixed{02}{064}{The findings from the verification shall be traced,
  and will be adequately addressed (taken into consideration, or
  postponed or discarded with a justification).}



\section{General Requirements on Verification}

\tbd{Reformulate text taken from the EN~50128 to avoid copyright infringements.}
{\footnotesize\sffamily\centering
  \begin{longtable}{||p{.15\textwidth}|p{.4\textwidth}|p{.4\textwidth}||}
    \hline\hline
    \bfseries Excerpt from EN~50128:2011 [N01] & \bfseries
    Requirement & \bfseries Project Relevance\\
    \hline\hline
    \endhead
    \hline\hline
    \endfoot
    5.3.2.7 & For each document, traceability shall be provided in
    terms of a unique reference number and a defined and documented
    relationship with other documents.  &
    fully applicable\\
    \hline 5.3.2.8 & Each term, acronym or abbreviation shall have the
    same meaning in every document.  If, for historical reasons, this
    is not possible, the different meanings shall be listed and the
    references given.  &
    \\
    \hline 5.3.2.9 & Except for documents relating to pre-existing
    software (see 7.3.4.7), each document shall be written according
    to the following rules:
    \begin{itemize}
    \item it shall contain or implement all applicable conditions and
      requirements of the preceding document with which it has a
      hierarchical relationship;
    \item it shall not contradict the preceding document.
    \end{itemize}
    &
    \\
    \hline 5.3.2.10 & Each item or concept shall be referred to by the
    same name or description in every document.  &
    \\
    \hline 6.5.4.14 & Traceability to requirements shall be an
    important consideration in the validation of a safety-related
    system and means shall be provided to allow this to be
    demonstrated throughout all phases of the lifecycle.  &
    \\
    \hline 6.5.4.15 & Within the context of this European Standard,
    and to a degree appropriate to the specified software safety
    integrity level, traceability shall particularly address
    \begin{enumerate}[a)]
    \item traceability of requirements to the design or other objects
      which fulfil them,
    \item traceability of design objects to the implementation objects
      which instantiate them.
    \item traceability of requirements and design objects to the tests
      (component, integration, overall test) and analyses that verify
      them.
    \end{enumerate}

    Traceability shall be the subject of configuration management.  &
    \\
    \hline 6.5.4.16 & In special cases, e.g. pre-existing software or
    prototyped software, traceability may be established after the
    implementation and/or documentation of the code, but prior to
    verification/validation.  In these cases, it shall be shown that
    verification/validation is as effective as it would have been with
    traceability over all phases.  & This requirement does not apply to
    the project.
    \\
    \hline 6.5.4.17 & Objects of requirements, design or
    implementation that cannot be adequately traced shall be
    demonstrated to have no bearing upon the safety or integrity of
    the system.  &
    \\
    \hline
\end{longtable}}


{\footnotesize\sffamily\centering
  \begin{longtable}{||p{.15\textwidth}|p{.8\textwidth}||}
    \hline\hline
    \textbf{Excerpt from EN~50128:2011 [N01]} & \textbf{Requirement} \\
    \hline\hline
    \endhead
    \hline\hline
    \endfoot
    6.1.4.1 & Tests performed by other parties such as the
    Requirements Manager, Designer or Implementer, if fully documented
    and complying with the following requirements, may be accepted by
    the Verifier.
    \\
    \hline 6.1.4.2 & Measurement equipment used for testing shall be
    calibrated appropriately.  Any tools, hardware or software, used
    for testing shall be shown to be suitable for the purpose.
    \\
    \hline 6.1.4.3 & Software testing shall be documented by a Test
    Specification and a Test Report, as defined in the following.
    \\
    \hline 6.2.4.2 & A Software Verification Plan shall be written,
    under the responsibility of the Verifier, on the basis of the
    necessary documentation.
    \\
    \hline 6.2.4.3 & The Software Verification Plan shall describe the
    activities to be performed to ensure proper verification and that
    particular design or other verification needs are suitably
    provided for
    \\
    \hline 6.2.4.4 & During development (and depending upon the size
    of the system) the plan may be sub-divided into a number of child
    documents and be added to, as the detailed needs of verification
    become clearer.
    \\
    \hline 6.2.4.5 & The Software Verification Plan shall document all
    the criteria, techniques and tools to be used in the verification
    process.  The Software Verification Plan shall include techniques
    and measures chosen from Table A.5, Table A.6, Table A.7 and Table
    A.8.  The selected combination shall be justified as a set
    satisfying 4.8, 4.9 and 4.10
    \\
    \hline 6.2.4.6 & The Software Verification Plan shall describe the
    activities to be performed to ensure correctness and consistency
    with respect to the input to that phase. These include reviewing,
    testing and integration.
    \\
    \hline 6.2.4.7 & In each development phase it shall be shown that
    the functional, performance and safety requirements are met.
    \\
    \hline 6.2.4.8 & The results of each verification shall be
    retained in a format defined or referenced in the Software
    Verification Plan.
    \\
    \hline 6.2.4.9 & The Software Verification Plan shall address the
    following:
    \begin{enumerate}[a)]
    \item the selection of verification strategies and techniques (to
      avoid undue complexity in the assessment of the verification and
      testing, preference shall be given to the selection of
      techniques which are in themselves readily analysable);
    \item selection of techniques from Table A.5, Table A.6, Table A.7
      and Table A.8;
    \item  the selection and documentation of verification activities;  
    \item  the evaluation of verification results gained;   
    \item  the evaluation of the safety and robustness requirements;  
    \item the roles and responsibilities of the personnel involved in
      the verification process;
    \item the degree of the functional based test coverage required to
      be achieved;
    \item the structure and content of each verification step,
      especially for the Software Requirement Verification (7.2.4.22),
      Software Architecture and Design Verification (7.3.4.41,
      7.3.4.42), Software Components Verification (7.4.4.13), Software
      Source Code Verification (7.5.4.10) and Integration Verification
      (7.6.4.13) in a way that facilitates review against the Software
      Verification Plan.
    \end{enumerate}
    \\
    \hline
\end{longtable}}

%\todo{Insert other tables.}

\section{Glossary}
\label{sec:glossary}

\begin{description}
\item[API:] Application Programming Interface. In the project, the API
  defines the interface of the EVC software to the operating system
  and hardware. \textit{The exact nature of the API still needs to be
    defined, whether it should be seen as a specification or as an
    implementation has yet to be resolved.}
\item[ATAM:] Architecture Tradeoff Analysis Method
\item [DAS2V:] Design Artifact Subject to Verification or Validation,
  e.g.\ some model or code fragment which has to be verified against
  its specification.
\item[EVC:] European Vital Computer
\item[FLOSS:] Free/Libre/Open Source Software
\item[FFM:] Fully Formal Model. Sometimes called ``Strictly Formal
  Model''.  A model of a part of the design which has a fully formal
  semantics and can thus be subjeced to rigorous analysis methods from
  the domain of mathematical or computational logic.
\item[HW:] Hardware
\item[SAAM:] Software Architecture Analysis Method 
\item[SFM:] Semi Formal Model. A model of some part of the design
  whose semantical interpretation is either not fully fixed or is
  similar to that of a program. I.e., the interpretation might depend
  on variations in the code generation or compilation, or it does not
  resolve ``semantic variation points'' (UML).
\item[SW:] Software
\end{description}

