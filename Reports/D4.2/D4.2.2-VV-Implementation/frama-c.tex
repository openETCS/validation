
\section{An Introduction to Formal Verification with Frama-C\slash WP}
\label{sec:frama-c}

Frama-C is platform dedicated to source-code analysis of C software.
It has a plug-in architecture and can thus be easily extended to 
different kinds of analyses.
The WP plugin of Frama-C allows to formally verify that a a piece of
C code satisfies its specification.
This implies, of course, that the user provides a \emph{formal specification}
of what the implementation is supposed to do.
Frama-C comes with its own specification language ACSL which stands for
\emph{ANSI\slash ISO C Specification Language}.
In order to help potential users to master ACSL we discuss in this section 
a very simple C function and explain various aspects of ACSL.

\subsection{First steps}

We will consider the function that computes the absolute value $|x|$
of an integer $x$.
In order to avoid name clashes with the function \inl{abs} in C standard library
we use the name \inl{abs_int}.

The mathematical definition of absolute value is very simple
\begin{align}
\label{eq:abs}
   |x| &= \left\{
            \begin{array}{rl}
               x  & \text{if $x \geq 0$} \\
               -x & \text{if $x < 0$}
            \end{array}
          \right.
\end{align}

A straightforward implementation of \inl{abs_int} is shown in Listing~\ref{fig:abs}.

\begin{listing}[hbt]
\begin{minipage}{\textwidth}
\lstinputlisting[style=acsl-block ]{./Abs/abs.c}
\end{minipage}
\caption{\label{fig:abs} An implementation of the absolute value function}
\end{listing}

In order to demonstrate that this implementation is correct we have to provide
a formal specification.
Listing~\ref{fig:abs1} shows our first attempt for an ACSL specification of \inl{abs_int} that
is based on the mathematical definition of $|\cdot|$ in Equation~\ref{eq:abs}.

\begin{listing}[hbt]
\begin{minipage}{\textwidth}
\lstinputlisting[style=acsl-block ]{./Abs/abs1.c}
\end{minipage}
\caption{\label{fig:abs1} A first attempt to formally specify \inl{abs_int}}
\end{listing}

The first thing to note is that ACSL specifications are placed in special C comments
(they start with \inl{/*@}).
Thus, they do not interfere with the execution of the code.
The \inl{ensures} clause in the specification 
expresses a \emph{postcondition}.
The ACSL reserved word \inl{\\result} is used to refer to the return value of a C function.
Note that we use the usual C operators \inl{==} and {<=} to express equalities and inequalities
in the specification.
There is, however, also an additional operator \inl{==>} which expresses logical implication.

\subsection{Why can Frama-C\slash WP not verify such a simple function?}

Although the specification and implementation in Listing~\ref{fig:abs1} look perfectly right, 
Frama-C\slash WP cannot verify that the implementation actually satisfies its specification.


\begin{listing}[hbt]
\begin{minipage}{\textwidth}
\lstinputlisting[style=acsl-block ]{./Abs/test_abs.c}
\end{minipage}
\caption{\label{fig:test_abs} Some simple test cases for \inl{abs_int}}
\end{listing}

The reason becomes clear if we look at some actual return values of \inl{abs_int}.
Listing~\ref{fig:test_abs} shows our test code whose output is shown
in Table~\ref{tbl:test_abs_output}

\begin{table}[hbt]
\begin{center}
\begin{tabular}{|r|r|}
\hline
\inl{x} &  \inl{abs_int(x)} \\ \hline\hline
0	&	0 \\ \hline
1	&	1 \\ \hline
10	&	10 \\ \hline
2147483647	&	2147483647 \\ \hline
-1	&	1 \\ \hline
-10	&	10 \\ \hline
-2147483648	&	-2147483648 \\ \hline
\end{tabular}
\end{center}
\caption{\label{tbl:test_abs_output} Test results for \inl{abs_int}}
\end{table}

The offending value is in the last line of Table~\ref{tbl:test_abs_output}
which basically states that \inl{abs_int(INT_MIN)} equals \inl{INT_MIN}
whereas it should equal \inl{-INT_MIN}.
The problem is that the type \inl{int} only present a 
finite subset of the (mathematical) integers.
Many computers use a two's-complement representation of integers
which cover the range $[-2^{31}\ldots 2^{31}-1]$ on a 32-Bit machine.
On such a machine \inl{-INT_MIN} cannot be  represented by a value
of the type~\inl{int}.

In a specification, Frama-C\slash WP interprets integers as mathematical entities
for which there is no such thing as an \emph{overflow}.
Thus, Frama-C\slash WP is perfectly right not being able to verify that \inl{abs_int}
satisfies the contract in Listing~\ref{fig:abs1}.

\subsection{Sharpening the contract of \inl{abs_int}}

It is of course well known that the operation \inl{-x} can overflow
and it is the fact that Frama-C can detect such overflows that 
prevents incorrect verification results.

The GNU Standard C Library clearly states that the absolute value of
\inl{INT_MIN} is undefined.\footnote{%
  See \url{http://www.gnu.org/software/libc/manual/html_node/Absolute-Value.html}
}
Under \textsf{OSX}, the manual page of \inl{abs} mentions under the field of ``Bugs'':
%
\begin{small}
\begin{verbatim}
    The absolute value of the most negative integer remains negative.
\end{verbatim}
\end{small}

Thus, our formal specification should exclude the value \inl{INT_MIN}
from the set of admissible value on which \inl{abs_int} can be applied.
In ACSL, we can use the \inl{requires} clause to express \emph{preconditions}
of a specification.
Listing~\ref{fig:abs1a} shows an extended specification of \inl{abs_int}
that takes the limitations of the type \inl{int} into account.

\begin{listing}[hbt]
\begin{minipage}{\textwidth}
\lstinputlisting[style=acsl-block ]{./Abs/abs1a.c}
\end{minipage}
\caption{\label{fig:abs1a} Taking integer overflows into account}
\end{listing}

Frama-C\slash WP is now capable to verify that the implementation of
\inl{abs_int} satisfies the specification of Listing~\ref{fig:abs1a}.

There is an important lesson that can be learned here:

\begin{framed}
Sometimes developers provide source code and imagine that a tool
like Frama-C\slash WP can verify the correctness of their implementation.
In order to fulfill its task, however, Frama-C\slash WP needs an ACSL specification. 
Such a specification, with a reasonably precise description of the admissible inputs
and expected behavior, must be derived from the \emph{requirements} of the software
and are not magically discovered from the source code by Frama-C\slash WP.
The code does what it does. 
In order to verify that the code does what someone expects, these expectations
must be clearly expressed, that is, they must be specified.
\end{framed}


\clearpage

\subsection{Separating specification and implementation}

Before we continue exploring more advanced specification and verification
capabilities of Frama-C\slash WP we turn to a simple software engineering question.

It is common practice to put function prototype into ``\inl{.h}'' files and
keep the implementation in files ending in~``\inl{.c}''.
Frama-C\slash WP support this separatation of specification of and implementation.

Listing~\ref{fig:abs2-h} shows the file \inl{abs2.h} which contains
a declaration of \inl{abs_int} together with an attached ACSL specification.

\begin{listing}[hbt]
\begin{minipage}{\textwidth}
\lstinputlisting[style=acsl-block ]{./Abs/abs2.h}
\end{minipage}
\caption{\label{fig:abs2-h} Specifying a function prototype in a header file}
\end{listing}

Listing~\ref{fig:abs2-c} shows the specification of \inl{abs_int} in a~\inl{.c} file.
Note that the file \inl{abs2.h} with the specification is included by this file.
Frama-C\slash WP can verify that this implementation satisfies the contract in
Listing~\ref{fig:abs2-h}.



\begin{listing}[hbt]
\begin{minipage}{\textwidth}
\lstinputlisting[style=acsl-block ]{./Abs/abs2.c}
\end{minipage}
\caption{\label{fig:abs2-c} Implementation at a different location than the specification}
\end{listing}

\FloatBarrier

\subsection{Modular verification}



\begin{listing}[hbt]
\begin{minipage}{\textwidth}
\lstinputlisting[style=acsl-block ]{./Abs/use_abs2_1.c}
\end{minipage}
\caption{\label{fig:use_abs2-1} A simple example of modular verification}
\end{listing}

\begin{listing}[hbt]
\begin{minipage}{\textwidth}
\lstinputlisting[style=acsl-block ]{./Abs/use_abs2_2.c}
\end{minipage}
\caption{\label{fig:use_abs2-2} Another example of modular verification}
\end{listing}

\begin{listing}[hbt]
\begin{minipage}{\textwidth}
\lstinputlisting[style=acsl-block ]{./Abs/use_abs2_3.c}
\end{minipage}
\caption{\label{fig:use_abs2-3} A more complex example of modular verification}
\end{listing}

\FloatBarrier

\subsection{Dealing with side effects}

\begin{listing}[hbt]
\begin{minipage}{\textwidth}
\lstinputlisting[style=acsl-block ]{./Abs/abs3a.c}
\end{minipage}
\caption{\label{fig:abs3a} An implementation with side effects}
\end{listing}

\FloatBarrier

\begin{listing}[hbt]
\begin{minipage}{\textwidth}
\lstinputlisting[style=acsl-block ]{./Abs/abs3b.c}
\end{minipage}
\caption{\label{fig:abs3b} Specifying the absence of side effects}
\end{listing}

\begin{listing}[hbt]
\begin{minipage}{\textwidth}
\lstinputlisting[style=acsl-block ]{./Abs/abs3c.c}
\end{minipage}
\caption{\label{fig:abs3c} Finer control of side effects}
\end{listing}


\FloatBarrier
