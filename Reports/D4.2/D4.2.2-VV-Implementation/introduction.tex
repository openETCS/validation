
\section{Introduction}

While major parts of the functionality of {Subset 026} are developed in 
higher-level languages, there is also a substantial part of \emph{supporting} software
that is developed in the programming language~C.

In this document we report about \emph{preliminary} results on the verification
of C-code developed in the OpenETCS project.
In particular, we report on the use of static analysis methods (including formal methods)
on C code that has been developed by the project partner Siemens (Germany).
Figure~\ref{fig:Bitwalker-Overview} gives an overview on the software that
is in the focus of this report.

\begin{figure}[hbt]
\begin{center}
\includegraphics[width=0.8\textwidth]{figures/Bitwalker-Overview.pdf}
\caption{\label{fig:Bitwalker-Overview} The place of \texttt{Bitwalker} with the OpenETCS software}
\end{center}
\end{figure}

The OpenETCS decoder is a large collection of functions dedicated to
the reading of of ETCS messages.
In order to fulfill their task these function rely on the relatively
small software package \inl{Bitwalker}.
The \inl{Bitwalker} software, as seen by the OpenETCS decoder,
is best understood as a ``class'' with a handful of methods.
Note that this class is implemented in~C as a \inl{struct} where
the methods are implemented as functions.
The core functionality  of this class, which consists in converting bit sequences to integers
and the other way round, depends on two more basic function, namely~\peek and~\poke.

This software has been analyzed by the OpenETCS project partners SQS (Spain)
and Fraunhofer FOKUS (Germany).
The Frama-C tool, which is developed by the French project partner {CEA LIST},
has been used for some of the analyses.

Section~\ref{sec:frama-c} gives a short overview on the \framacwp tool
that plays a central role in the verification of the Bitwalker functions.

In Section~\ref{sec:fokus} we take a more formal approach by 
\begin{enumerate}
\item formally specifying the expected functional behavior in the ACSL specification language of {Frama-C}
      and
\item using the {Frama-C} verification platform to establish a formal proof that the C code
      satisfies the formal specification.
\end{enumerate}

In Section~\ref{sec:sqs} we report about the results of a broad range
of static analyses. 
These methods are aimed at finding well-known deficiencies that might occur in C or \CC\ software.

Regarding the more formal approach it must be pointed out that so far only a 
part of Siemens' \emph{BitWalker} has been formalized and verified.
In the process of this work several enhancements for the Frama-C verification platform have 
been identified and reported to the developers at {CEA LIST}.




