

\subsubsection{Test Generation Applied to the Movement Authority Mechanism (Institut T\'el\'ecom)  }
\label{sec:TestGen-IT}

\paragraph{Contributing project partners}
% Usually, one main partner, perhaps with contributions from others

This work has been performed by (Institut T\'el\'ecom)

\paragraph{Process step}
% Classification of the activity according to D2.3a
% Name what is verified and to which report this would contribute.
% Use the numbers, e.g. System Design Verification Report (1-12).

The techniques used in this activity would fit best the phase SW Specification, contributing
to the Overall SW Test Specification (3-17), or the phase SW Design
with results used in the SW Component Test Specification (4-22). 

\paragraph{Object of verification}
% Which openETCS artifact (github link) or other documents/programs
% etc. (provide references)

The object of verification is an executable which animates the
handling of the movement authority (MA) by the train.

\paragraph{Available specification}
% The specification against which the object is to be checked. Usually
% coming form some openETCS artifacts (GitHub reference, process
% artifact number) or background material (reference, artifact number).
The specification consists of a formal model of the handling of the
movement authority in the ETCS system (from Sec.~\ref{sec:Subset-026-IT}).

\paragraph{Objective}
%In an ordinary development, the main objective would be to verify or validate
%something. Here, in openETCS, it would be quite common to demonstrate
%the applicability of some method/tool, or evaluate its capabilities.

This activity shall demonstrate the applicability and usefulness of
test generation from a formal model to check an implementation.   


\paragraph{Method/Approach}
% Short description of how the verification/validation is performed

Tests are generated which would constitute a part of a test suite
covering the behavior the formal model. The tests are applied to a
simulation of the system. Thereby, is  conformity of
the implementation behavior with that of the model is checked. 

\paragraph{Means/Tools}
% Could be integrated with the previous paragraph. Assign an
% approrpiate tool class (T1, T2, T3) according to EN 50128. Try to assign a
% maturity level 
%https://github.com/openETCS/validation/blob/master/VerificationAndValidationPlan/V02/VnVUsrStrTmplt-140709-02.pdf

The approach has been realized with TestGen-IF, the test generation facilities of
the IF-Tools \cite{IFTools14}. For demonstrating the application to an
executable system simulation, such is generated by deriving JAVA
simulators of the components of the model.  

\paragraph{Results}
% Results related to the objective.
% Refer to appropriate document (preferably GitHub) for more complete
% description.

The tests have been generated and applied to the JAVA simulation of
the system. By that, the applicability of the model-based test
generation approach has been demonstrated.


\paragraph{Conclusion}
%What has been achieved, what is missing, what has been learned

it has been demonstrated that test generation from a formal model can
be applied successfully on parts of the ETCS system. This could be
applied for the Overall SW Test Specification (3-17) or the SW
Component Test Specification (4-22). 