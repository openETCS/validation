

\subsubsection{Verification of the Chapter~3.8 of Subset~026 (Institut T\'el\'ecom)  }
\label{sec:Subset-026-IT}

\paragraph{Contributing project partners}
% Usually, one main partner, perhaps with contributions from others

This work has been performed by (Institut T\'el\'ecom)

\paragraph{Process step}
% Classification of the activity according to D2.3a
% Name what is verified and to which report this would contribute.
% Use the numbers, e.g. System Design Verification Report (1-12).

This activity is part of the verification of the Elaborated System
Requirements which are based on Subset~026 \cite{subset-026:3.3.0}. It
contributes to the System Design Verification Report (1-12). In
formalizing and analyzing the procedures its findings contribute also
to the definition of the Elaborated System Requirements themselves
(1-07).

\paragraph{Object of verification}
% Which openETCS artifact (github link) or other documents/programs
% etc. (provide references)

The object of verification is the concept of the movement authority
(MA) as defined in Chapter~3.8 of Subset~026
\cite[Sec.~3.8]{subset-026:3.3.0} and a model derived from that
description.

\paragraph{Available specification}
% The specification against which the object is to be checked. Usually
% coming form some openETCS artifacts (GitHub reference, process
% artifact number) or background material (reference, artifact number).
The specification consists of safety properties relating to the
movement authority.

\paragraph{Objective}
%In an ordinary development, the main objective would be to verify or validate
%something. Here, in openETCS, it would be quite common to demonstrate
%the applicability of some method/tool, or evaluate its capabilities.

This activity shall demonstrate the applicability and usefulness of
the IF model checker for analyzing abstract models. As a by-product,
parts of the specification in Subset~026 are checked.   


\paragraph{Method/Approach}
% Short description of how the verification/validation is performed

The  control mechanism of movement authorities
and the movement of the train  in a simplified environment is modeled
by an extended timed state machine. The safety properties express that
the train stays within the bounds set by the movement authorities.

\paragraph{Means/Tools}
% Could be integrated with the previous paragraph. Assign an
% approrpiate tool class (T1, T2, T3) according to EN 50128. Try to assign a
% maturity level 
%https://github.com/openETCS/validation/blob/master/VerificationAndValidationPlan/V02/VnVUsrStrTmplt-140709-02.pdf

The approach has been realized with the IF-Tools \cite{IFTools14}.

\paragraph{Results}
% Results related to the objective.
% Refer to appropriate document (preferably GitHub) for more complete
% description.

The safety properties have been verified on the model. During
modeling, three inconsistencies, ambiguities and gaps in the
specification have been identified. They have been reported
in~\cite{specfindingsTSP}.

\paragraph{Observations/Comments}
% An optional section where anything can be included which has been
% observed without direct connection

The activity has been published in \cite{NgCa14}.

\paragraph{Conclusion}
%What has been achieved, what is missing, what has been learned

Formal modeling and formal verification can be applied to improve the
informal ETCS specification. 