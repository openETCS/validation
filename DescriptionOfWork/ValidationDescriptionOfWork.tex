\documentclass{template/openetcs_article}
% Use the option "nocc" if the document is not licensed under Creative Commons
%\documentclass[nocc]{template/openetcs_article} 
\usepackage{lipsum,url}
\usepackage{xspace}
\usepackage{graphicx}
\usepackage{fixme}
\usepackage{lscape} 
\usepackage{pgfgantt}
\usepackage{adjustbox}
\usepackage{datetime}



%user specified macros


\newcommand{\VV}{Verification \& Validation\xspace}
\newcommand{\vv}{verification \& validation\xspace}

\def\CC{{C\nolinebreak[4]\hspace{-.05em}\raisebox{.4ex}{\tiny\bf ++}}}

\newcommand{\bitwalker}{\mbox{\texttt{Bitwalker}}\xspace}

\newcommand{\poke}{\mbox{\texttt{Bitwalker\_Poke}}\xspace}
\newcommand{\peek}{\mbox{\texttt{Bitwalker\_Peek}}\xspace}
\newcommand{\acsl}{\mbox{\textsf{ACSL}}\xspace}
\newcommand{\isoc}{\mbox{\textsf{C}}\xspace}
\newcommand{\framac}{\mbox{\textsf{Frama-C}}\xspace}
\newcommand{\framacwp}{\mbox{\textsf{Frama-C\slash WP}}\xspace}
\newcommand{\why}{\mbox{\textsf{Why}}\xspace}
\newcommand{\wpframac}{\mbox{\textsf{WP}}\xspace}
\newcommand{\altergo}{\mbox{\textsf{Alt-Ergo}}\xspace}
\newcommand{\cvc}{\mbox{\textsf{CVC4}}\xspace}
\newcommand{\z}{\mbox{\textsf{Z3}}\xspace}
\newcommand{\coq}{\mbox{\textsf{Coq}}\xspace}
\newcommand{\cealist}{\mbox{\textsf{CEA LIST}}\xspace}
\newcommand{\init}{\mbox{\texttt{Bitwalker\_IcrementalWalker\_Init}}\xspace}
\newcommand{\peeknext}{\mbox{\texttt{Bitwalker\_IcrementalWalker\_Peek\_Next}}\xspace}
\newcommand{\pokenext}{\mbox{\texttt{Bitwalker\_IcrementalWalker\_Poke\_Next}}\xspace}
\newcommand{\pokefinish}{\mbox{\texttt{Bitwalker\_IcrementalWalker\_Poke\_Finish}}\xspace}
\newcommand{\peekfinish}{\mbox{\texttt{Bitwalker\_IcrementalWalker\_Peek\_Finish}}\xspace}
\newcommand{\locals}{\mbox{\texttt{T\_Bitwalker\_Incremental\_Locals}}\xspace}

\newcommand{\inl}[1]{\lstinline[style=inline]{#1}}




\graphicspath{{./template/}{.}{./images/}}
\begin{document}
\frontmatter
\project{openETCS}

%Please do not change anything above this line
%============================
% The document metadata is defined below

%assign a report number here
\reportnum{OETCS/WP4/DescriptionOfWork}

%define your workpackage here
\wp{Work Package 4: ``Validation \& Verification Strategy''}

%set a title here
\title{openETCS Validation \& Verification Strategy Work Package}

%set a subtitle here
\subtitle{Description of Work}

%set the date of the report here
\date{March 2012}

%define a list of authors and their affiliation here

\author{Marc Behrens}

\affiliation{WP4 Leader}
 
\author{Hardi Hungar}

\affiliation{WP4.1 Task Leader (Idetntification of tools and profile usage)}

\author{\ }

\affiliation{WP4.2 Task Leader (\VV of the formal model )}
 
\author{Jens Gerlach}

\affiliation{WP4.3 Task Leader (\VV of the implementation \/ code)}

\author{Hansj\"{o}rg Manz, Jan Welte}

\affiliation{WP4.4 Task Leader (Verification of the tools and processes)}

\author{Cyril Cornu }

\affiliation{WP4.5 Task Leader (Internal Assessment)}
  
% define the coverart
\coverart[width=350pt]{chart}

%define the type of report
\reporttype{Description of work}



\begin{abstract}
%define an abstract here
This work package will focus on the validation and verification of the model. At the very first beginning the target and requirements of the \VV strategy have to be described, i.e., what should be checked? Depending on the Modelling framework, the modelling language and formalization of the System requirements a strategy in form of a concept has to be defined how the consistency, coherence of the model as well as the coverage of system requirements will be transparently verified. Additionally, it is important to validate the model, i.e., to evidence the equivalence of the model and the ETCS system requirement specification (Subset-026 et al.). In other words the reliability and acceptance of the model has to be generated, e.g. nothing is lost or added or mutated and so on. Additionally, it has to be checked that the code is consistent with the model. The WP is intended to be performed in parallel with the modelling in order to apply the strategy and to generate feedback to the modelling process as well as to measure the quality and maturity of the model.
Beside a subtask will manage the consideration of all relevant safety requirements (e.g. EN 50128/129) in the modelling process.
\end{abstract}

%=============================
%Do not change the next three lines
\maketitle
\tableofcontents
\listoffiguresandtables
\newpage
%=============================

% The actual document starts below this line
%=============================


%Start here


\chapter{Introduction}

In this intermediate report we describe the activities to formally verify
the correctness of parts of the software developed in the OpenETCS project.

While major parts of the functionality of {Subset 026} are modelled in 
higher-level languages, there is also a substantial part of \emph{supporting} software
that is developed in the~C programming language.

In this document we report about results on the verification of that C~code.
In particular, we report on the use of static analysis methods (including formal methods)
on C~code that has been developed by the project partner Siemens.

\begin{figure}[hbt]
\begin{center}
\includegraphics[width=0.95\textwidth]{figures/scope-of-formalization.pdf}
\caption{\label{fig:scope-of-formalization} Scope of formal methods with in OpenETCS}
\end{center}
\end{figure}

Figure~\ref{fig:scope-of-formalization} outlines the roles of formal methods
within the OpenETCS project.
What this figure shall convey is that even a subsystem such as described by
\emph{Subset 026} of the ETCS specification
is usually too complex to be completely formally specified.
Therefore, \emph{semi-formal modelling techniques} and \emph{tests} and 
\emph{simulations} play a crucial role to verify that the implementation
satisfies its specification.
However, for clearly defined modules and select system properties, formal methods
can well be applied to establish the correctness of an implementation.

Figure~\ref{fig:scope-of-code-verification} shows slightly more detailed
the OpenETCS software.
The report at hand is limited to the parts encapsulated by C software encapsulated 
in a \fbox{dashed box}.

\clearpage

\begin{figure}[hbt]
\begin{center}
\includegraphics[width=0.95\textheight,angle=90]{figures/OpenETCS-Stack.pdf}
\caption{\label{fig:scope-of-code-verification} Scope of code verification}
\end{center}
\end{figure}

\FloatBarrier

\section{Software layers}

Figure~\ref{fig:software-layers} shows the layer structure of the OpenETCS C~code.
The OpenETCS decoder\slash enocder is a collection of data structures and associated functions
for reading and writing ETCS packets from\slash to a bit stream.

\begin{figure}[hbt]
\begin{center}
\includegraphics[width=1.0\textwidth]{figures/software-layers.pdf}
\caption{\label{fig:software-layers} Software layers of the OpenETCS C~code}
\end{center}
\end{figure}

\FloatBarrier

In order to fulfill their task the decoder and encoder function rely on an
implementation of bit streams in C.
The \inl{Bitstream} package in turn is built on top of the so-called \emph{bitwalker} layer.
In order to accomplish the task of formal verification of these layers 
we also provided several functions that read and write individual bits for basic C~types.

This software has been analyzed by the OpenETCS project partners SQS (Spain)
and Fraunhofer FOKUS (Germany).
SQS used several static analysis tools to identify defects and to derive useful metrics.
Fraunhofer FOKUS, on the other hand, used the \framac tool set,
which is developed by the French project partner \cealist,
in order to \emph{formally verify} various properties of the \bitwalker.

These analyses contribute to the ultimate verification goals, which are the following:

\begin{enumerate}
\item provide evidence that the Bitwalker software satisfies 
      accepted quality standards
\item develop a formal specification for the Bitwalker software
\item verify that the Bitwalker software satisfies its formal specification
\item show that the Bitwalker software does not raise runtime errors
\item verify that OpenETCS decoder calls the Bitwalker software only
      according to its specification
\end{enumerate}

We are confident that all these verification goals can be reached.
For this preliminary verification report,
we provide partial answers to the first four topics.
In order to achieve the last goal, more development and verification
work is currently conducted by Fraunhofer ESK and Fraunhofer FOKUS. 

\section{Structure of this document}

Chapter~\ref{sec:frama-c} gives a short overview on the \framacwp tool
that plays a central role in the verification of the Bitwalker functions.
Here we also try to rectify some misunderstandings about formal verification
that we have encountered in our work.

In Chapter~\ref{sec:formal-verification} we analyze
the functions \peek and \poke from the Bitwalker core and
\begin{enumerate}
\item formally specify the
      expected functional behavior in the \acsl specification language of {Frama-C}
      and
\item report on the formal proof 
	that these
      C~functions do not raise runtime errors when called according to their
      formal specification, established using 
      the {Frama-C} verification platform.
\end{enumerate}

So far only a part of Siemens' \bitwalker has been formalized and verified.
In the process of this work several enhancements for the \framac verification platform
have been identified and reported to the developers at {CEA LIST}.

In Chapter~\ref{sec:static-analysis}, we report about the results of
SQS' application of a broad range of static analysis tools on the \bitwalker. 
In contrast to \framac, these tools cannot exhaustively
detect all potential defects of a given kind.
Nevertheless, these they are very useful at finding well-known quality deficiencies that
might occur in C or \CC\ software.

In Chapter~\ref{sec:conclusions}, we draw conclusions from this preliminary work
and outline the next steps in our verification efforts.



%-----------------------------------------------------------------------
\section{Identification of Tools and Profile Usage}
%-----------------------------------------------------------------------
The objective of this task is to prepare the activities of the tasks
4.2 and 4.3, which are concerned with \vv of the model and the code,
respectively. It defines an overall \vv strategy and plans for
verification and validation, detailing how and with what means the
strategy is going to be implemented. Formally, the requirements on \vv
which are to be covered in the plans are listed in D2.9 ``Requirements
for \VV''. An essential requirement is adhering to
the applicable standards, mainly the EN~50128. The plans shall define
activities adequate for a complete development, but also foresee a
tailoring to the partial development actually realised in the project.

The WP~2 deliverable D2.3 ``Process Definition'' defines the
development process and its steps, and thus also identifies the main
verification steps. These will be detailed in the verification plan,
defining what has to be achieved on a more technical level. Also, a
selection of potentially applicable tools and methods will be
given. The validation plan shall take the overall development approach
including verification activities into account and define what methods
are suitable for demonstrating that all requirements (to be defined by
WP~2) are met by the end result, and also address the question of
safety integrity.

Thus, the plans shall contain a selection of methods and a list tools
suitable for applying the chosen V\&V methods for\footnote{Terms
  according to the draft of D2.3, abbreviations according to the draft
  of D2.6.}
\begin{enumerate}
\item the sub-system requirements specification (SSRS) and models
  (SFM, FFM)
\item the software semi-formal model and software architecture
  description
\item code derived from the software semi-formal model
\item the software strictly formal model and the software design
  description
\end{enumerate}

D2.1 ``Report on Existing Methodologies'' shall already provide a list
of potentially relevant methods and tools. Each method and tool
applied in WP~4 shall be described in a format detailing its purpose,
role and characteristics in terms of requirements on \vv.  Not all
steps will be automatic or semi-automatic. Manual techniques
like review or walkthrough will play a role, too. In selecting tools,
besides the requirement of openness (FLOSS), the question of
qualification, depending on the role the tool will play, has to be
answered. The formats describing methods and tools and criteria for
their evaluation will be given in a document D4.1a ``Preliminary
Evaluation Criteria on Verification and Validation'' supplementing the
deliverable to be produced by this task, D4.1 ``Report on V\&V Plan \&
Methodology''. 

Important classes of objects subjected to \vv activities are the
different models, be they semi-formal or formal, the code derived from
them and the versions of the demonstrator. Similar to methods and
tools, these objects need to be defined w.r.t.\ their nature and role
in the development process, including the requirements for \vv. A
format for their description shall be provided with D4.1.

An analysis of these objects and the methods with which they are
developed shall lead to a refinement and concretisation of the
verification and validation plans in the course of the project after
termination of T4.1.


\begin{table}[h]
\caption{T4.1 Inputs, Outputs and Deliverables} %title of the table
\begin{adjustbox}{width=\textwidth}
\begin{tabular}{|l|l|r|r|r|}
\hline
\multicolumn{5}{|c|}{\textbf{T4.1 Identification of Tools and Profile Usage}} 
\\\hline
Type & Description & Due Date & Due Month & status 
%status output going to other tasks/wps    : not started, started, complete
%status input coming from other tasks/wps: no, yes
%\\\hline
%$\rightarrow$ & \todo{Ox.2.3: Sample Input Information}  & \shortmonthname[1]-2014 & T0+19 & no 
%\\\hline
%$\leftarrow$ & \todo{O4.1.1: Sample Output Information}   & \shortmonthname[10]-2013  & T0+16 & started  
\\\hline
$\rightarrow$ & \emph{D 2.1} Report on Existing Methodologies & \shortmonthname[3]-2013 & T0+9 &  no
\\\hline
$\rightarrow$ & \emph{D 2.3} Process Definition & \shortmonthname[5]-2013 & T0+11 &  no
\\\hline
$\rightarrow$ & \emph{D 2.4} Methods Definition & \shortmonthname[5]-2013 & T0+11 &  no
\\\hline
$\rightarrow$ & \emph{D 2.9} Set of Requirements for V\&V & \shortmonthname[5]-2013 & T0+11 &  no
\\\hline
 D &\emph{D 4.1a} Preliminary Evaluation Criteria on V\&V & \shortmonthname[3]-2013 & T0+12 & started
\\\hline
  D &\emph{D 4.1} Report on V\&V Plan \& Methodology  & \shortmonthname[7]-2013 & T0+13 & started
\\\hline
\end{tabular}
\end{adjustbox}
\end{table}

%-----------------------------------------------------------------------
\section{\VV of the Formal Model}
%-----------------------------------------------------------------------
\tbc

%old introduction
To ensure the correctness and consistency of the model and its implementation, the validation and verification has to be performed alongside with the modelling process. Thus these tasks will be performed repeatedly during WP3 and will provide feedback to it.

This task handles the verification and validation of the formal model. This will be accomplished by applying the methods chosen in WP4 Task 1 onto the formal model from WP3 using the tool chain developed in WP3. Depending on the chosen approach and applicable tools a variety of verification methods can be applied like:
\begin{enumerate}
\item proof technique
\item model checking technique
\item Simulation
\end{enumerate}
As the verification and validation is part of the development chain, this task is being applied iteratively in parallel to the development of the formal model in WP3. The feedback given should focus on the consistency and correctness of the model and development process in WP3.
The results of this task are the verification and validation specifications (how to perform the V\&V on the formal model), the basic materials (the actual tests cases, checklists, etc.) and the V\&V report on the formal model.

%new introduction
The main objective of this task is to guarantee the correctness and consistency of the semi-formal model (SysML) and the fully formal model (SCADE), by applying verification (concerned with building the model right) and validation (concerned with building the right model) techniques. To ensure this, the requirements described in D2.9 “Requirements for Verification \& Validation” for the Subset-026 ought to be met by the model. Also, following the standard Technical Specification for Interoperability is mandatory.

As detailed in D2.9, a list of safety requirements will be collected and refined to create requirements and properties. Once these requirements are ready, there are a variety of techniques (described below) that can be used to verify and validate the formal model. 


%-----------------------------------------------------------------------
%\subsection{\tbd}
%-----------------------------------------------------------------------
%\tbd
%Should Risk Assessment be done in parallel to the verification activities?

%-----------------------------------------------------------------------
\subsection{Proof Technique}
%-----------------------------------------------------------------------
A proof is a demonstration that if some fundamental statements (axioms) are assumed to be true, then some mathematical statement is necessarily true. As mentioned in the requirements document produced by WP2, as much as possible, formal proof would then be used to prove that the OpenETCS model never enter a Feared State, as long as the other subsystem (RBC, communication layer. . . ) fulfill their own safety properties (axiom describing the environment). Such theorem proving helps to increase our confidence on the specified model. The proof techniques should be integrated in the selected tool chain.

In order to use formal proof to verify if the SFM (Semi-formal model) and FFM (fully formal model) comply with the safety and function requirements (cf. R-WP2/D2.6-02-058), the properties to be proven have to be identified and described. There will be a set of axioms that will describe both functional and/or safety properties of the system. The choice of axioms describing functional and/or safety properties will be provided by safety analysis in an independent way from approaches used to specify, design, validate or verify. It must be noted that the model obtained from the Subsystem Requirements Specification should be verified in this manner at a first stage.


%-----------------------------------------------------------------------
\subsection{Model Checking}
%-----------------------------------------------------------------------
Model checking is an automatic technique for verifying finite-state reactive systems. As such, one could automatically check if the model specifies most of the requirements of the system, such as the important safety properties described in Task 4.4.

Similar to proof techniques, in order to use model checking to verify if the SFM (Semi-formal model) and FFM (fully formal model) comply with the safety and function requirements (cf. R-WP2/D2.6-02-058), the properties to be proven have to be identified and described. To implement the use model checking, it is mandatory to specify the model using finite-state reactive systems, and they should also provide an intuitive way to express the properties to be model checked. The set of critical requirements to be verified need to be clearly identified. The criteria for the model to be considered a representation of the standard is that all properties are checked. The proposed model checking techniques should be supported in the selected tool chain.

%-----------------------------------------------------------------------
\subsection{Simulation}
%-----------------------------------------------------------------------
As for simulation, the artifacts should provide means to execute the model. The simulator must be automatically generated, so that, when run against test scenarios (inputs/outputs for the model), we may conclude whether the model follows the specification or not. In particular, it is important to define test scenarios for the safety critical properties. Since, the development within openETCS has to the goal to reach the CENELEC EN 50128 SIL 4 standard, it is highly recommended (cf. SIL 4) that the simulation needs to cover all states, transitions, data-flow, and paths in the model. It would also be desirable to include graphical representation of the simulation/model and also provide a report of the visited components as specified by CENELEC EN 50128 SIL 4. 

CENELEC EN 50128 SIL4 also advocates to perform tracing. Being able to trace the requirements that are met during a simulation is also advisable to allow simple requirement coverage. 


%-----------------------------------------------------------------------
\subsection{Testing Methods}
%-----------------------------------------------------------------------
Testing methods will be applied in order to check the correctness of the model with respect to the informal specification and safety requirements. Testing techniques will also be used to check properties that cannot be checked using the proof, model checking or simulation. These techniques will be complementary of the above mentioned techniques. In particular, if a safety requirement/property cannot be proven, testing covering all reasonable possible events/transitions must be used (cf. safety requirements R-WP2/D2.6-02-058.04 and R-WP2/D2.6-02-058.02).

%-----------------------------------------------------------------------
\subsection{Other Methods}
%-----------------------------------------------------------------------
Reviews, Inspections, static analysis and walkthroughs, mostly manual techniques, are also to be considered for the verification of models. 


The inputs required by this task are
\begin{itemize}
\item[•] Requirements for the model provided by ERA, SSRS and safety analisys;
\item[•] D2.9 - Set of requirements for V\&V;
\item[•] D7.2 - Report on all aspects of secondary tooling;
\item[•] D7.4 – Tool chain first release;
\item[•] Model of the system (WP3).
\end{itemize}	
		
As the goal of this task is the verification and validation of the formal model using the selected tool chain. We will also provide a sequence in which these tools should be executed to check the correctness of the model (e.g., simulation can be used to check properties that cannot be checked by model checking and/or proof techniques).

We propose to start the procedure of tools validation by the application of the following process:

\begin{enumerate}
\item To check tools based on proof techniques in order to evaluate what requirements –properties of the model they are able to validate. This evaluation will be based on the work performed in WP7, Task 2; 
\item To check what are the properties that can be evaluated using model checking, in particular we will verify what properties can checked by model checking and not by proof techniques;
\item To check what are the properties that can be evaluated using simulation, in particular we will verify what properties can checked by simulation and not by model checking nor by proof techniques;
\item For all tools:  to precise if the tools satisfy CENELEC requirements;
\item We will provide a report on the applicability of the tools to check model correctness.
\end{enumerate}

We expect from the partners participating to T4.2, to send to us:
-	a short description of the methods used for the model description;
-	first description of the models elaborated by the partners.





\begin{table}[h]
\caption{T4.2 Inputs, Outputs and Deliverables} %title of the table
\begin{adjustbox}{width=\textwidth}
\begin{tabular}{|l|l|r|r|r|}
\hline
\multicolumn{5}{|c|}{\textbf{T4.2 \VV of the Formal Model}} 
\\\hline
Type & Description & Due Date & Due Month & status 
%status output going to other tasks/wps    : not started, started, complete
%status input coming from other tasks/wps: no, yes
%\\\hline
%$\rightarrow$ & \todo{Ox.2.3: Sample Input Information}  & \shortmonthname[1]-2014 & T0+19 & no 
%\\\hline
%$\leftarrow$ & \todo{O4.2.1: Sample Output Information}   & \shortmonthname[10]-2013  & T0+16 & started  
\\\hline
D & \emph{D 4.4} Final report on \VV of the model  & \shortmonthname[6]-2015 & T0+36 & not started
\\\hline
\end{tabular}
\end{adjustbox}
\end{table}

%-----------------------------------------------------------------------
\section{V\&V of the Implementation \& Code}
%-----------------------------------------------------------------------

The objective of this task is to verify and validate the actual implementation of the formal model.
Therefore the tool chain from WP3\slash WP7 will be used to apply the chosen methods from
WP4 Taskr~ 1 onto the implementation of the formal model from WP3.
The chosen combination of methods and tools in WP4 Task 1 can result
in a wide variety of techniques to be used:

\begin{itemize}
\item Software-in-the-Loop
\item  Model-in-the-Loop
\item  Model-based testing
\item  Static analysis (e.g. coding guidelines)
\item  Formal methods (abstract intertpretation, deductive verification, model checking)
\item  Monitoring
\end{itemize}

Analogue to WP4 Task 2 the verification and validation of the formal model
implementation is part of the development chain.
Therefore this task runs parallel to the development of the formal model in WP3, and is being applied iteratively.
Therefore feedback regarding the validity and correctness is reported to the
development process in WP3.
The results of this task are the verification and validation specifications, that is,

\begin{itemize}
\item 
how to perform the V\&V on the formal model implementation,

\item
the basic materials (the actual tests cases, checklists, etc.) and

\item
the V\&V report on the implementation of the formal model.
\end{itemize}

As first steps the relevant properties and techniques concerning the code and
implementation are to be identified. 
We list therefore the following software properties that we think are most relevant for \vv:

\begin{itemize}
\item functionality
\item robustness (absence of runtime errors)
\item performance
\item real time behaviour
\item dataflow
\item absence of deadlocks
\end{itemize}

Table~\ref{tbl:task43} shows the main deliverables of Task~4.3.

\begin{table}[h]
\begin{adjustbox}{width=\textwidth}
\begin{tabular}{|l|l|r|r|r|}
\hline
\multicolumn{5}{|c|}{\textbf{T4.3 V\&V of the implementation \/ code}} 
\\\hline
Type & Description & Due Date & Due Month & status 
\\\hline
%status output going to other tasks/wps    : not started, started, complete
%status input coming from other tasks/wps: no, yes
 D &\emph{D 4.2} Interim report on the applicability of the V\&V approach  & \shortmonthname[12]-2013 & T0+18 & started
\\\hline
 D &\emph{D 4.4} Final Report on the V\&V of the Implementation & \shortmonthname[1]-2015 & T0+31 & not started
\\\hline
\end{tabular}
\end{adjustbox}
\caption{\label{tbl:task43} Inputs, Outputs and Deliverables of Task 4.3}
\end{table}


%-----------------------------------------------------------------------
\section{Verification of the Tools and Processes}
%-----------------------------------------------------------------------
Since one of the openETCS project goals is to define processes and a corresponding tools which are suited to develop ETCS train Onboard Unit software based on a model of the system requirement specifications, the development processes have to follow the CENELEC Standards foremost EN 50128. As this should be done using open source principles and agile development methods it has to be demonstrated that all processes have been applied properly respecting the required principles. Therefore all lifecycle stages (SW Requirement Specification, SW Design, SW Coding, etc.) and their related verification and validation activities have to be designed, performed and documented consistently and as required by the CENELEC standards. 

The work of this task 4.4 will verify proper design, performance and documentation of the overall development process and its corresponding tools by inspection of the respective documents. According to EN 50129 a safety plan as documentary evidence that during the development of an safety-related electronic railway system the conditions for safety acceptance a satisfied has to be submitted to the relevant safety authorities. Therefore a safety case covers evidence of quality management, safety management, and functional and technical safety in a structured justification. Since  the main product of the openETCS project shall be a non-vital demonstrator implementation it is unnecessary and infeasible for task 4.4 to write a complete safety case. First and foremost the documentation of the actual openETCS development will cover quality management aspects (among others):
\begin{enumerate}
\item  Documentation of the development process
\item  Roles, responsibilities and competences of the involved bodies
\item  Traceability during the development
\item  Documentation control and Configuration management
\item  Fault management
\item  Grievance handling
\item  Modification and change control
\item  Tool functionality and tool handling documentation
\end{enumerate}

The safety related verification of tools and processes has to deal with the to aspects, safety management and the functional (and technical) safety. However, the openETCS project will not deliver a vital software implementation consequently most of the activities will not or only for demonstration purposes be performed in the project. Respectively instead of verifying and documenting the actual activities in a safety case, the work of task 4.4 will rather be to present a generic justification structure for a safety case, which can be used in following projects or by the industrial partners for the development of an openETCS based Onboard Unit. The partial execution of safety activities shall be used to demonstrated the structure and to show their serviceability. 

To reach this goal the safety management team working in task 4.4 will design a safety plan to outline the overall structure of the safety management and the pursued way to demonstrate the functional safety. Accordingly the sequence of safety activities will be identified including all verification and validation activities for safety requirements. Overall the safety plan will describe the following points:
\begin{enumerate}
\item  Safety specific roles, responsibilities and competences of the involved bodies
\item  Principles and overall process of the safety management
\item  Requirements on the different tools supporting the development process and criteria for the tool categorization
\item  Safety related activities and all their corresponding documents to prove functional (and technical) safety (including those for the supporting tools)
\item  Structure of the safety case based on the documentation of the safety activities and their supporting tools
\item  Principles and procedures for preparing the safety case
\item  Specific procedures for maintaining the functional safety and the corresponding safety documents over time
\end{enumerate}

All these activities are closely connected to the design, verification and validation activities and therefore have to be defined considering the input of this teams. Especially concerning the applied tools, it is likely that 3rd parties have to be engaged to perform the verification and validation or deliver the needed documentation.

Following the safety plan sample activities shall be chosen to be performed on part of the Onboard unit. This shall give evidence that the presented safety management is exercisable and can be serviced by the used tools. Additionally it shall demonstrate the safeguards used for individual safety properties to ensure functional and technical safety. 
Since in a model based development the hazard identification, risk assessment and safety requirement verification is very closely related to the design activities and the supporting tools, safety activities have to be performed mainly by other teams in the development process. The safety team of task 4.4 itself can only guide these activities and process the results to present them in the safety case. 




%Specify the tool classes and the potentials to Categorize which tools have to be T3 and which T2.}

%Safety Evaluation criteria

\begin{table}[h]
\caption{T4.4 Inputs, Outputs and Deliverables} %title of the table
\begin{adjustbox}{width=\textwidth}
\begin{tabular}{|l|l|r|r|r|}
\hline
\multicolumn{5}{|c|}{\textbf{T4.4 Verification of the Tools and Processes}} 
\\\hline
Type & Description & Due Date & Due Month & status 
%status output going to other tasks/wps    : not started, started, complete
%status input coming from other tasks/wps: no, yes

%\\\hline
%$\rightarrow$ & \todo{Ox.2.3: Sample Input Information}  & \shortmonthname[1]-2014 & T0+19 & no 
%\\\hline
%$\leftarrow$ & \todo{O4.2.1: Sample Output Information}   & \shortmonthname[10]-2013  & T0+16 & started 
\\\hline
$\rightarrow$ & \emph{D 1.3.1} Project Guide on Quality Assurance  & \shortmonthname[6]-2013 & T0+12 & no 
\\\hline
$\rightarrow$ &	\emph{D 2.4} Methods definition & \shortmonthname[2]-2013 & ? & no
\\\hline
$\rightarrow$ &	\emph{D 2.6-9} Set of Requirements & \shortmonthname[6]-2013 & ? & no
\\\hline

$\rightarrow$ &	\emph{D 7.1} Report on the final choice(s) for the primary tool chain (means of description, tool and platform) & \shortmonthname[6]-2013 & ? & no
\\\hline
$\rightarrow$ &	\emph{D 7.2} Report on all aspects of secondary tooling & \shortmonthname[6]-2013 & ? & no
\\\hline
$\rightarrow$ &	\emph{I: O 7.2.8} Safety analyses tools choices & \shortmonthname[6]-2013 & ? & no
\\\hline
$\rightarrow$ &	\emph{D 7.3.1.2} Tools Interoperability Description & \shortmonthname[6]-2013 & ? & no
\\\hline
$\rightarrow$ &	\emph{I: O 7.3.1} Tool chain development plan (or equivalent) & \shortmonthname[6]-2013 & ? & no
\\\hline
$\rightarrow$ &	\emph{I: O 7.3.2} Specification of tool interoperability mechanisms & \shortmonthname[6]-2013 & ? & no
\\\hline
$\rightarrow$ &	\emph{I: O 7.3.4} Specification of primary and support tool chain architecture and its embedding into the platform & \shortmonthname[6]-2013 & ? & no
\\\hline
$\rightarrow$ &	\emph{D 7.3} Tool Chain Qualification Process Description & \shortmonthname[6]-2013 & ? & no
\\\hline
$\rightarrow$ &	\emph{D 7.4} Tool chain first release & \shortmonthname[2]-2014 & T0+20 & no
\\\hline
$\rightarrow$ & \emph{WP3} Feedback concerning the quality and safety management processes	 & \shortmonthname[6]-2013 & ? & no
\\\hline
$\rightarrow$ & \emph{WP3} Feedback concerning potential hazards & \shortmonthname[6]-2013 & ? & no
\\\hline
$\rightarrow$ & \emph{D 4.1} Report on V\&V Plan \& Methodology & \shortmonthname[7]-2013 & T0+13 & no
\\\hline
$\rightarrow$ & \todo{I: O Task 4.2}  & \shortmonthname[7]-2013 & ? & no
\\\hline
$\rightarrow$ & \todo{I: O Task 4.3}  & \shortmonthname[7]-2013 & ? & no
\\\hline
$\rightarrow$ & \emph{WP4 - T 4.5} Feedback concerning documentation quality  & \shortmonthname[7]-2013 & ? & no

\\\hline
$\leftarrow$ & \emph{O 4.4.1} Safety Plan   & \shortmonthname[10]-2013  & T0+16 & started 
\\\hline
$\leftarrow$ & \emph{O 4.4.2} Report on safety demonstration activities   & \shortmonthname[2]-2014  & T0+20 & not started 

\\\hline
D & \emph{D 4.4} Final report concerning the Safety Case  & \shortmonthname[6]-2015 & T0+36 & not started
\\\hline
\end{tabular}
\end{adjustbox}
\end{table}

%-----------------------------------------------------------------------
\section{Internal Assessment}
%-----------------------------------------------------------------------
One of the major point for a SIL4 compliant Software is the Whole Software Development Project Assessment by a Safety Authority (e.g CERTIFER in France, TÜV in Germany). As none of these companies are involved in openETCS Software Development assessment, the Internal Assessment objective is to simulate a real Assessor's tasks during the whole Open ETCS Software Development activities.

An assessment is a ¨ Process of analysis to determine whether software, which may include process, documentation, system, subsystem hardware and/or software components, meets the specified requirements, and to form a judgment whether the software is fit for its intended purpose. Safety assessment is focused on but not limited to the safety properties of a system.¨
%-----------------------------------------------------------------------
\subsection{Assessment tasks}
%-----------------------------------------------------------------------
The Assessor shall write a Software Assessment Plan. It is like an assessment process which is linked to the software development process.
More precisely, he shall explain the tasks needed to assess the software of the project OpenETCS.

{\itshape 
Note: The Verifier shall write a Software Assessment Verification Report, as required in the standard EN50128, to verify in the first time that the Software Assessment Plan meets the general requirements for readability and traceability.
}

During the software development, he shall evaluate the software verification and validation activities.
We propose that the Assessor intervenes at least seven times during the software development process (this is equivalent to one time at least by Work Product).

\textit{
Note: the numbers of work packages are not given in the chronological order, e.g. WP1 is performed during all the development process and WP5 occurs before the end of WP4.
}


\textbf{
During WP1: Project Management.
}

The Assessor is able to assess:
\begin{itemize}\itemsep=0pt
  \item The Quality Assurance
  \item The capability of the Project Manager and the quality of his deliverables
 \end{itemize}

The Assessor shall assess the Software Quality Plan. We propose that he gives a formal approval of this document.


\textbf{
During WP2: Requirements for Open Proof.
}


The Assessor is able to assess:
\begin{itemize}\itemsep=0pt
  \item The System requirements specification, including:
  \begin{itemize}\itemsep=3pt
    \item functions and interfaces;
    \item application conditions;
    \item configuration or architecture of the system;
    \item hazards to be controlled;
    \item safety integrity requirements;
    \item apportionment of requirements and allocation of SIL to software and hardware;
    \item timing constraints
   \end{itemize}
  \item The software requirements specification,
  \item The software architecture and design specification,
  \item The software component specification,
  \item The personnel key roles, responsibilities and competence,
  \item The Quality Assurance
 \end{itemize}
Nevertheless, he shall assess the implementation of both activities and deliverables of WP 2.


\textbf{
During WP3: Modeling of (part of) ETCS specification.
}

The Assessor shall evaluate the software implementation respectively the software modeling.
Furthermore, he is able to assess:
\begin{itemize}\itemsep=0pt
  \item A part of the life cycle and the documentation,
  \item The Quality Assurance,
  \item The personnel roles and responsibilities and competence.
 \end{itemize}
The Assessor shall assess the implementation of both activities and deliverables of WP 3.


\textbf{
During WP4: Validation \& Verification Strategy.
}

The Assessor shall assess:
\begin{itemize}\itemsep=0pt
  \item the Software Verification Plan and the Software Validation Plan,
  \item the Quality Assurance.
 \end{itemize}
We propose that he gives a formal approval on these documents. 
He shall mainly evaluate the verification activities and the implementation of both activities and deliverables of the WP 4.


\textbf{
During WP5: Demonstrator.
}

The Assessor shall assess the specific openETCS software.
Indeed, before the beginning of the validation activity (WP4), the Assessor shall assess the Software Integration Test Report to give or not the approval for software validation. This point is the Validation first step (the previous steps are related to the verification).


\textbf{
During WP6: Dissemination, Exploitation and Standardization.
}

The Assessor shall verify that the software maintenance plan is written and compliant with the software safety integrity level (SIL4).

\textbf{
During WP7: Language, Tool Chain and Open source Ecosystem.
}

The Assessor shall assess the developed tool chain according to Tool class T3 of EN 50128:2011. The other tools (T2) have to be assessed as well, but the effort is liter regarding the T3 assessment effort.

At the end of the software development process, the Assessor shall perform a final assessment. Indeed, he shall evaluate that the life cycle processes and products resulting are such that the software is of the defined software safety integrity level and fits for its intended application. All the steps of assessment performed during the software development process shall be gathered in the Software Assessment Report. This report could be updated all along the process.

{\itshape
Note: the Software Assessment Verification Report permit to verify the internal consistency of the Software Assessment Report.
}


\begin{table}[h]
\caption{T4.5 Inputs, Outputs and Deliverables} %title of the table
\begin{adjustbox}{width=\textwidth}
\begin{tabular}{|l|l|r|r|r|}
\hline
\multicolumn{5}{|c|}{\textbf{T4.5 Internal Assessment}} 
\\\hline
Type & Description & Due Date & Due Month & status 
%status output going to other tasks/wps    : not started, started, complete
%status input coming from other tasks/wps: no, yes
%\\\hline
%$\rightarrow$ & \todo{Ox.2.3: Sample Input Information}  & \shortmonthname[1]-2014 & T0+19 & no 
%\\\hline
%$\leftarrow$ & \todo{O4.5.1: Sample Output Information}   & \shortmonthname[10]-2013  & T0+16 & started  
\\\hline
 D &\emph{D 4.5} Quality recommendation to prepare the Assessment  & 29/03/2013 & March & 
\\\hline
\end{tabular}
\end{adjustbox}
\end{table}






\noindent{
\begin{landscape}
%-----------------------------------------------------------------------
\section{GANTT chart}
%-----------------------------------------------------------------------

\begin{table}[h]
%\caption{WP4 GANTT chart} %title of the table
\begin{adjustbox}{height=\textheight/4*3}% ajusting graphic size for landscape
%\begin{adjustbox}{width=\textwidth}% ajusting graphic size for non-landscape
\begin{tikzpicture}[x=.5cm, y=1cm]
\begin{ganttchart}%
[hgrid=true, vgrid={*5{dotted},*1{solid},*5{dotted},*1{dashed}},%
today=8,
today label=\textcolor{blue}{Current Month}, 
today rule/.style={blue, line width=3pt},
y unit title=0.4cm,
y unit chart=0.5cm,
title label anchor/.style={below=-1.5ex},
title height=1,
bar height=.6,
bar label font=\normalsize\color{black!80},
milestone height=.6,
milestone yshift=.6,
milestone/.style={fill=black,draw=black},
group right shift=0,
group top shift=.6,
group height=.3,
group peaks={}{}{.2},
inline
]{36} %36 months

% project title
\gantttitle[title/.style={draw=none}, title height=1,
title label font={\color{black}\scshape}%
]{Verification \& Validation Strategy}{36} \\

%timing header
\gantttitle{2012}{6}
\gantttitle{2013}{12}
\gantttitle{2014}{12}
\gantttitle{2015}{6} \\
\gantttitlelist[title height=1]{7,...,12}{1}
\gantttitlelist[title height=1]{1,...,12}{1}
\gantttitlelist[title height=1]{1,...,12}{1}
\gantttitlelist[title height=1]{1,...,6}{1} \\
\gantttitlelist[title height=1]{1,...,36}{1} \\

%project groups and tasks
\\
\ganttbar[name=T41, inline=false]{Idetntification of tools and profile usage \emph{T4.1}}{7}{13} 
\ganttmilestone[name=D41a, bar label inline anchor/.style=left]{\emph{D 4.1a}}{9}  %Preliminary Evaluation criteria on V\&V \emph{D 4.1a}
\ganttmilestone[name=D41,  bar label inline anchor/.style=left]{\emph{D 4.1}}{13} \\ %V\&V Plan \& Methodology \emph{D 4.1}
\\
\ganttgroup[]{V\&V of prototypical Model}{14}{16} \\
\\
\ganttgroup[]{V\&V of Model \& Functional API propotype}{20}{24} \\
\\
\ganttgroup[]{V\&V of Model \& Functional API final}{32}{35} \\
\\
\ganttbar[name=T42, inline=false]{V\&V of the formal model \emph{T4.2 }}{10}{35}
\ganttmilestone[name=D42]{\emph{D 4.2}}{16} %Interim report on the applicability of the V\&V approach \emph{D 4.2}
\ganttmilestone[name=M41]{\emph{M 4.1}}{24} %Applicability of the V\&V approach to the prototype \emph{M 4.1}
\ganttmilestone[name=D43, , milestone label inline anchor/.style={anchor=south east}]{\emph{D 4.3}}{35} %Report on the prototypical application of the V\&V \emph{D 4.3}
\ganttmilestone[name=D45]{{\emph{D 4.5}}}{36} \\ %Final report and conclusions \emph{D 4.5}
\\
\ganttgroup[]{V\&V of prototypical Code \& API}{14}{16} \\
\\
\ganttgroup[]{V\&V of Architecture \& System API propotype}{21}{24} \\
\\
\ganttgroup[]{V\&V of Architecture \& System API final}{32}{35} \\
\\
\ganttbar[name=T43, inline=false]{V\&V of the implementation \/ code \emph{T4.3}}{10}{35} 
\ganttmilestone[name=D42]{\emph{D 4.2}}{16} %Interim report on the applicability of the V\&V approach \emph{D 4.2}
\ganttmilestone[name=M41]{\emph{M 4.1}}{24} %Applicability of the V\&V approach to the prototype \emph{M 4.1}
\ganttmilestone[name=D43, milestone label inline anchor/.style={anchor=south east}]{\emph{D 4.3}}{35} %Report on the prototypical application of the V\&V \emph{D 4.3}
\ganttmilestone[name=D45]{{\emph{D 4.5}}}{36} \\ %Final report and conclusions \emph{D 4.5}
\\
\ganttbar[name=T44, inline=false]{Verification of the tools and processes \emph{T4.4}}{7}{35} 
\ganttmilestone[name=D41a]{\emph{D 4.4a}}{9} %Preliminary Evaluation criteria on safety \emph{D 4.4a}
\ganttmilestone[name=D43, milestone label inline anchor/.style={anchor=south east}]{\emph{D 4.3}}{35} %Report on the \ganttmilestone[name=D45]{{\emph{D 4.5}}}{36} \\ %Final report and conclusions \emph{D 4.5}
\\
\ganttgroup[]{\tbd}{8}{35} \\
\\
\ganttbar[name=T45, inline=false]{Internal Assessment \emph{T4.5}}{8}{35}
\ganttmilestone[name=D45]{\emph{\todo{D 4.5}}}{35} \\  %Quality recommendation to prepare the Assessment \emph{D 4.5}

%project linking between tasks
%\ganttlink{D41a}{T43}
%\ganttlink[link type=s-s]{T41}{T42}

\end{ganttchart}
\end{tikzpicture}
\end{adjustbox}
\end{table}

\end{landscape}
}

\nocite{*}
%===================================================
%Do NOT change anything below this line

\end{document}
