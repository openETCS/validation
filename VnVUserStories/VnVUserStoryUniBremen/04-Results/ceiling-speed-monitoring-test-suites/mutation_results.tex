% .......................................................................
\subsection{Test Strength Comparison}\label{sec:mutations}

\subsubsection*{Experimental Setup.}

The strength of a test suite is the ability to detect real faults. 
\cite{just_are_2014} shows that mutants are a valid substitute for real faults
when comparing generated test suites. Mutants are modifications of the SUT with (automatically) seeded single faults. 
Mutants in large numbers can be generated 
by mutation operators, which apply slight syntactical changes
to the original code.
To compare the test strength of the generated test suites we measured the mutation score of 
each test suite. The mutation score is the ratio of mutants that were
``killed'' by a test suite to the total number of \textit{real} mutants\footnote{Only non-I/O-equivalent mutants are considered, since not every 
syntactical modification results in a mutant showing deviating observable behaviour.}. A mutant is killed, if at least one test case of the test suite 
does not pass.

For the experimental evaluation we generated a correct Java implementation
from the model shown in section~\ref{chap:model}. The java implementation was performed by
hand in a straight forward way, resulting in 176 lines of code. Next, mutants
were automatically generated from each implementation with the Major mutation
framework~\cite{Just2014}.
All applicable mutation operators were executed to generate single-fault
mutants. For our concrete implementations these
operators were as follows:
arithmetic operator replacement (AOR),
conditional operator replacement (COR), relational operator replacement (ROR),
statement deletion (STD) and literal value replacement (LVR).
Note that the mutation tool is unaware of any conformance relation.
Therefore we manually investigated the generated mutants. Discarding I/O-equivalent mutants resulted  in
a collection of 186 erroneous
implementations.
Finally, the three test suites specified above were executed against
all mutants to measure the mutation score of the
test suites.




\subsubsection*{Experimental Results.}

Table~\ref{tab:mutation} shows the mutation scores of the test suites.
The manually defined test cases achieve the lowest mutation score of all
examined strategies. This seems to correlate with the low model coverage shown in the previous section. 
Further, the mutation score of the model
derived tests is superior to the manually defined tests, as already expected by
the higher (complete) model coverage.
The last column of Table~\ref{tab:mutation} confirms, that for our model
the equivalence class testing strategy behaves best in revealing the faults
injected by the mutation tool.
The test suite revealed all but one mutation.\footnote{Note that this result differs slightly from the result in \cite{huebner15}. There all mutants were killed.
The reason for this deviation results from a different model we use in
this paper. This model has more inputs than the reduced
version in \cite{huebner15}.} Although the model derived test cases yield a full
coverage of the test model, this criterion is not sufficient to reveal a
considerable number of faults. 
However, the equivalence class testing approach uses more rigorous criteria and
this results in the detection of defects, which would not be revealed by test
suites considering only model coverage criteria as described in section~\ref{sec:automated_ETCS_test}. 
Note that the large number of test procedures 
needed for the equivalence class tests is due to the detailed refinement and the use of
boundary value tests. The equivalence class testing strategy can be applied with a coarser
IECP for the cost of lower test strength. The reader may refer to \cite{huebner15} for a detailed evaluation of the different
refinements and the impact on size and mutation score. 

\begin{table}
\caption{Mutation Score Overview.}
\centering
\begin{tabular}{p{40mm}p{18mm}p{18mm}p{24mm}p{18mm}}
\hline\hline
&
{\bf Manually Defined Tests} &
{\bf Extended Manual Tests} &
{\bf Automatically Defined Tests} &
{\bf Equivalence Class Testing}
\\\hline
{\bf Number of Test Cases} &
9 &
11 &
45 &
5524
\\\hline
{\bf Mutation Score}&
108 of 186\newline(58 \%)&
116 of 186\newline(62 \%)&
126 of 186\newline(68 \%)&
186 of 186\newline(100 \%)
% \\\hline
% {\bf Mutation Score} of equal sized random tests&
% 115 of 358\newline(32 \%)&
% todo of 358\newline(todo \%)&
% 129 of 358\newline(36 \%)&
% 336 of 358\newline(91 \%)
\\\hline\hline
\end{tabular}
\label{tab:mutation}
\end{table}%
