\documentclass{article}

\usepackage{verbatim}
\usepackage{hyperref}
\usepackage{lineno}
\newcommand{\FIXME}[1]{\marginpar{FIXME}\textsf{FIXME: #1}}
\linenumbers


 \title{Verification Report for Architecture and Design of Train Positioning \\Version 0.1}
 \author{Marc Behrens (DLR), Bernd Gonska (DLR),\\ Jens Gerlach (Fraunhofer), Bernd Hekele (DB),\\ Jan Welte (TU-BS)} 
 \date{Oct 29, 2014}

%based on VnVRprtTmplt-131101-02.tex by Hardi Hunger

\newcommand{\tbi}[1]{$<$\textit{#1}$>$}

% Starts a new line nearly everywhere
\newcommand{\nl}{\mbox{}\\}
\newcommand{\nlskip}[1]{\mbox{}\\[#1]}

%
%Comments
\newcommand{\cmmnt}[1]{\framebox{#1}}
\newcommand{\bgcmmnt}[1]{\nl\framebox{\parbox{.95\textwidth}{#1}}\nl[2mm]}
%\renewcommand{\bgcmmnt}[1]{}
%

\newcommand{\eod}{\nl\rule{.95\textwidth}{1pt}\nl\textit{End of Document}}

\begin{document}
\maketitle

\begin{abstract}

This verification report presents the verification results for the architecture, interfaces and design artifacts for the component "Train Positioning" in the overall openETCS Kernel architecture.

\begin{comment}
This template provides the required content to complete the verification of architecture and design artifacts.
To close the development phase for this artifact all required information shall be given, even if it can only be stated that specific aspects are missing in the artifact due to open points in related artifacts. 

The template should be used as a guideline to check whether all
information is given appropriately. The wording used in this proposal
is by no means mandatory. And if you feel that more information is
useful to describe your activity within the context of openETCS, you
should of course do so. Feel free to add additional categories of
description as adequate. 

Also the \LaTeX{} macros may be changed, though the use of
\texttt{paragraph} and \texttt{subparagraph} enables easy integration
into higher-level documents (they are not numbered automatically,
which may be a drawback in other respects). 
\end{comment}
\end{abstract}

\section{Roles}

\begin{itemize}
\item Fausto Cochetti - Design (PM)
\item Jens Gerlach - Verification of SW/Implementation
\item Uwe Steinke - Design/Implementation
\item Jan Welvaarts - Design
\item Vincent Nuhaan - Simulation/Design
\item Bernd Gonska - Verification
\item Marc Behrens - Verification
\item Jan Welte - Verification
\item Bernd Hekele - Verification
\end{itemize}


\section{Verification Object}

\input{verification_object}


\section{Software Architecture, Interface and Design Verification}

This section presents all verification results concerning the verification object \texttt{Train Positioning}. 

\begin{comment}
\bgcmmnt{For all verification aspects addressed in the following section the following 3 points shall be state clearly:
\begin{enumerate}
\item Responsible verifier
\item Use verification strategy and technique (with reference to the V\&V Plan)
\item Verification results (level of conformity, detected errors or deficiencies and made assumptions)
\end{enumerate}
\end{comment}

The following subsection present all different verification aspects in accordance with EN 50128 7.3.4.42 \cite{EN50128}.

\subsubsection{Internal Consistency}

\textit{by Jan Welte and Marc Behrens}

Contant:
\begin{itemize}
\item relations
\item historical development
\item claim of same approach
\item of naming between documents consistence 
\end{itemize}

Are the internal functional allocation and all related input and output consistent?


\subsubsection{Adequacy to fulfill Software Requirements}

by Bernd Gonska

content
\begin{itemize}
\item fulfilled SRS functionality 
\item differnces in approach to SRS
\end{itemize}

Are the listed functions and all input and outputs adequate to cover the intended Software Requirements?
Therefore the following 2 aspects shall be assess:

\begin{itemize}
\item Consistency
\item Completeness
\end{itemize}


\subsubsection{Readability and Traceability}

by Marc Behrens

Content
\begin{itemize}
\item traceability of requirements
\item unique references
\end{itemize}


Are all related system and software requirements uniquely referenced and is the relationship to other documents clearly defined?
Are all parts of the architecture and inputs and outputs referenced to the related requirements.
Are the elements referred to in the same way in all documents?

** 10' item: look for findings inside the two verification reports
**  10' structure: two chapters, one for each report
** 20' 3 documents: readability: statistics: wordlength + syllable over sentence length
** 20' look in 2.x: traceability to openetcs req
** 10' only 3.6 available traceability to SRS requirements
*** high level
** traceability to TSI requirements
*** 20' are there more req in TSI to positioning?
*** 10' what is missing
*** 15' design reasoning: what is needed for performance resons
*** 5' what is the least cycle time
*** 5' are realtime requirements defined on architecture level?
** 10' traceability to higher level artifacts
*** 10' high level requirements were defined during workshop as RO US 
  who will be response 
** 10' application of glossary
*** 5' subset-023  
*** 5' openETCS glossary
*** 5' make jan's document openETCS licensed + upload



\subsubsection{Consideration of hardware and software constraints}



Hardware design is out of scope of the openETCS project. 
The considerations related to hardware are planned to be based on assumptions which have not yet been formulated.

\noindent 
Software constraints encompass 

\begin{itemize}
   \item restrictions implied by the coding standards and software design method
   \item timing\slash performance constraints
   \item memory constratints 
   \item constraints implied by 
         the interfacing system (e.g. decoder and encoder functions)
   \item the operating system
\end{itemize}




\section{Conclusions}
Concluding the meeting the following points were agreed on as action item list:

\subsection{Implementation and Design}

Differences within the simulation and implementation has been detected
and the two approaches have to be merged (Jan: Specify, Uwe: Implement)
\begin{enumerate}
\item synchronize on using the inaccuracies of the BG passed or just of the last BG and the 1st BG (Uwe and Jan Welvaarts)
\item backwards calculation of inaccuracies need to be synchronized (Uwe, Jan)
\item agree on the output data (Uwe, Jan Welvaarts, Vincent)
\item synchronize data structure for balise storage and storage of other trackside related information
\item agree on the input data (Uwe, Jan Welvaarts, Vincent)
\item integer arithmetic only on the model (design decision of WP3)
\item resolution of train position in cm (design decision of WP3)
\item Inaccuracy of the Odometer is agreed to be taken into account (Uwe)
\item Center Detection accuracy is agreed to be taken into account (Uwe)
\item define national values to be used (Jan Welvaarts)
\item national values can also change at some point on track (Uwe)
\item a as common reference for saving the data the start-up position of the OBU is taken
\item rules have to be implemented correctly
\item  Q\_LOCACC will be made dependant on the national values and on the linking information (Uwe, Vincent)
\item  Implement distance calculation to track objects, other than balises (Uwe)
\end{enumerate}

\subsection{Scenarios}

Concerning the scenario simulation the following was decided:
\begin{enumerate}
\item Two scenarios to contain all possible situations are to be sketched (Jan)
\item Calculate distance related values with the LabView Simulation to be compared with the SCADE execution (Vincent)
\end{enumerate}

\subsection{Report}

Concerning the follow up of the verification report the following points were decided:
\begin{enumerate} 
\item Verification report is to be written (Marc, Jan Welte, Jens Gerlach, Bernd Gonska)
\item once the design is merged and implemented the verification will rework the results based on the merged version (Marc)
\item traceability to the SRS will be delivered (Jan, Marc)
\item justification on the deviations need to be documented (Jan, Uwe)
\end{enumerate}
               
\bibliographystyle{plain}
\bibliography{verification}

\end{document}

