\newpage
\section{Test Procedures}\label{sec:conventionaltests}


\subsection{Test Automation Tool}

The tests for the CSM provided under www.mbt-benchmarks.org have been performed with the model-based testing component RTT-MBT of the test automation tool RT-Tester~\cite{EPTCS111.1}.
RT-Tester supports all test levels from unit testing to system
integration testing and provides different functions  for manual test
procedure development, automated test case, test data and test procedure
generation, as well as management functions
for large test campaigns. The typical application scope covers (potentially
safety-critical) embedded real-time
systems involving concurrency, time constraints, discrete control decisions as
well as integer and floating point data and calculations. While the tool has
been used in industry for about 15 years and has been qualified for avionic, automotive and railway 
control systems under test according to the standards~\cite{do178b,iso26262-4,CENELEC50128}, its MBT functionality is based on more recent extensions that have been validated during the
last years in various projects from the transportation domains and are  
now made available to the public.

% =========================================================================
\subsection{Test Categories}

The tests presented for the CSM under www.mbt-benchmarks.org can be structured according to the following coverage strategies. 
\begin{enumerate}
  \item Test procedures aiming at  basic state coverage
  \item Test procedures aiming at    transition coverage
  \item Test procedures aiming at   MC/DC coverage
  \item Test procedures based on  user defined test cases, aiming at requirements coverage
  \item Test procedures based on input equivalence class partitioning principle described in this technical report. These are described in Section~\ref{sec:iecptestresults}.
\end{enumerate}

\begin{table}[htbp]
\centering
\caption{\label{table:tp} Test procedures}
\bgroup
\def\arraystretch{1.5}% 
\begin{tabular}{|l|c|c|c|c|c|c|}\hline\hline
TP Name & MCDC & UD & HITR & TR & BCS & REQ \\\hline
TP-001-BCS & 42\% & 15\% & 0\% & 45\% & 100\% & 12\% \\\hline
TP-001-BCS-NOSB & 42\% & 15\% & 0\% & 45\% & 100\% & 12\% \\\hline
TP-002-OnlyCSM\_ON & 57\% & 23\% & 0\% & 81\% & 100\% & 40\% \\\hline
TP-002-OnlyCSM\_ON-NOSB & 57\% & 23\% & 0\% & 81\% & 100\% & 36\% \\\hline
TP-002-Transition & 64\% & 38\% & 100\% & 90\% & 100\% & 56\% \\\hline
TP-002-Transition-NOSB & 78\% & 23\% & 100\% & 100\% & 100\% & 52\% \\\hline
TP-003-MCDC & 85\% & 15\% & 0\% & 54\% & 100\% & 20\% \\\hline
TP-003-MCDC-NOSB & 85\% & 30\% & 0\% & 54\% & 100\% & 20\% \\\hline
TP-004-3.13.10-DMIOutputs & 85\% & 46\% & 100\% & 90\% & 100\% & 80\% \\\hline
TP-004-3.13.10-DMIOutputs-NOSB & 85\% & 61\% & 100\% & 90\% & 100\% & 80\% \\\hline
TP-005-3.13.10-EBvsSB & 64\% & 30\% & 20\% & 90\% & 100\% & 48\% \\\hline
TP-006-3.13.10-IndicationStatus & 57\% & 38\% & 60\% & 63\% & 100\% & 24\% \\\hline
TP-006-3.13.10-IndicationStatus-NOSB & 57\% & 46\% & 60\% & 63\% & 100\% & 28\% \\\hline
TP-007-3.13.10-RevokeCmd & 57\% & 15\% & 0\% & 81\% & 100\% & 36\% \\\hline
TP-007-3.13.10-RevokeCmd-NOSB & 64\% & 30\% & 0\% & 81\% & 100\% & 36\% \\\hline
TP-008-3.13.10-Trigger & 85\% & 38\% & 100\% & 90\% & 100\% & 76\% \\\hline
TP-008-3.13.10-Trigger-NOSB & 85\% & 69\% & 100\% & 90\% & 100\% & 84\% \\\hline
\hline
\end{tabular}

%\begin{tabular}{|l|c|c|c|c|c|}\hline\hline
%  Test procedures    & BC    & TC    & MC/DC & Requirements & time\\\hline\hline
%  TP-001-BCS         & 100\% & 37\%  & 35\%  &   8\%        & 2s \\\hline
%  TP-002-Transitions & 100\% & 100\% & 78\%  &   64\%       & 6s\\\hline 
%  TP-003-MC/DC       & 100\% & 75\%  & 100\% &   20\%       & 20s \\\hline 
%  TP-004             & 100\% & 93\%  & 57\%  &   80\%       & 16s \\\hline 
%  TP-005             & 100\% & 69\%  & 78\%  &   52\%       & 3s \\\hline 
%  TP-006             & 100\% & 62\%  & 57\%  &   28\%       & 4s  \\\hline 
%  TP-007             & 100\% & 56\%  & 64\%  &   52\%       & 5s  \\\hline 
%  TP-008             & 100\% & 93\%  & 78\%  &   84\%       & 25s \\\hline 
%\hline 
%\end{tabular}
\egroup
\end{table}


% =========================================================================
\subsection{Tests of Categories 1 --- 4}


Table \ref{table:tp} summarises the coverage achieved by the generated
tests of categories 1 --- 4. Each test procedure is duplicated to take into account the
different values of the flag \verb+SBAvailable+, except for the test
procedure TP-005 that checks the behaviour only when service brakes are
available. TP-001-BCS is the
test procedure realising the basic state coverage, TP-002-Transitions
the transition coverage and TP-003-MC/DC the MC/DC coverage.
Requirement-based testing is performed by the Test procedures TP-004
to TP-008. Table~\ref{table:tp-req} defines the requirements associated
with each test procedure. TP-004 focuses on the DMI interface.  TP-005
provides a test to check the behaviour when the service brake is not
available. TP-006 validates the special transition when entering the
ceiling speed monitoring with the indication status set to
one. TP-007 and TP-008 are the test procedure for testing the
revocation and the triggering of $\dmicmd$ and $\ticmd$.

\begin{table}[htbp]
 \centering
\caption{\label{table:tp-req} Requirement-based Test procedures}
\bgroup
\def\arraystretch{1.5}% 
\begin{tabular}{|l|l|}\hline\hline
Test Procedure & Requirements \\\hline
               & REQ-3.13.10.2.1 \\
 TP-004        & REQ-3.13.10.3.1\\
               & REQ-3.13.10.3.2\\\hline
               & REQ-3.13.10.2.3\\
TP-005         & REQ-3.13.10.2.4\\
               & REQ-3.13.10.3.3.r1\\\hline
TP-006         & REQ-3.13.10.3.6\\
               & REQ-3.13.10.2.5\\\hline
               & REQ-3.13.10.2.5\\
TP-007         & REQ-3.13.10.3.3.r0\\
               & REQ-3.13.10.3.3.r1\\\hline
TP-008         & REQ-3.13.10.3.3.t[1-5]\\
               & REQ-3.13.10.3.4\\


\hline \hline
\end{tabular}
\egroup
\end{table}

%\subsection{Test transition coverage experiments}

\paragraph{TP-002-OnlyCSM\_ON, TP-002-OnlyCSM\_ON-SNOSB}
  These two
test procedures generated by the tool have $\areb$ equals 0 and are of the
  form specified in Table~\ref{tab:tp002discuss} and Table~\ref{tab:tp002discussb}.
  The two test procedures are able to kill the mutant in Example~1 thanks
to the sequence {\footnotesize NORMAL $\rightarrow$ OVERSPEED
$\rightarrow$  NORMAL}. The second mutant is not killed by the first
test procedure but the sequence  {\footnotesize NORMAL $\rightarrow$ OVERSPEED
$\rightarrow$  WARNING} in the second test procedure can. Finally,
these two procedures are not able to kill the mutant of Example~3.

\begin{table}[htbp]
\caption{TP-002-OnlyCSM\_ON Test suite}
\footnotesize
\noindent\begin{tabular}{|l|cccccccc|}\hline
$(\vest,\vmax)$ & $(192,169)$  & $ (0,179)$ & $ (183,179)$ & $ (183,231)$ & $ (183,0)$ & $ (0,0)$ & $ (4.5,0)$ & $ (4.5,247)$
\\\hline
State & \tiny{S\_BRAKE} &  \tiny{NORM} &\tiny{OVSP} & \tiny{NORM} & \tiny{E\_BRAKE} & \tiny{NORM} & \tiny{WARN} & \tiny{NORM} \\\hline
\end{tabular}
\normalsize
\label{tab:tp002discuss}
\end{table}

\begin{table}[htbp]
\caption{TP-002-OnlyCSM\_ON-NOSB Test suite}
\footnotesize
\noindent\begin{tabular}{|l|cccccccc|}\hline
$(\vest,\vmax)$ & $(12.5,6.5)$  & $ (0,6.5)$ & $ (6.9,6.5)$ & $ (6.9,7)$ & $ (6.9,1.5)$ & $ (0,1.5)$ & $ (32,1.5)$ & $ (0,1.5)$
\\\hline
State & \tiny{S\_BRAKE} &  \tiny{NORM} &\tiny{OVSP} & \tiny{NORM} & \tiny{WARN} & \tiny{NORM} & \tiny{E\_BRAKE} & \tiny{NORM} \\\hline
\end{tabular}
\label{tab:tp002discussb}
\normalsize
\end{table}



The following tables summarise the experiments performed with the tests of categories 1 --- 4, in order to assess
their strength.   Mutant 1,2 and 
3 refers to the Example 1,2 and 3, as presented  section \ref{sec:test-strength}.
Table \ref{tbl:mutant} shows the results obtained after running  each
test procedure against the 3 mutants. Whenever the test passed it means
that the mutant was {\it not} detected by the test procedure. However when
the test failed, the test procedure could witness the erroneous
behaviour and therefore kill the
mutant. Table \ref{tbl:sequence} summarises the test sequences
that killed the mutants for each test procedure. None of the category 1 --- 4 tests could kill mutant~3.


\begin{table}[htbp]
\centering
\caption{\label{tbl:mutant}Mutants experiments results}
\begin{tabular}{|l|c|c|c|}\hline\hline
Test-Procedure&Mutant 1& Mutant 2& Mutant 3\\\hline
TP-001-BCS                & PASS & PASS & PASS \\\hline
TP-001-BCS-NOSB           & PASS & PASS & PASS \\\hline
TP-002-OnlyCSM\_ON        & \cellcolor{red!25}FAIL & PASS & PASS \\\hline
TP-002-OnlyCSM\_ON-NOSB   & \cellcolor{red!25}FAIL & \cellcolor{red!25}FAIL & PASS \\\hline
TP-002-Transition         & \cellcolor{red!25}FAIL & \cellcolor{red!25}FAIL & PASS \\\hline
TP-002-Transition-NOSB    & \cellcolor{red!25}FAIL
& \cellcolor{red!25}FAIL  & PASS \\\hline
TP-003-MCDC & PASS        & PASS & PASS \\\hline
TP-003-MCDC-NOSB          & PASS & PASS & PASS \\\hline
TP-004-3.13.10-DMIOutput         & \cellcolor{red!25}FAIL & PASS & PASS  \\\hline
TP-004-3.13.10-DMIOutput-NOSB    & \cellcolor{red!25}FAIL & PASS & PASS \\\hline
TP-005-3.13.10-EBvsSB & \cellcolor{red!25}FAIL & PASS & PASS \\\hline
TP-006-3.13.10-IndicationStatus & \cellcolor{red!25}FAIL & PASS &
PASS \\\hline
TP-006-3.13.10-IndicationStatus-NOSB & \cellcolor{red!25}FAIL & PASS & PASS \\\hline
TP-007-3.13.10-RevokeCmd & \cellcolor{red!25}FAIL
& \cellcolor{red!25}FAIL & PASS \\\hline
TP-007-3.13.10-RevokeCmd-NOSB & \cellcolor{red!25}FAIL & \cellcolor{red!25}FAIL & PASS \\\hline
TP-008-3.13.10-Trigger & \cellcolor{red!25}FAIL
& \cellcolor{red!25}FAIL & PASS \\\hline
TP-008-3.13.10-Trigger-NOSB & \cellcolor{red!25}FAIL & \cellcolor{red!25}FAIL & PASS \\\hline\hline
\end{tabular} 
\end{table}

\begin{table}[htbp]
\caption{\label{tbl:sequence}Test sequences that kills the mutant}
\centering
\begin{tabular}{|l|c|l|}\hline\hline
 Mutant & Test Procedure & sequences\\\hline
\multirow{5}{*}{Mutant 1} &  TP-002 & \small{NORM} $\rightarrow$
\small{OVSP}$\rightarrow$\small{NORM} \\
&  TP-004 & \small{NORM} $\rightarrow$ \small{OVSP}$\rightarrow$\small{NORM} \\
&  TP-005 & \small{NORM} $\rightarrow$ \small{OVSP}$\rightarrow$\small{NORM} \\
&  TP-006 & \small{NORM} $\rightarrow$ \small{OVSP}$\rightarrow$\small{NORM} \\
&  TP-007 & \small{NORM} $\rightarrow$ \small{OVSP}$\rightarrow$\small{NORM} \\
&  TP-008 & \small{NORM} $\rightarrow$ \small{OVSP}$\rightarrow$\small{NORM} \\\hline
 \multirow{4}{*}{Mutant 2} & TP-002 & \small{OVSP} $\rightarrow$ \small{NORM}$\rightarrow$\small{OVSP} \\
& TP-005 & \small{OVSP}$\rightarrow$ \small{NORM}$\rightarrow$ \small{WARN}\\
& TP-007 & \small{OVSP}$\rightarrow$ \small{NORM}$\rightarrow$ \small{WARN}\\
& TP-008 & \small{OVSP}$\rightarrow$\small{NORM}$\rightarrow$\small{OVSP} \\\hline
\hline
\end{tabular}
\end{table}


\subsection{Tests of Category 5 -- IECP Tests}\label{sec:iecptestresults}

In addition to the   tests of categories 1 --- 4 described above, an IECP test suite was
generated, based on a prototype implementation of the IECP test strategy described in
Section~\ref{sec:iecpstart} in RTT-MBT.   This implementation was used to generate IECP test cases
for the lower-level state machine CSM\_ON modelling 
the behaviour of the active CSM with the  input, output, and 
internal model variables  $\vest$, $\vmax$, $\dmicmd$, $\ticmd$, $\sbidisplay$. The tests were based further on the assumptions $\sbz = 1$ and $\areb = 0$

In addition to the IECP test suites introduced in Section~\ref{sec:completesuites}, a refined test suite based on 66 input equivalence classes was performed; this suite provided full requirements
coverage and adds selected boundary value test cases.   The generation of input classes, selection of the input alphabet, and the application of the W-Method results in 1630 test cases,
assuming no additional states in the implementation ($m = 4$).
The generation of the whole test suite took approx.~11 minutes. 
As expected, the IECP test suites killed the three mutants from Example~1 --- 3.
 For more details readers are referred to www.mbt-benchmarks.org. 

 

%%% Local Variables: 
%%% mode: latex
%%% TeX-master: "main"
%%% End: 
