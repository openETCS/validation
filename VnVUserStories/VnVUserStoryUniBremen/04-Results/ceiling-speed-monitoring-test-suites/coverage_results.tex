% .......................................................................
\subsection{Requirements and Model Coverage}\label{sec:coverage}

\begin{table}
\caption{Requirements Coverage and Model Coverage Overview.}\label{tbl:coverage}
\centering
% ============================================================================
% @done: Uwe: Put the new numbers/Check the numbers consistency
% ============================================================================

\footnotesize
\begin{tabular}{p{38mm}p{20mm}p{20mm}p{20mm}p{20mm}}
\hline\hline
&
{\bf Manually Defined Tests (A)} &
{\bf Extended Manual Tests (B)} &
{\bf Automatically Defined Tests (C)} &
{\bf Equivalence Class\newline Testing (D)}
\\\hline
{\bf Requirements Coverage}\newline(ETCS specification)&
24 of \numsubreq\newline(82,76\%)&
\numsubreq{} of \numsubreq\newline(100\%)&
\numsubreq{} of \numsubreq\newline(100\%)&
\numsubreq{} of \numsubreq\newline(100\%)
\\\hline
{\bf Test Cases}&
9&
11&
45&
5524
\\\hline
{\bf Model Coverage}\newline
States&
9 of 9\newline(100\%)&
9 of 9\newline(100\%)&
9 of 9\newline(100\%)&
9 of 9\newline(100\%)
\\\hline
{\bf Model Coverage}\newline
Transitions&
10 of 10\newline(100\%)&
10 of 10\newline(100\%)&
10 of 10\newline(100\%)&
10 of 10\newline(90\%)
\\\hline
{\bf Model Coverage}\newline
MCDC&
8 of 10\newline(80\%)&
8 of 10\newline(80\%)&
10 of 10\newline(100\%)&
10 of 10\newline(100\%)
\\\hline

{\bf Model Coverage}\newline
HITR&
1 of 5\newline(20\%)&
3 of 5\newline(60\%)&
5 of 5\newline(100\%)&
5 of 5\newline(100\%)
\\\hline\hline
\end{tabular}
\normalsize
\label{tab:rtt:advanced}
\end{table}%

% --------------------------------------------------------------------------
%Paragraphs describing requirements coverage:

In Table \ref{tab:rtt:advanced} we compare the requirements and the model coverage
of the test suites. Although all test suites cover all states and transitions of
the model at least once, \emph{Test Suite A} does not cover all requirements
completely (not all sub-requirements are covered).
Each of the other three test suites covers all requirements
of ETCS defined in \cite{ETCS-Subset076} including the sub-requirements from our
analysis.

Though \emph{Test Suite B} does cover all requirements, it does not cover all
test cases that are automatically generated from the test model.
Fewer combinations for guard conditions and high level transitions are executed
with this test suite.

The fully automatically generated \emph{Test Suite C} provides full requirements
coverage which is expected, as this is the goal of the test suite.
It also provides full model coverage which is an indication for testable
requirements, a model that has been developed for testing purposes and for
carefully defined satisfy relations from model elements to requirements.

The test traces that are generated for \emph{Test Suite D} have been evaluated on
the test model as well and through this, a test case coverage and requirements coverage
has been calculated using the tracing information from the test model.
\emph{Test Suite C} and \emph{Test Suite D} both achieve the same requirements
and model coverage, but \emph{Test Suite D} uses more test cases to reach this
coverage. In the following section, we will evaluate if the additional test cases
result in an increased test strength of this test suite.
