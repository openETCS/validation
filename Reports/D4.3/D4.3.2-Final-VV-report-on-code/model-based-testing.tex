

\chapter{Model-based Testing  of the ETCS Ceiling Speed Monitor}
\label{cha:csmunibremen}

\section{Introduction}
 
A new method of model-based test generation has been applied to a
component implementing the ceiling speed monitoring (CSM) in the EVC
software. To measure the strength of the generated test suite
its ability to detect mutants of the implementation has been studied,
and it has been compared to the standard set of conformity tests.


\section{Objective}
%In an ordinary development, the main objective would be to verify or validate
%something. Here, in openETCS, it would be quite common to demonstrate
%the applicability of some method/tool, or evaluate its capabilities.

The objective of this work is to demonstrate that the new input equivalence class partition testing method
developed by the team of the University of Bremen can be effectively applied for 
testing speed monitoring functions of the ETCS onboard: the new test strategy 
guarantees complete fault detection for implementations whose behaviour is captured
by a specific fault domain. For implementations outside the fault domain it is expected that the new strategy outperforms ``conventional'' -- usually heuristic -- approaches to test case identification. Compare the test cases created with the new strategy to
the system test cases for the CSM, as they have been defined in then ETCS standard,
SUBSET-0076. Identify missing test cases -- if any -- and potential for improvement for 
the test cases specified in SUBSET-076.


\section{Method/Approach}
% Short description of how the verification/validation is performed

A test model specifying the expected behaviour of the CSM has been developed 
in SysML, using state machines and block diagrams.  The model elements have been linked to the associated ETCS system requirements.  Since this SysML language subset 
can be associated with a formal semantics, it is possible to execute algorithms that automatically
\begin{enumerate}
\item identify the test cases needed to achieve requirements coverage,
\item identify the input equivalence classes and associated test cases for achieving 
full fault coverage in relation to a given failure model,
\item calculate concrete test data using a mathematical constraint solver (SMT-Solver),
\item generate the test procedures executing the test cases with the concrete test data,
\item generate the test oracles (integrated in the test procedures) for comparing 
the actual system under test (SUT) behaviour against the expected behaviour represented in the test model, and
\item  compile the traceability information documenting which requirements have been covered by which test cases, and which test results have been achieved
\end{enumerate}
from the model.

The existing SUBSET-076 test cases were formalised using linear temporal logic (LTL), 
so that the same test data generation concept could be applied as for the test cases that were automatically identified: SUBSET-076 test cases do not provide concrete test 
data for every test step, but specify the general constraints from which concrete
data can be elaborated.  This approach also allows to trace the model coverage achieved by the SUBSET-076 test cases.

All tests were executed against software mutants derived from a reference implementation, using 3 different mutation generators in order to avoid a mutation bias. For each 
testing strategy applied it was checked 
\begin{itemize}
\item which parts of the test model were covered by the test execution, and
\item which fault coverage (percentage of ``killed'' mutants) was achieved.
\end{itemize}





\section{Means/Tools}
% Could be integrated with the previous paragraph. Assign an
% approrpiate tool class (T1, T2, T3) according to EN 50128. Try to assign a
% maturity level 
%https://github.com/openETCS/validation/blob/master/VerificationAndValidationPlan/V02/VnVUsrStrTmplt-140709-02.pdf

The whole approach is fully supported by RT-Tester and its model-based testing component RTT-MBT. RTT-MBT has an integrated SMT-solver and can generate solutions for LTL formulas provided by the users (this is needed for evaluating the SUBSET-076 test cases). From 
SysML test models, the tool automatically performs the steps 1 -- 6 listed above:
test cases identified by the tool are internally encoded as LTL formulas which are solved by the SMT-solver to generate concrete test data; these steps do not 
require any user interaction.

\section{Results}
% Results related to the objective.
% Refer to appropriate document (preferably GitHub) for more complete description.
The results can be summarised as follows.
\begin{enumerate}
\item The new equivalence class testing method shows significantly higher test strength than all other methods used in the comparison. It achieved nearly 100\% fault coverage for mutants outside the fault domain (mutants inside the fault domain are always killed, due to the guaranteed  fault detection properties).  

\item The new method is very well suited for software testing and HW/SW integration testing, where the high number of test cases (approx.~5000 cases) can easily be executed, in particular, because the test suite is fully automated. The new method, however,
 yields too many 
test cases to be applied on system testing level with real  trains on real tracks.


\item The SUBSET-076 test cases are missing 2 cases for the CSM in order to achieve
requirements coverage. These can be easily identified and added. As a result,
these test comprise 11 cases.

\item With the missing test cases added, the SUBSET-076 achieve only a fault coverage of
62\% -- this would certainly not suffice to obtain certification credit. It is
possible, however, to add an acceptable number of test cases to the SUBSET-076
suite for the CSM which would significantly increase its test strength.


\end{enumerate}


All   results have been published in 
\begin{itemize}
\item  Jan Peleska and Wen-ling Huang: Complete model-based equivalence class testing. Int J Softw Tools Technol Transfer. Published online: 21 November 2014. DOI 10.1007/s10009-014-0356-8.

\item Felix Hübner, Wen-ling Huang, and Jan Peleska: Experimental Evaluation of a Novel Equivalence Class Partition Testing Strategy. In Jasmin Christian Blanchette and Nikolai Kosmatov (eds.): Tests and Proofs - 9th International Conference, TAP 2015, Held as Part of STAF 2015, L'Aquila, Italy, July 22-24, 2015. Proceedings. Lecture Notes in Computer Science 9154, Springer, 2015, pp. 155-172, doi 10.1007/978-3-319-21215-9\_10.

\item C{\'e}cile Braunstein, Anne E. Haxthausen, Wen-ling Huang, Felix Hübner, Jan Peleska, Uwe Schulze, and Linh Vu Hong: Complete Model-Based Equivalence Class Testing for the ETCS Ceiling Speed Monitor. In S. Merz and J. Pang (eds.): Proceedings of the ICFEM 2014. Springer, LNCS 8829, pp. 380-395, 2014. DOI 10.1007/978-3-319-11737-9\_25.


\item Technical Report http://www.informatik.uni-bremen.de/agbs/testingbenchmarks/
\newline
openETCS/ceiling-speed-monitoring/testing\_the\_etcs\_csm.pdf





\item C{\'e}cile Braunstein,   Wen-ling Huang, Felix Hübner, Jan Peleska, and Uwe Schulze: Evaluation of Model-Based Testing Strategies for the ETCS Ceiling Speed Monitor.
Submitted to Software Testing, Verification and Reliability journal.

\end{itemize}






\section{Observations/Comments}
% An optional section where anything can be included which has been
% observed without direct connection

It is interesting to note that typical model-coverage driven test cases (e.g. transition coverage, MC/DC coverage), while achieving higher model coverage than the
SUBSET-076 tests, do not achieve much higher fault coverage (approx.~68\%). 
The reason is that   these test cases are not invariant under syntactic model transformations: with another -- through semantically equivalent -- model, higher or lower test strength would be achieved with the coverage-driven test cases derived
from that model. 

In contrast to that, the new equivalence class testing strategy is elaborated from the
{\it semantic} representation of the model and is therefore invariant (i.e.~always maximal) under all syntactic model transformations that leave the behavioural semantics unchanged.

Verified Systems International GmbH who maintain the commercial
version of RT-Tester have won the runner-up trophy of the EU
Innovation Radar Innovation Prize\footnote{see {\tt
    https://www.verified.de/publications/papers-2015/\newline
    eu-innovation-radar-price-runner-up-trophy-for-verified-systems-international/}}
for implementing the equivalence class testing strategy described
above in the commercial version of RT-Tester.

\section{Conclusion}
%What has been achieved, what is missing, what has been learned

The new test strategy has shown to provide superior test strength when compared to
SUBSET-076 test cases and conventional model-coverage driven test cases that are 
typically provided by other model-based testing tools. As of today, RT-Tester is the
only testing tool where the new test strategy is implemented.


\chapter{Model-based Testing of the ETCS Target Speed Monitor}
\label{cha:targetspeedmonitorbremen}

\section{Introduction}
% Usually, one main partner, perhaps with contributions from others
A refinement of the test generation method of
Sec.~\ref{cha:csmunibremen} is used to produce test cases with
physically meaningful data. The discriminatory strength of the test
cases is not affected by the refinement. The method is demonstrated on
the target speed monitor functionality of the EVC SW.

\section{Object of verification}
% Which openETCS artifact (github link) or other documents/programs
% etc. (provide references)

The object of verification is the target speed monitoring function of the EVC,
see technical report
\begin{itemize}
\item[[~1]] Felix H{\"u}bner, Christoph Hilken, and Jan Peleska.
  Combination of Behavioral and Parametric Diagrams for Model-based
  Testing -- Application to ETCS Target Speed Monitoring. Submitted
  to DAC 2016, also available as technical openECTS report 2014-11-25.

Available under  \url{https://github.com/openETCS/validation/tree/master/VnVUserStories/VnVUserStoryUniBremen/04-Results}.
\end{itemize}

Parts of a functional model of the target speed monitor used for test
generation are available in [1]. The whole target speed monitoring
model will be made available on {\tt http://www.mbt-benchmarks.org}.

\section{Objective}
%In an ordinary development, the main objective would be to verify or validate
%something. Here, in openETCS, it would be quite common to demonstrate
%the applicability of some method/tool, or evaluate its capabilities.
For creating a SysML test model of the target speed monitoring
function, both time-discrete (e.g.~trigger of the emergency brakes)
and time-continuous (e.g.~time-dependent train location, speed, and
acceleration) variables need to be considered. SysML state machines
are suitable for modelling concurrent real-time behavior of
time-discrete control functions. For time-continuous aspects, the
report [1] describes how to use parametric constraints and associated
diagrams for modelling. It is also explained how the parametric
specifications are made available to the SMT solver creating concrete
test data from models. As a result, the solver generates data that
complies with the time-continuous physical constraints of the model.


\section{Method/Approach}
% Short description of how the verification/validation is performed

Parametric constraints represent a language aspect of the SysML which
has not yet been fully investigated in the research communities. Using
so-called constraint blocks, these constraints can be
specified. Typically, parametric constraints represent system
invariants or -- this is the relevant aspect for the target speed
monitor -- physical laws, such as acceleration-dependent speed and
speed-dependent location.  For our application, these laws also
comprise the ETCS braking curves modelling the speed changes of the
braking train.  Parametric constraints can be specified using general
physical variables; these are bound to concrete model variables using
parametric diagrams.

It is shown in [1] how parametric constraints can be used to calculate
physically meaningful train behaviours, that is, meaningful changes of
speed and location over time, taking into account the braking
actions. The method follows a 2-step approach: first, a model
abstraction is created, and the equivalence class testing strategy
described in Section~\ref{cha:csmunibremen} is used to identify test
cases with guaranteed fault detection properties. Next, the calculated
tests are refined with respect to time-dependent behaviour, so that
still the same equivalence classes are used, but the representatives
for location and speed are selected in a way that complies with the
physical laws.

 


\section{Means/Tools}
% Could be integrated with the previous paragraph. Assign an
% approrpiate tool class (T1, T2, T3) according to EN 50128. Try to assign a
% maturity level 
%https://github.com/openETCS/validation/blob/master/VerificationAndValidationPlan/V02/VnVUsrStrTmplt-140709-02.pdf

The method has been implemented in the RT-Tester tool as part of the WP7-related
activities of the University of Bremen team.

\section{Results}
% Results related to the objective.
% Refer to appropriate document (preferably GitHub) for more complete description.

The results show that the method can be automatically performed with acceptable 
computation time. 

\section{Observations/Comments}
% An optional section where anything can be included which has been
% observed without direct connection

To our best knowledge, this is the first SysML-based method for
calculating test data with guaranteed fault detection properties in
presence of both time-discrete and time-continuous observables.


\section{Conclusion}
%What has been achieved, what is missing, what has been learned

The method developed here is highly relevant for testing
cyber-physical systems in general. Verified Systems International GmbH
who maintain the commercial version of RT-Tester has already decided
to make this method available in 2016.

