

\subsubsection{Verification of Procedure On-Sight (Systerel) }
\label{sec:}

\paragraph{Contributing project partners}
% Usually, one main partner, perhaps with contributions from others
The work has been performed by Systerel

\paragraph{Process step}
% Classification of the activity according to D2.3a
% Name what is verified and to which report this would contribute.
% Use the numbers, e.g. System Design Verification Report (1-12).


This activity contributes to:
\begin{itemize}
\item the Software Requirement Specification  (3.16)
\item the Software Specification Verification Report  (3.18)  
\item the Software Architecture and Design Specification (4.19)
\item the Software Design Verification Report  (4.23)  
\item the Software Components (5.24)
\item the Software Component Verification Report  (5.26)  
\end{itemize}



\paragraph{Object of verification}
% Which openETCS artifact (github link) or other documents/programs
% etc. (provide references)

The object of verification is the Classical B model for the procedure On-Sight at {\url{https://github.com/openETCS/model-evaluation/tree/master/model/Classical_B_Systerel/obu_classicalB}}. 


\paragraph{Available specification}
% The specification against which the object is to be checked. Usually
% coming form some openETCS artifacts (GitHub reference, process
% artifact number) or background material (reference, artifact number).

The model implements the requirements for the The Procedure On-Sight, as described in {\itshape System Requirements Specification, Chapter 5}.



\paragraph{Objective}
%In an ordinary development, the main objective would be to verify or validate
%something. Here, in openETCS, it would be quite common to demonstrate
%the applicability of some method/tool, or evaluate its capabilities.


The goal is to produce a B model which implement the On-Sight procedure.
The B model shall be formally proved (verified) and shall allow to generate an executable C code.



\paragraph{Method/Approach}
% Short description of how the verification/validation is performed

At first, a formal model is defined with the B method using the AtelierB  tool.
Then the model is formally verified by proof and model-checking.
Functional properties can also be defined and validate by proof or model-checking on the B model.

Finally, C code is automatically translated from the B model.

\paragraph{Means/Tools}
% Could be integrated with the previous paragraph. Assign an
% approrpiate tool class (T1, T2, T3) according to EN 50128. Try to assign a
% maturity level 
%https://github.com/openETCS/validation/blob/master/VerificationAndValidationPlan/V02/VnVUsrStrTmplt-140709-02.pdf


The means used are:
\begin{itemize}
\item Atelier B to  design, check, verify and prove the model (qualified by industrial railway actors)
\item ProB to perform model-checking
\item Atelierb translators to produce C code (Code translator shall be T3 level to obtain certified code).
\end{itemize}


\paragraph{Results}
% Results related to the objective.
% Refer to appropriate document (preferably GitHub) for more complete description.


\begin{itemize}
\item The result is a fully formal model of the procedure On-Sight.
\item The C code has been automatically  translated.
\item The model  is formally proved.
\item Some properties have been checked by model checking.
\end{itemize}

See {\url{https://github.com/openETCS/validation/blob/master/Reports/D4.3/D4.3.1-Final-VV-report-on-model/D4.3.1.pdf}} and {\url{https://github.com/openETCS/validation/blob/master/VnVUserStories/VnVUserStorySysterel/04-Results/b-ClassicalB-VnV/BmodelVnV.pdf}}.



\paragraph{Observations/Comments}
% An optional section where anything can be included which has been
% observed without direct connection


\paragraph{Conclusion}
%What has been achieved, what is missing, what has been learned

The B~method, along with its verification processes and tools, meets the goals and activities of the openETCS project in terms of quality, rigor, safety and credibility.\\
There is yet to develop open-source POG and build a framework for proving, but this is compensated by the fact that work on the subject is ongoing, and ProB is an effective tool for verification.
