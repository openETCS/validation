\section{Conclusions}
\label{sec:conc}
% ====================================================================



% ====================================================================
\paragraph{Discussion of results}
In this article, the ceiling speed monitoring function of the European Train
Control System has been used as a case study for evaluating various
model-based testing strategies and comparing their  strength with 
the associated tests contained in the ETCS standard, the latter having
been developed by
domain experts in a manual way. 
A test model of the CSM has been presented. It is expressed in SysML and interpreted
in a formal behavioural semantics based on Kripke structures.
The MBT strategies comprised ``standard'' 
model coverage techniques
(basic control state, transition, MC/DC coverage, and hierarchic
transition coverage applicable to composite state machines), as well as a complete
input equivalence class testing strategy. The comparison of the strategies involved focused on requirements coverage and test strength. 
It turned out that the manually 
developed test suite contained in the ETCS standard has some deficiencies regarding requirements coverage; as a result, their test strength is considerably lower than 
the strength of any of the MBT strategies. 

The test strength of the different strategies has been evaluated using a CSM reference
implementation programmed in Java, from which 368 mutations have been generated.
The complete input equivalence class testing strategy 
turned out to have significantly higher test strength 
(mutation score of nearly 100\%), when compared to any of the 
other strategies, and the test suite contained in the ETCS standard was significantly
weaker than the other strategies. Suggestions were presented how the latter suite 
should be extended, resulting in moderate additional effort only. 

The nearly complete mutation score achieved by the equivalence class method
confirms the hypothesis that complete testing strategies are also of great
practical value when applied against implementations that are outside the 
specified fault domain where {\it every} SUT error is detected by at least one test case. On the other hand, the complete strategy increased the test suite size
by a factor of 100 and more in comparison to the suites resulting from the other
strategies. This indicates that the complete strategies may be less suitable for
applications outside the safety-critical systems domain, where lesser test strength
may still be acceptable. It has been analysed why the non-complete MBT strategies 
still resulted in a mutation score of 85\%, and it was pointed out that this
is due to the simple decision structure in the CSM, where guard conditions are
relatively wide and most control modes can be
reached from any other mode in one step. 



% ====================================================================
\paragraph{Ongoing work}
The nature of the input equivalence classes used in the CSM case study is quite ``conventional'' in the sense that they are made up from intervals of real numbers. The underlying equivalence class testing theory, however, does not depend on this simple
class structure: since the equivalence concept is based on equivalent behaviours derived from the behavioural model's transition relation, the theory covers classes
of discrete points in the state space as well. Exploiting this fact, the construction of
equivalence classes for train configurations in railway networks 
is currently  investigated in collaboration with other authors~\cite{ftscs}, 
for the purpose of testing interlocking control systems with acceptable effort. 

In this context, an extension of the equivalence class testing theory to nondeterministic models is currently being prepared for publication. This extension makes use of an
abstraction of nondeterministic IOSTS with infinite input domains, but 
finite internal state and output domains to nondeterministic finite state machines 
(NFSMs). This
abstraction enables us to transfer   complete test suites for NFSM
as presented in~\cite{petrenko_testing_2011,petrenko_testing_2014,hierons_testing_2004}
to complete test suites for IOSTS. The motivation of this theory extension to nondeterministic models is as follows: when performing MBT in a safety-critical context,
the correctness and completeness of the test model is of utmost importance. Therefore
formal verification techniques, in particular variants of model checking, are applied
to verify the test model before using it for automated test suite generation.
Typical verification methods that can be applied successfully for this purpose
use over-approximation of the model in order to facilitate the verification process. 
As a consequence, the test model to be used as input for the automated testing campaign
becomes nondeterministic. An example for a model verification in the interlocking 
control systems domain by means of bounded model checking in combination with $k$-induction is presented
in~\cite{ftscs}. 





% ====================================================================
\paragraph{Acknowledgements}
C{\'e}cile Braunstein's work is funded by ITEA2 project openETCS under grant agreement~11025. 
Felix H{\"u}bner's research contribution is funded by 
Siemens~AG in the context of the SyDE Graduate School on System Design.\footnote{http://www.informatik.uni-bremen.de/syde/index.php?home-en}
The work of Wen-ling Huang and Jan Peleska is funded by the project \emph{ITTCPS -- Implementable Testing Theory for Cyber-physical Systems}\footnote{http://www.informatik.uni-bremen.de/agbs/projects/ittcps/index.html} which has been granted in the context of the German Universities Excellence Initiative.\footnote{http://en.wikipedia.org/wiki/German\_Universities\_Excellence\_Initiative}