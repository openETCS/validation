\documentclass{article}

\title{Proposal for an Entry to a Verification Report\\Version 1.0}
\author{Hardi Hungar}
\date{Nov 01, 2013}

\newcommand{\tbi}[1]{$<$\textit{#1}$>$}

% Starts a new line nearly everywhere
\newcommand{\nl}{\mbox{}\\}
\newcommand{\nlskip}[1]{\mbox{}\\[#1]}

%
%Comments
\newcommand{\cmmnt}[1]{\framebox{#1}}
\newcommand{\bgcmmnt}[1]{\nl\framebox{\parbox{.95\textwidth}{#1}}\nl[2mm]}
%\renewcommand{\bgcmmnt}[1]{}
%

\newcommand{\eod}{\nl\rule{.95\textwidth}{1pt}\nl\textit{End of Document}}

\begin{document}
\maketitle

\begin{abstract}
This document contains a proposal for structure and content of an
entry to a verification or validation report. The information items
appearing in this proposal should be supplied, though not necessarily
in always the same form. For instance, if there is already a good
description of the method applied, the report should contain a
reference to that description and only a short summary of it. If there
is none, more text is called for.  

The proposal should be used as a guideline to check whether all
information is given appropriately. The wording used in this proposal
is by no means mandatory. And if you feel that more information is
useful to describe your activity within the context of openETCS, you
should of course do so. Feel free to add additional categories of
description as adequate. 

Also the \LaTeX{} macros may be changed, though the use of
\texttt{paragraph} and \texttt{subparagraph} enables easy integration
into higher-level documents (they are not numbered automatically,
which may be a draback in other respects). 
\end{abstract}

\subsubsection{Verification of \tbi{verification object}}

This sections reports on the verification activities of
\tbi{verification object}. The goal of the activity is to establish \tbi{goal}. 

\bgcmmnt{There are two categories of goals: One is to do something for
  the EVC software to be developed in openETCS. That is, a design
  artifact is verified here. The other is to do something for the VnV
  methods, eg.\ evaluating or demonstrating the suitability of tools.} 

The acitivity is described in the Verification and Validation Plan
\tbi{reference}. In short, \tbi{short summary}.   
\bgcmmnt{If there is no adequate plan, add something here).} 

\paragraph{Object of verification}
\nl
The object of verification is \tbi{name, github ref}. It is from
\tbi{design phase} and represents/describes/implements \tbi{the
  what}. 

\bgcmmnt{Design steps according to the openETCS process described in
  D2.3. Add more detailed characterisation if suitable.} 

\paragraph{Available specification}

\bgcmmnt{Specification to be implemented by the verification object.}

\tbi{References, short description}


\paragraph{Detailed verification plan}

\subparagraph{Goals}

\bgcmmnt{Which specification items are to be established?}

\subparagraph{Method/Approach}

\subparagraph{Means}

\bgcmmnt{Mainly tools}

\subparagraph{Other}

 \bgcmmnt{optional, might be renamed}

\paragraph{Results}
\subparagraph{Summary}
 \bgcmmnt{Goals [not] achieved}
 
 \subparagraph{Evidence produced}

 \bgcmmnt{Description and reference to details (if applicable)}
 
 \subparagraph{Conclusions/Lessons learned}
 
 \bgcmmnt{Contribution of the activity to the goals of openETCS}

\paragraph{Future Activities}

\bgcmmnt{Sketch of revised or new plans concerning the same or similar
  objectives (eg.\ tool application, new verification campaign on the
  same object, tool improvements)} 

\eod

\end{document}