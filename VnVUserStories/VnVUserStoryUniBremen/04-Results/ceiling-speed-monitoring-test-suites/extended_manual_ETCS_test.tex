The test specification provided by the European Railway agency (\cite{ETCS-Subset076})
contains test suites for the different ETCS features. The sub-requirements
REQ-3.13.10.3.4.r1c2, REQ-3.13.10.3.4.r3c2, REQ-3.13.10.3.4.r4c2 and REQ-3.13.10.3.4.r5c2
describe a transition from supervision status \emph{Indication} to \emph{Normal}, \emph{Overspeed}, \emph{Warning} and \emph{Intervention}.
Supervision status \emph{Indication} is not used by CSM, but requirement
REQ-3.13.10.3.6 states, that if CSM is ativated with supervision status
being \emph{Indication}, the state changes shall occur as defined in the
sub-requirements above.
The manual test case for REQ-3.13.10.3.6 defined in \cite{ETCS-Subset076}
only covers the state transition from \emph{Indication} to \emph{Normal}.
Therefore the test manual test suite does not cover the sub-requirements
REQ-3.13.10.3.4.r3c2, REQ-3.13.10.3.4.r4c2 and REQ-3.13.10.3.4.r5c2.

Our analysis shows that the sub-requirements REQ-3.13.10.3.4.r5c3 and
REQ-3.13.10.3.4.r5c4 are not covered by the tests defined in \cite{ETCS-Subset076}.
These two sub-requirements define the behaviour of CSM with
supervision status being \emph{Overspeed} and the trigger conditions
require to directly change to \emph{Intervention} (REQ-3.13.10.3.4.r5c3) and
supervision status being \emph{Warning} and the trigger conditions
require to change to \emph{Intervention} (REQ-3.13.10.3.4.r5c4).

To be able to compare the test manually defined suite with automatically
generated test suites presented in \ref{sec:automated_ETCS_test} and the equivalence class testing
approach presented in \ref{sec:ecpt}, additional test procedure have manually been added
to cover the missing requirements\footnote{See table \ref{tbl:extendedmanualtest}}.

\begin{table}
\tabsize
\renewcommand{\arraystretch}{1.2}
\caption{Additional test cases to cover the remaining sub-requirements.}
\label{tbl:extendedmanualtest}
\begin{tabular}{lp{.37\textwidth}p{6.4cm}}
\hline\hline
Identifier & Target of the test & Covered Requirements \\
\hline
TC-CSM-8 & 
When entering CSM mode the Indication status is overwritten &
REQ-3.13.10.3.4.r1c2, REQ-3.13.10.3.4.r3c2,\newline
REQ-3.13.10.3.4.r4c2, REQ-3.13.10.3.4.r5c2,\newline
REQ-3.13.10.3.6 \\ \hline
TC-CSM-9 & 
Switching from \emph{Overspeed} to \emph{Intervention} and \emph{Warning} to \emph{Intervention}&
REQ-3.13.10.3.4.r5c3, REQ-3.13.10.3.4.r5c4 \\
\hline\hline
\end{tabular}
\normalsize
\end{table}
