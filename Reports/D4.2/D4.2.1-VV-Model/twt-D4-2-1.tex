\section{TWT GmbH Science \& Innovation}

One of our activities regarding model verification is the modelling of the behavioral part of the ETCS (i.e., the procedures described in Subset 026, Chapter 5). The models are then used to validate the specification, to support the modeling using Scade and the verification of Scade models on a higher\footnote{In comparison to Scade models} level of abstraction.

\subsection{Achievements}

So far, we have modeled parts of Subset 026-5. During this process and with the resulting model, we have conducted a first validation of the specification.

\subsubsection{The Model}

Until now, we have modeled the following five procedures of Subset 026-5:
\begin{itemize}
	\item Start of Mission (Subset 026-5.4)
	\item End of Mission (Subset 026-5.5)
	\item Shunting Initiated by Driver (Subset 026-5.6)
	\item Override (Subset 026-5.8)
	\item Train Trip (Subset 026-5.11)
\end{itemize}

As a formal model, we use \textit{colored Petri nets} (CPNs)~\cite{CPN-book}, an extension of classical Petri nets~\cite{PNbook} with data, time, and hierarchy. CPNs are well-established and have been proven successful in numerous industrial projects. They have a formal semantics and with CPN Tools~\cite{Westergaard2013apn}, there exists tool support for modeling CPNs. Moreover, CPN Tools also comes with a simulation tool and a model checker, thereby enabling formal analysis of CPN models. 

We focus on modeling the \textit{system behavior}, that is, the control flow of the on-board unit and the interplay with its environment (e.g., the driver and the RBC). Figure~\ref{fig:Top} depicts the CPN representing the highest level of abstraction. It shows the decomposition of the overall system into the on-board unit and its environment: the driver, the RBC, the RIU, the STM, and the GSM module. Each component is modeled as a subpage (i.e., a component). Graphically, a subpage is depicted as a rectangle with a double-lined frame. Furthermore, the model shows through which message channels and shared variables the on-board unit is connected to its environment. A channel or shared variable is modeled by a place which is graphically represented as an elipse. As an example, the driver (i.e., subpage \texttt{Driver}) may send a message to the on-board unit (i.e., subpage \texttt{On-board Unit}) via the place \texttt{msg from driver}, and receives messages sent by the on-board unit via the place \texttt{msg to driver}.

\begin{figure}[tb]
	\centering
		\includegraphics[width=0.9\textwidth]{figures/Top.pdf}
	\caption{Top level model}
	\label{fig:Top}
\end{figure}

Zooming in subpage \texttt{On-board Unit} yields the CPN model in Fig.~\ref{fig:OBUnit}. This CPN model has two subpages: Subpage \texttt{Start} models the states S0 and S1 of the specification (i.e., Subset 026-5.4) and subpage \texttt{Rest} the remaining states. At this level of abstraction, we see on the left hand side seven places (green frame). Each such place models (a part) of the state of the on-board unit, for example, the mode and the train running number. The current model has 689 places, 173 transitions and 1,227 arcs.

\begin{figure}[tb]
	\centering
		\includegraphics[width=0.9\textwidth]{figures/OBUnit.pdf}
	\caption{CPN model of the on-board unit}
	\label{fig:OBUnit}
\end{figure}

Having a more detailed look at Fig.~\ref{fig:OBUnit}, we observe that our model does not represent all variables of the on-board unit as given in the specification and also partially abstracts from data. We abstract from those details, because the model is tailored to formalize the \textit{control flow} of the on-board unit and, in particular, the \textit{communication behavior} with its environment. As a benefit, this abstraction reduces the complexity of the model and improves its understandability. Additional details, such as data and precise message values, can be added in a refinement step.


\subsubsection{Validation of the Specification}

The modeled procedures have been manually modeled using CPN Tools. Thereby, each element in the model has been reviewed against the respective requirement, as given in the specification. To improve the confidence in the model, in a second step, a person other than the modeler checked the model against the specification. In addition, we used the simulator to check whether the modeled behavior of the CPN matched the intended behavior.

So far, the primary goal of modeling has been to validate the specification. During the modeling we discovered 36 inconsistencies, ambiguities and gaps in the specification which we reported in~\cite{specfindings}. 

\subsubsection{Other application of the model}

The CPN model can also serve as a \textit{reference model}, for example, to compare and check other models and to generate test cases.


\subsection{Next Steps}

We shall continue our work by completing the model, contributing to the modeling of (parts of) Subset 026-5 using Scade, and verifying the Scade model. In addition, we are planning to exploit synergies by collaborating with the project partner LAAS who advocate the Petri net model checker Tina~\cite{BerthomieuV2006}.


\subsubsection{Modeling the Subset 026-5}

We plan to model the remaining parts of Subset 026-5, thereby reporting possible additional findings in the specification. The goal is to have a CPN modeling all procedures that are described in Subset 026-5. We also want to compare our model with the (corresponding part of the) ERTMS model~\cite{ertms}.


\subsubsection{Scade Modeling}

As the ETCS will be modeled using Scade, we shall contribute to this modeling process. To use the experience that we gained from modeling Subset 026-05 with CPN Tools, we want to contribute to the Scade modeling of (parts of) the Subset 026-5. The Scade design flow starts with modeling all components and their interplay using SysML block diagrams (with the tool Scade Designer). The resulting SysML diagrams provide a functional and an architectural view. They are similar to the CPN model in Fig.~\ref{fig:Top}. In a second step, the behavior of each block has to be fully modeled on the system level using Scade Suite. As Scade currently does not support state machine models on the level of SysML, the Scade design flow misses an abstraction level. Our CPN model closes this gap and will, therefore, be useful for the Scade modeling.


\subsubsection{Verification of the Scade Model}

Another task concerns the verification of the resulting Scade model. Recently, researchers reported on complexity problems already for medium-sized Scade models that restrict the verification using the Scade prover~\cite{HuhnM2014scp,DaskayaHM2011fmics}. Given the complexity of the ETCS, we assume that we will face similar challenges. To alleviate those complexity problems, we aim to apply the following techniques:
 
\paragraph{Abstraction}  We will apply abstraction techniques on the Scade model to prove safety properties on a higher level of abstraction whenever possible. On the one hand, we can apply Scade contracts to restrict the domain of the input values. This technique is known as environment abstraction. Or, we can transform the Scade model into a model of higher abstraction, thereby using different formalisms such as timed automata, transition systems and Petri nets. We can then use verification tools that are dedicated to the properties of interest and the chosen formalism. We see the chance that our CPN model can be used for this task, too. For example, Uppaal~\cite{BehrmannDLHPYH2006} can analyze timed automata, Spin~\cite{Holzmann97} and NuSMV~\cite{CimattiCGGPRST2002} can analyze transition systems, and Tina~\cite{BerthomieuV2006}, LoLA~\cite{Wolf2007}, and CPN Tools~\cite{Westergaard2013apn} are tools for analyzing (different variants of) Petri nets.

\paragraph{Compositional Reasoning} Another approach is to prove properties for individual components and deduce from it the correctness of a property concerning the entire ETCS. Here we think that we can, in particular, combine our model with the MoRC model~\cite{braunstein_MorC_2013} and apply compositional reasoning. 

\paragraph{Correctness by Design} The two previous approaches support \textit{correctness by verification}; that is, first the model is designed and in the next step it is verified. A different methodology is \textit{correctness by design}. The idea is to model on a higher level of abstraction and to prove that certain safety properties hold. Then the model is iteratively refined. Each refinement step has to guarantee that all properties that hold for the more abstract level also hold in the refined model. The challenge is to find property-preserving refinement rules or a refinement relation between an abstract model and a refined model that preserves the desired properties and to verify that this relation holds. 

