This section describes the verification plan of TWT. It concerns
the validation of the ETCS specification. The goals of the activity are:

\begin{itemize}
\item establishing the consistence and correctness of the specification
\item supporting the modelling activities in WP3 by identifying deficiencies
\item providing suggestions for improvements of the standard
\end{itemize}

\paragraph{Object of verification}
\nl
The object of verification is the ETCS specification, in particular the procedures in Subset 026, Chapter 5.

\paragraph{Available specification}

see above

\paragraph{Methods and Means}

Colored Petri nets are used to describe the procedures of the standard on an abstract level. 
Details of data are mainly abstracted but the focus is on the control- and message-flow. This enables to identify potential weak spots
in the standard, e.g., scenarios where messages are lost or incorrectly received.

In this context we will use different verification methods
\begin{itemize}
\item simulation (with CPN Tools)
\item model checking (CPN Tools, ASCoVeCO)
\end{itemize}

It is planned to use the results also to support the model development in WP3.

\paragraph{Results to be achieved}

\begin{description}
  \item[Specification Findings] We have already collected specification findings that resulted from the modelling of the procedures in Subset 026, Chapter 5. This list is going to be further extended as we expect to find further deficiencies which are not directly evident and require a thorough analysis.
  \item[Improvement Suggestions] Suggestions on how to improve the specification; these will also help the modelling activities in WP3
  \item[Feedback for Standardisation] We plan to provide concrete feedback to improve the standard
\end{description}

\paragraph{Timeline \& step description}
~\\
\begin{tabular}{ll}
  \textbf{until}&\textbf{activity}\\
  November 2014 & Completion of the model\\
  December 2014 & Evaluation of model checking capabilities with CPNs\\& using different verification methods\\
  December 2014 & Partitioning of the model complete\\&(currently covering several procedures)\\
  June 2015 & All analysis results available\\
  August 2015 & Review of findings and improvement suggestions complete\\
  October 2015 & Final report as feedback to standardisation complete\\
\end{tabular}


\paragraph{Maturity Classification}

The tools applied have the following TRLs (Technology Readiness
Levels):
\begin{description}
\item[CPN Tools] TRL 7 (in use in several areas, available for years, not qualified)
\item[Others:] other tools to be applied not yet fixed
\end{description}

The activity shall comply in the following way to the requirements of
a SIL~4 development:

SIL~4 compliance is not necessary as the results of the activities are not automatically processed further but instead involve a review by a human.
