%-----------------------------------------------------------------------
\section*{Introduction}
%-----------------------------------------------------------------------

\subsection*{Objectives}

\emph{Verification} is the activity to ascertain that a particular
step in the development has achieved its goals, i.e., that its result
correctly refines or implements its input, which may be a higher-level
design or a specification. \emph{Validation} is about making sure that
the end result of the development meets its initial specification,
that is, the requirements of the user. The term validation is also
used when a design artifact is checked against requirements from
previous steps. Verification and/or validation is required for most
development artifacts.  What exactly has to be checked depends on the
set of items produced in the development process, their role and their
nature.  As the EVC software contributes to several safety-critical
functions of the ETCS onboard unit, the specific requirements
concerning the safety aspects of the standards EN~50128 and EN~50129
have to be respected throughout.

A main obligation of this work package is the verification or
validation of development artifacts produced by WP~3. This work will
concentrate on the functional and safety aspect. Besides \vv,
the work package shall also establish a coherent and comprehensive
chain of methods and tools for V\&V in cooperation with WP~7. Specific
challenges in this respect arise from the wish to use models
extensively in the development process, which means that more common
approaches have to be improved or substituted to fit a
model-based development style, and from the requirement of
using open-source tools, or even trying to realize the EVC software as
an \emph{open proof} item.

In pursuing these goals, the work package generates feedback
concerning the adequacy, correctness and safety of development
artifacts for WP~3, and the usefulness of tools and methods for WP~7.  

\subsection*{Organisation of the Work package}

The work packages consists of five tasks. The first task defines the
\vv strategy and formulates the initial \vv plan. This plan defines the
\vv activities to be done in openETCS and proposes the means to
perform them. In later stages of the project the plan will be extended
and revised to reflect the findings made while applying methods and
tools to the artifacts at hand.

Both model and code of the EVC software are subjected to \vv as they
are produced by WP~3, and the tools and methods proposed in the \vv
plan as well as newly developed or improved tools from WP~7 are applied
and evaluated in the process. Findings from these steps are
iteratively fed back to the respective work package and used to refine
the \vv plan.

A dedicated activity studies the safety aspect of \vv. It takes into
cosideration what the standards (mainly EN~50128 and EN~50129) mandate
and defines how these requirements can be met by the combination of
life cycle, methods and tools.

An internal assessment will simulate a real Assessor's task doing a 
Software Development assessment of the project impacting Working Packages 
1, 2, 3, 4, 5, 6 and 7.

The phases defining the \vv process is divided into the design phase and 
the application phase. The \emph{design phase} covers the time before the 
release of the artifacts which are to be evaluated. In this phase the 
findings of the last application phase are taken into account to improve 
\vv.
The \emph{application phase} covers the time after the release of the 
artifacts to be evaluated until the \vv report is written. For this phase 
the artifacts to be evaluated are frozen to a fixed release date.  

%-----------------------------------------------------------------------
\subsection*{Techniques for \VV}
%-----------------------------------------------------------------------

\VV techniques can be roughly classified into \emph{dynamic} and
\emph{static} techniques.  The most common dynamic \vv techniques are
various forms of \emph{testing}, which execute the code or the
model. They are classfied by their object or their purpose. These
include:
\begin{itemize}
\item Unit testing
\item Integration testing
\item Acceptance test
\item Software-in-the-Loop
\item Model-in-the-Loop
\item Model-based testing
\item Monitoring
\item Coverage analysis
\end{itemize}
A related dynamic activity is \emph{animation}, which may play a role
in analysing an executable model.

Static \vv techniques---not executing model or code---include:
\begin{itemize}
\item Checking of coding guidelines
\item Review
\item Walkthrough
\item Formal methods
\begin{itemize}
\item Model checking
\item Deductive verification (theorem proving)
\item Abstract interpretation
\end{itemize}
\end{itemize}

%\begin{figure}[h]
%\centering
%\input{sections/openETCSOpenProofsDevelopmentProcess.pdf_tex}
%\caption{openETCS open proofs concept}
%\end{figure}

\todo{description on V\&V classification non formal-> formal -> formal
  -> code \& description}

\subsection*{Coping with a Model-Based Development Style}

Models appear at different stages of the development. An important
artifact of openETCS is a semi-formal model of the requirements. 
Depending on the modelling framework, the modelling language and
formalization of the system requirements a
concept has to be defined how the consistency and coherence of the
model as well as the coverage of system requirements will be
transparently verified. For this task, static verification techniques
will very likely offer the best approach.

To verify that the model correctly captures the ETCS system
requirement specification (Subset-026 et al.), also dynamic techniques
like animation might be useful. And finally, it may be helpful to also
validate the model against the user requirements.

For later development stages, correct refinement or implementation of
the model will have to be established. Again, techniques to be applied
depend heavily on the nature of the model(s) and the process of how
the code is derived. Model-based testing, i.e., deriving test cases
from a model to ascertain that an executable behaves consistent to a
model, is a technique to be used. Alternatively, if code is generated
automatically from a model, other means like tools checking the
correctness of a generation procedure (or its outcome on a
case-by-case basis) may be chosen.

An important issue to be kept in mind is the suitability of models
and tools for a safety-critical development. Modelling languages that
lack a formal semantics or the expressive power to capture system
aspects essential for safety considerations are of limited
usefulness. And tools need to be qualifiable according to their role.  
For instance, a code generator needs to be verified or qualified or it
must be accompanied by some tool checking the correctness of the
generation step. Otherwise, the resulting code will have to verified
similar as manually written code.



