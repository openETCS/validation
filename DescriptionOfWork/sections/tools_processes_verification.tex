%-----------------------------------------------------------------------
\section{Verification of the Tools and Processes}
%-----------------------------------------------------------------------
\tbc
The software will be developed according to the guidelines specified in the CENELEC Standard 50128. Each of the Lifecycle stages (SW Requirement Specification, SW Design, SW Coding, etc.) must be fully documented and simultaneously verification and validation tasks must be performed.
In this task the safety management team draws a safety plan to identify the safety management structure, safety related activities and safety approval milestones. A hazard log will be created and maintained throughout the whole development process. In addition, the safety plan will include plans for verifying that each development phase meets its safety requirements. The safety plan also describes (among others):
\begin{enumerate}
\item  Roles, responsibilities and competences of the involved bodies
\item  Safety-related deliverables with milestones
\item  Procedures of preparing the safety case
\item  Procedures for maintaining safety documents
\end{enumerate}
All safety principles followed in the development process will be described along with documented quantitative analyses. Evidences of technical safety shall describe the safeguards used for individual safety properties. The V\&V reports are to be referred in this part.
Concerning the applied tools, 3rd parties may be engaged to perform the V\&V. Certainly, the respective results will be referred to in the safety case.


%Specify the tool classes and the potentials to Categorize which tools have to be T3 and which T2.}

%Safety Evaluation criteria

\begin{table}[h]
\caption{T4.4 Inputs, Outputs and Deliverables} %title of the table
\begin{adjustbox}{width=\textwidth}
\begin{tabular}{|l|l|r|r|r|}
\hline
\multicolumn{5}{|c|}{\textbf{T4.4 Verification of the Tools and Processes}} 
\\\hline
Type & Description & Due Date & Due Month & status 
%status output going to other tasks/wps    : not started, started, complete
%status input coming from other tasks/wps: no, yes
%\\\hline
%$\rightarrow$ & \todo{Ox.2.3: Sample Input Information}  & \shortmonthname[1]-2014 & T0+19 & no 
%\\\hline
%$\leftarrow$ & \todo{O4.2.1: Sample Output Information}   & \shortmonthname[10]-2013  & T0+16 & started  
\\\hline
D & \emph{D 4.4} Final report concerning the Safety Case  & \shortmonthname[6]-2015 & T0+36 & \tbd
\\\hline
\end{tabular}
\end{adjustbox}
\end{table}