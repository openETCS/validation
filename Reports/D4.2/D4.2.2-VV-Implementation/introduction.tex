
\chapter{Introduction}

In this intermediate report we describe the activities to formally verify
the correctness of parts of the software developed in the OpenETCS project.

While major parts of the functionality of {Subset 026} are modelled in 
higher-level languages, there is also a substantial part of \emph{supporting} software
that is developed in the programming language~C.

\fxfatal{provide figure on scope of formal verification with OpenETCS}

In this document we report about \emph{preliminary} results on the verification
of that C-code.
In particular, we report on the use of static analysis methods (including formal methods)
on C code that has been developed by the project partner Siemens (Germany).
Figure~\ref{fig:Bitwalker-Overview} gives an overview on the software that
is in the focus of this report.

\begin{figure}[hbt]
\begin{center}
\includegraphics[width=0.8\textwidth]{figures/Bitwalker-Overview.pdf}
\caption{\label{fig:Bitwalker-Overview} The place of \texttt{Bitwalker} with the OpenETCS software}
\end{center}
\end{figure}

The OpenETCS decoder is a large collection of functions dedicated to
the reading of ETCS messages.
In order to fulfill their task these function rely on the relatively
small software package \inl{Bitwalker}.
The \inl{Bitwalker} software, as seen by the OpenETCS decoder,
is best understood as a ``class'' with a handful of methods.
Note that this class is implemented in~C as a \inl{struct} where
the methods are implemented as functions.
The core functionality  of this class, which consists in converting bit sequences to integers
and vice versa, depends on two more basic function, namely~\peek and~\poke.

This software has been analyzed by the OpenETCS project partners SQS (Spain)
and Fraunhofer FOKUS (Germany).
SQS used several static analysis tools to identify defects and to derive useful metrics.
Fraunhofer FOKUS, on the other hand, used the \framac tool set,
which is developed by the French project partner \cealist,
in order to \emph{formally verify} various properties of the \bitwalker.


These analyses contribute to the ultimate verification goals, which are the following:

\begin{enumerate}
\item provide evidence that the Bitwalker software satisfies 
      accepted quality standards
\item develop a formal specification for the Bitwalker software
\item verify that the Bitwalker software satisfies its formal specification
\item show that the Bitwalker software does not raise runtime errors
\item verify that OpenETCS decoder calls the Bitwalker software only
      according to its specification
\end{enumerate}

We are confident that all these verification goals can be reached.
For this preliminary verification report,
we provide partial answers to the first four topics.
In order to achieve the last goal, more development and verification
work is currently conducted by Fraunhofer ESK and Fraunhofer FOKUS. 

\section*{Structure of this document}

Chapter~\ref{sec:frama-c} gives a short overview on the \framacwp tool
that plays a central role in the verification of the Bitwalker functions.
Here we also try to rectify some misunderstandings about formal verification
that we have encountered in our work.

In Chapter~\ref{sec:formal-verification} we analyze
the functions \peek and \poke from the Bitwalker core and
\begin{enumerate}
\item formally specify the
      expected functional behavior in the \acsl specification language of {Frama-C}
      and
\item report on the formal proof 
	that these
      C~functions do not raise runtime errors when called according to their
      formal specification, established using 
      the {Frama-C} verification platform.
\end{enumerate}

So far only a part of Siemens' \bitwalker has been formalized and verified.
In the process of this work several enhancements for the \framac verification platform
have been identified and reported to the developers at {CEA LIST}.

In Chapter~\ref{sec:static-analysis}, we report about the results of
SQS' application of a broad range of static analysis tools on the \bitwalker. 
In contrast to \framac, these tools cannot exhaustively
detect all potential defects of a given kind.
Nevertheless, these they are very useful at finding well-known quality deficiencies that
might occur in C or \CC\ software.

In Chapter~\ref{sec:conclusions}, we draw conclusions from this preliminary work
and outline the next steps in our verification efforts.

