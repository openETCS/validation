

\chapter{ETCS data packets}
\label{sec:packets}

\section{Formal specification of \inl{AdhesionFactor}}
\label{sec:adhesionfactor}

\subsection{\inl{AdhesionFactor} in ETCS}
\label{sec:adhesionfactor-etcs}

\fxfatal{add table from chapter 7}

\subsection{The type \inl{AdhesionFactor}}
\label{sec:adhesionfactor-type}

Listing~\ref{lst:adhesionfactor-type} shows the definition of type
\adhesion as it is generated from the ETCS specification shown in Section~\ref{sec:adhesionfactor-etcs}.

\begin{listing}[hbt]
\begin{minipage}{0.99\textwidth}
\begin{lstlisting}[style=acsl-block]
struct AdhesionFactor
{
    PacketHeader header;

    // TransmissionMedia=Any
    // This packet is used when the trackside requests a change of
    // the adhesion factor to be used in the brake model.
    // Packet Number = 71

    uint64_t   Q_DIR;            // # 2
    uint64_t  L_PACKET;         // # 13
    uint64_t   Q_SCALE;          // # 2
    uint64_t  D_ADHESION;       // # 15
    uint64_t  L_ADHESION;       // # 15
    uint64_t   M_ADHESION;       // # 1
};

typedef struct AdhesionFactor AdhesionFactor;

\end{lstlisting}
\end{minipage}
\caption{\label{lst:adhesionfactor-type}Defintion of the type \adhesion}
\end{listing}

\FloatBarrier  % forces the output of listings/tables

\subsection{\acsl predicates \inl{AdhesionFactor}}

\subsection{Formal specification of \inl{AdhesionFactor_UpperBitsNotSet}}

\subsection{Formal specification of \inl{AdhesionFactor_DecodeBit}}

\subsection{Formal specification of \inl{AdhesionFactor_EncodeBit}}

\subsection{Formal verification of \inl{AdhesionFactor}}

\section{Formal specification of other packets}

