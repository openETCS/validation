%\subsection{System Integration  Testing (Uni Bremen/DLR)}

This section describes the verification plan of Uni Bremen. It
concerns the test generation for part of the on board unit as
specified in the subset -026. he main goal are :

\begin{enumerate}
\item Create a test model in SysML. From the model evaluation
  activities, the management of the radio communication is already
  available. To cover different aspect of the specification, the
  ceiling speed monitoring model will be also provided.
\item Generate test cases according to the defined interface given by
  DLR. (RT-Tester) 
\item Provide simulation environment for the
  track-to-train simulation (including braking curves/speed
  profiles)\footnote{ OpenETCS system testing for the EVC on-board
    computer requires test execution in real physical time and also
    track layout with realistic speed profiles.  We also want to
    contribute to the track and Speed Simulation by automatically
    generates ``relevant'' layout for OpenETCS.}  along routes used
  for testing
\item Study the automatic generation of these track layouts and speed
  simulations; if feasible, implement a generator and integrated it in
  the DLR laboratory environment
\item Set up a test environments for
  \begin{itemize}  
  \item Hardware-in-the-loop Testing within DLR laboratory.
  \item Software-in-loop testing with code provided by SCADE (Siemens)
  \end{itemize}
\end{enumerate}


\paragraph{Track simulation}
OpenETCS system testing for the EVC on-board computer requires test
execution in real physical time and also track layout with realistic
speed profiles.  We also want to contribute to the track and Speed
Simulation by automatically generates ``relevant'' layout for
OpenETCS.



\paragraph{Test cases generation}
We also plan different activities to ensure the pertinence of our test cases.
\begin{enumerate}
\item Check the test model (SysML) : RTT-BMC or other model checkers
\item Add relevant LTL properties if needed
\item Test case analysis by 
  \begin{itemize}
    \item Structural coverage
    \item Requirement coverage
    \item Mutation coverage 
    \item Data coverage
  \end{itemize}
\end{enumerate}

\paragraph{Test cases analysis -- comparison to Subset 76}
\begin{enumerate}
\item Provide techniques and Howto describing how test cases from
  Subset 76 can be executed in the RT-Tester environment, either as SW
  integration test or as HiL test in the DLR simulation environment
\item Create new set of test cases for the ceiling speed monitoring
  (As far as we know,they do not yet exist in Subset 76)
\item Compare new test cases created by RT-Tester to new test cases
  for ceiling speed monitoring provided by ERTMS standardization
  group, as soon as available; suggest improvements for the Subset 76
  test cases. 
 \end{enumerate}



\paragraph{Object of verification}
\nl
The object of verification is the speed and distance monitoring test model
(https://github.com/openETCS/validation/tree/master/VnVUserStories/VnVUserStoryUniBremen/05-Work/SpeedAndDistanceMonitoring)
and represents the chap 13.10.3 of the subset 026.



\paragraph{Available specification}

The specification is the subset 026 chap 3.13.10.1-3, 3.13.10.5-6.
It describes the ceilingand the target speed monitoring of the
train. It specifies when the braking commands should apply depending
on the speed and the position of the train.

\paragraph{Methods and Means}

Model based testing with RT-tester (see \label{sec:RTTester}) is used for the tests generation
from a test model design with Papyrus.



\paragraph{Results to be achieved}

The main goal is to obtain automatically generated test that cover
part of the subset 026 for the speed and distance monitoring that can
be then played by the DLR laboratory.
Furthermore, the tests should ensure some coverage criteria, and be
realistic enough to be sound and usable.

In order to have strong and sound test we need to constraints the test
generation. We can see two different approaches. One is to model some physical effect
such as realistic deceleration and acceleration embedded in the test
model.
The second approach consists of adding track layout information to the
model that will generate tests for a given track layout.


\paragraph{Timeline}


\begin{tabular}{lp{5cm}}
December 2014 & Create and improve & the Speed and distance monitoring test model\\
March 2014 &  Generate test conforms to the DLR specification\\
August 2014 & Study how to model stronger environment \\
September 2014 & Study how to generate test for a given track (or partial track) layout\\
September 2014 &  Implement the stronger environment for better test generation\\
November 2014 &  Implement test generation for a given track layout\\
January 2014&  Compare the generated tests with the different environment
  strategies.\\
 February 2015 & Define Test representation\\
 March 2015 & Represent Subset 076 to be compared with the
  generated tests\\
April 2015 & Write Reports \\
\end{tabular}
